\documentclass[a4paper]{article}
\usepackage[utf8]{inputenc} 
\pagestyle{headings} 
\usepackage[T1]{fontenc}
\usepackage{lmodern}  
\usepackage[ngerman]{babel}
\usepackage{graphicx}
\graphicspath{/home/Sebastian/Documents/pdfWork}
\DeclareGraphicsExtensions{.pdf,.jpeg,.png}

\pagestyle{headings}
\usepackage{multicol}
\usepackage{amsfonts}   
\usepackage{amssymb}
\usepackage{amsmath}
\usepackage{xcolor}
\usepackage{paralist}
\usepackage{ulem}
\usepackage{enumitem}
\usepackage{hyperref}
\usepackage{pdfpages}
\usepackage{bbm}

\usepackage[paper=a4paper,left=20mm,right=20mm,top=20mm,bottom=20mm]{geometry}

\newcommand*\laplace{\mathop{}\!\mathbin\bigtriangleup}
\newcommand*\dalembert{\mathop{}\!\mathbin\Box}
\newcommand{\intd}{\int\!\mathrm{d}}



\title{Klassische Feldtheorie}
\author{}
\date{}
\begin{document}
\maketitle
\begin{verbatim}






\end{verbatim}
\begin{abstract}
Das Skript zur Vorlesung klassische Feldtheorie im Wintersemester 2013/14. Die Vorlesung wurde von Professor Spiesberger gehalten.
\end{abstract}
%\renewcommand{\abstractname}{Abstract}
\newpage
\tableofcontents
\setcounter{tocdepth}{5}
\newpage

\section{Spezielle Relativitätstheorie}

Einstein 1905:\\
Zur Elektrodynamik bewegter Körper
%Gut zu verstehen, ev. verlinken.

\subsection{Postulate}
\subsubsection{1. Postulat}
Die Naturgesetze gelten in jedem
\underline{Inertialsystem} in der gleichen Weise.\\
Inertialsysteme: Bezugssysteme in denen ein sich kräftefrei bewegender Körper,
konstante Geschwindigkeit besitzt.
$\rightarrow$ verschiedene Inertialsysteme bewegen sich relativ zueinander mit
konstanter Geschwindigkeit.
\subsubsection{2.Postulat}
Prinzip der in jedem Inertialsystem konstanten Lichtgeschwindigkeit.\\
Prinzip: es gibt eine maximale Signalgeschwindigkeit\\
Wirkungsausbreitung = Lichtgeschwindigkeit $c=2.99*10^8\frac{m}{s}$\\
(Klassische Mechanik: Grenzfall: $c \rightarrow \infty$)\\
z.B. $\Pi^0 \rightarrow \gamma + \gamma$\\
$\rightarrow$ Geschwindigkeiten können nicht trivial addiert werden\\
$\rightarrow$ Zeit muss in der Speziellen Relativitätstheorie mit transformiert
werden\\
$\rightarrow$ 4-dimensionale Schreibweise einer vierdimensionalen Raumzeit.\\
Dazu spricht man in der speziellen Relativitätstheorie von einem Ereignis oder
Weltpunkt $(t,\vec{x})$ (Raum-Zeit-Punkte). Die physikalische Dimension wird zu
$x^0=ct$ normiert.\\
$\rightarrow x^{\mu}=(x^0,x^1,x^2,x^3)=(x^0,\vec{x})$\\
$\vec{x}=(x^1,x^2,x^3)\leftarrow x^i$\\
Aus der Konstanz der Lichtgeschwindigkeit in allen Koordinatensystemen folgt
das der 4-Abstand zweier Ereignisse a, b mit den Weltpunkten
$x^{\mu}_a$,$x^{\mu}_b$, die durch ein Lichtsignal verbunden sind, konstant
ist.
\begin{align}
S^2_{ab}&=(x^0_a-x^0_b)^2-\sum_{i=1}^3(x^i_a-x^i_b)^2 \\
&=c^2(t_a-t_b)^2-(\vec{x}_a-\vec{x}_b)^2
\end{align}
$S^2_{ab}$ ist für beliebige Ereignisse invariant beim Wechsel zwischen
Inertialsystemen.
\subsubsection{Definition der Ko- und Kontravarianten Vektoren}
Kontravariante Komponenten von 4er Vektoren: $x^{\mu \leftarrow}$\\ 
Kovariante Komponenten von 4er Vektoren :$x_{\mu \leftarrow}=g_{\mu
\nu}x^{\nu}$\\
$g_{\mu \nu} =$ Metrischer Tensor (Tensor 2. Stufe)
	     = diag(1,-1,-1,-1)\\
		  $g_{00}=1$ $g_{ii}=-1$\\

\paragraph{Einsteinsche Summenkonvention:} 
$\sum$ wird implizit über alle Indizepaare gebildet, bei denen ein Index oben
und der andere unten steht. Diese summe geht bei griechischen Buchstaben von 0
bis 3 und bei lateinischen Buchstaben von 1 bis 3
\begin{align}
&x_{\mu \leftarrow}=g_{\mu \nu}x^{\nu}=\sum_{\nu=0}^3g_{\mu \nu}x^{\nu}\\
&x^{\mu}=(ct,\vec{x}) \rightarrow x_{\mu}=(ct,-\vec{x})
\end{align}
damit lässt sich der 4-Abstand schreiben als
\begin{align}
S^2_{ab}&=g_{\mu \nu}(x_a-x_b)^{\mu}*(x_a-x_b)^{\nu}\\
&=\sum_{\mu}(x_a-x_b)^{\mu}\left(\sum_{\nu}g_{\mu \nu}(x_a-x_b)^{\nu}\right)\\
&=(x_a-x_b)^{\mu}*(x_a-x_b)_{\mu}
\end{align}
$\rightarrow$ Invarianz des 4-Abstandes?\\
$\Rightarrow$ 4-Abstand (Metrik)\\
für differentielle Abstände $\mathrm{d}x^{\mu}=(c\mathrm{d}t,\mathrm{d}\vec{x})$ gilt:
\begin{align}
\mathrm{d}S^2=\mathrm{d}x^2&=g_{\mu \nu}*\mathrm{d}x^{\mu}*\mathrm{d}x^{\nu}\\
&=c^2*\mathrm{d}t^2-\mathrm{d}\vec{x}^2
\end{align}
Hier ist die Metrik durch den metrischen Tensor gegeben.
\begin{align}
g_{\mu \nu}=\text{diag}(1,-1,-1,-1)
\end{align}
%XXXX Definition angeben \mathbb{R}^4\rightarrow\mathbb{R}^4,\
$g:\ x^{\mu} \mapsto x_{\mu}=g_{\mu
\nu}*x^{\nu}$ ist damit eine Abbildung.\\

\subsection{Geometrie}
Eigenzeit: wähle ein Inertialsystem K in dem ein Körper in Ruhe ist
$\mathrm{d}\vec{x}=0$, $\mathrm{d}t\neq0$, $\mathrm{d}x^0=c*\mathrm{d}t$ mit der Schreibweise $\mathrm{d}\tau$.\\
Im Koordinatensystem $K^{'}$ des Beobachters in dem sich der Körper bewegt gelte
nun $\mathrm{d}\vec{x}^{'}\neq0$, $\mathrm{d}t^{'}\neq0$\\
\subsubsection{Invarianz des 4-Abstands} 
\begin{align}
\mathrm{d}s^2&=c^2*\mathrm{d}\tau^2=c^2*(\mathrm{d}t^{'})^2-(\mathrm{d}x^{1'})^2-(\mathrm{d}x^{2'})^2-(\mathrm{d}x^{3'})^2\\
\Rightarrow
\mathrm{d}\tau&=\mathrm{d}t^{'}*\sqrt{1-\frac{(\mathrm{d}x^{1'})^2+(\mathrm{d}x^{2'})^2+(\mathrm{d}x^{3'})^2}{c^2
(\mathrm{d}t^{'})^2}}\\
&=\mathrm{d}t^{'}*\sqrt{1-\frac{1}{c^2}\left(\frac{\mathrm{d}|\vec{x^{'}}|}{\mathrm{d}t^{'}}\right)^2}\\
&=\mathrm{d}t^{'}*\sqrt{1-\beta^2}\text{ mit }\beta=\frac{v}{c}\leq1 \\
\mathrm{d}\tau&=\frac{\mathrm{d}s}{c}
\end{align}

für beliebige Bewegungen:
\begin{align}
\tau=\int \mathrm{d}t^{'}\sqrt{1-\beta^2}
\end{align}
\subsection{Lorentz-Transformationen}
Galilei-Trafo.: nicht-relativistische Kinematik
\begin{align}
t^{'}=t, & x^{'1}=x^1+v*t, x^{'2}=x^2,x^{'2}=x^2
\end{align}
für Inertialsysteme, die sich relativ zueinander mit konstanter Geschwindigkeit
v in $x^1$-Richtung bewegen.\\
in der SRT muss die zeit mit transformiert werden:
\begin{align}
x^{\mu}\rightarrow x^{'\mu}=\Lambda^\mu_\nu x^\nu
\end{align}
$\Lambda^\mu_\nu$ aus der Forderung der Invarianz des 4er Abstandes
(0. $x^{\mu}\rightarrow x^{'\mu}=x^{\mu}+a^{\mu}, a^\mu=$konstant,
Transformation in Raum-Zeit\\
1. eine Lösung: $x^{'\mu}=x^{\mu}$, d.h. $\Lambda^\mu_\nu= \begin{pmatrix} 1 &
0& 0&0\\ 0& \\ 0& &  \mathbb{R} \\ 0 \end{pmatrix}$ \\
$\vec{x^{'}}= \mathbb{R}*\vec{x}$ wobei $ \mathbb{R}$ eine räumliche Drehung
darstellt also $ \mathbb{R}^T \mathbb{R}= 1$)\\

z.B. Drehung um $x^3$-Achse
\begin{align}
&\mathbb{R}=\begin{pmatrix} \cos\theta & \sin\theta & 0 \\ -\sin\theta
& \cos\theta &0 \\ 0&0&1\end{pmatrix}\\
&\vec{x^{'}}= \mathbb{R}*\vec{x},\vec{x^{'}}^2= \vec{x}^2\\
&\cos^2\theta+\sin^2\theta=1
\end{align}
z.B. Versuche
\begin{align}
\Lambda^\mu_\nu= \begin{pmatrix} \cosh\phi & \sinh\phi& 0&0\\ \sinh\phi&\cosh\phi
&0&0\\0&0 &1&0 \\0&0 &0&1 \end{pmatrix}\\
\cosh^2\phi - \sinh^2\phi=1
\end{align}
Dies entspricht einer Drehung mit einem Imaginären Drehwinkel.\\
im K-System $K$: Komponenten $x^\mu$\\
im K-System $K^{'}$: Komponenten $x^{'\mu}$\\
$K$ und $K^{'}$ bewegen sich relativ zueinander mit konstanter Geschwindigkeit
$v$, so dass $\beta=\frac{v}{c}=-\tanh\psi$\\
Ursprung von $K^{'}$: 
\begin{align}
x^{'\mu}&=0=\sinh\psi*x^0+\cosh\psi*x^1\\
\Rightarrow
\frac{x^1}{x^0}&=\frac{x^1}{c*t}=-\frac{\sinh\psi}{\cosh\psi}=-\tanh\psi\\
\tanh\psi&=-\beta \\
\sinh\psi&=-\frac{\beta}{\sqrt{1-\beta^2}}, &
\cosh\psi=-\frac{1}{\sqrt{1-\beta^2}}
\end{align}
Abkürzung: $\gamma=\frac{1}{\sqrt{1-\beta^2}}$
\begin{align}
\Rightarrow \Lambda^\mu_\nu= \begin{pmatrix} \gamma & \gamma \beta & 0&0\\
\gamma \beta & \gamma &0&0\\0&0 &1&0 \\0&0 &0&1 \end{pmatrix}
\end{align}
Diese spezielle Lorentz-Transformation wird "boost" genannt.\\
Sie ist hier jedoch ausschließlich in 1-Richtung. In 2-Richtung und 3-Richtung
werden einfach die entsprechenden Zeilen vertauscht. Andere Richtungen lassen
sich mit Drehungen realisieren. Beliebige Lorentz Transformationen der Form
$x^{'\mu}=\Lambda^\mu_\nu x^\nu$ können dargestellt werden als Produkt von
jeweils eine Lorentzboost und einer Drehung

\begin{align}
x^{'1}=\frac{x^1+vt}{\sqrt{1-\beta^2}}, x^{'2}=x^2, x^{'3}=x^3,
t^{'}=\frac{t+\frac{\beta}{c}*x^1}{\sqrt{1-\beta^2}}
\end{align}

$\Rightarrow$ Längenkontraktion, Zeitdilatiation\\
Ansatz: 4-Abstand ist invariant\\
differentiell: $\mathrm{d}s^2=(\mathrm{d}s^{'})^2$\\
aus Definition folgt: $\mathrm{d}s^2=c^2\mathrm{d}t^2-\mathrm{d}\vec{x}^2$\\
Lichtsignal: $\mathrm{d}s^2=0$
Ansatz: $\mathrm{d}s^{'}=a*\mathrm{d}s$\\
Aus der Homogenität und Isotropie des Raumes folgt,dass das $a$ von keiner
der Koordinaten abhängen kann. $\Rightarrow a(x,\frac{\mathrm{d}x}{\mathrm{d}t})=a(|\vec{v}|)$\\
Betrachte nun 3 Koordinatensysteme:
\begin{enumerate}
  \item $K_1$-System: ds
  \item $K_2$-System: relativ beweg zu $K_1$ mit $\vec{v_{12}}$
  \item $K_3$-System: relativ beweg zu $K_1$ mit $\vec{v_{13}}$
\end{enumerate}
$\Rightarrow$ $K_2$ relativ zu $K_3$ bewegt mit $\vec{v_{23}}$
\begin{align}
\mathrm{d}s \rightarrow \mathrm{d}s^{'}&=a(|\vec{v}_{12}|)*\mathrm{d}s (K_1 \rightarrow K_2)\\
\mathrm{d}s^{''}&=a(|\vec{v}_{13}|)*\mathrm{d}s (K_1 \rightarrow K_3)\\
\mathrm{d}s^{''}&=a(|\vec{v}_{23}|)*\mathrm{d}s^{'} (K_2 \rightarrow K_3)\\
\Rightarrow  a(|\vec{v}_{13}|)&=a(|\vec{v}_{12}|)*a(|\vec{v}_{23}|)
\end{align}
Da gilt $\vec{v}_{13}=\vec{v}_{12}+\vec{v}_{23}$ liegt eine Winkelabhängigkeit
bei der obigen Gleichung vor und damit muss $a$ eine Konstante sein, die wir per
Konverntion auf $a=1$ setzen.

\subsection{4-Vektoren}
4-komponentige Größen $A^\mu=(A^0,\vec{A})$ die sich bei einem Übergang von
einem Inertialsystem zum anderen nach:
\begin{align}
A^{\mu}\rightarrow A^{'\mu}=\Lambda^\mu{}_\nu A^\nu
\end{align}
transformieren, heißen 4-Vektor
$\rightarrow$ Kovariante 4-Vektor $A_\mu=g_{\mu \nu} A^\nu$\\
$\rightarrow$ Quadrat $A^2 = A_\mu A^\nu= g_{\mu \nu} A^\mu  A^\nu =
(A^0)^2-(\vec{A})^2$\\
$\Rightarrow$ $A^2$ sind invariant unter Lorentztransformationen d.h.
$A^{'2}=A^2$ und $\Lambda \Lambda g=g$
%XXXX
%XXXX Einfügen einiger weiterer Sachen
%XXXX


\section{Allgemeine Feldtheorie}

Skalarfelder $\Phi(x)$ mit $x$ 4-Vektor
Vektorfelder $V^\mu(x)$
Transformationsverhalten:
\begin{align}
\Phi '(x ')&=\Phi(x) \\
x^\mu {}'&=\Lambda^\mu{}_\nu x^\nu\\
V^\mu {}'(x)&=\Lambda^\mu{}_\nu V^\mu(x)
\end{align}
Beispiele:\\
für $\Phi$: Potential, Ladungsdichte\\
für $V^\mu(x)$: Stromdichte $\frac{\mathrm{d}x^\mu(\tau)}{\mathrm{d}\tau}$\\

\subsection{Differentialoperatoren}
(sillschweigend: Steitigkeit, Differenzierbarkeit ist implizit vorrausgesetzt)
Skalarfeld $\rightarrow$ Gradientenfeld
\begin{align}
\partial_\mu\Phi(x)&=\frac{\partial}{\partial x^\mu}\Phi(x)\\
&=(\frac{1}{c}\frac{\partial\Phi}{\partial t},\vec{\nabla}\Phi(x))\\
\partial^\mu\Phi(x)&=(\frac{1}{c}\frac{\partial\Phi}{\partial
t},-\vec{\nabla}\Phi(x))\\
\text{auch }(\vec{\nabla}\Phi(x))^i&=\frac{\partial}{\partial
x^i}=\partial^i=(grad\Phi)^i
\end{align}
Wobei in der letzten Gleichung $i$ eine Komponente selektiert.
\begin{align}
\text{Bew.: }(\partial_\mu\Phi(x)){}'&=\frac{\partial}{\partial x^\mu{}'}\Phi{}'(x{}')\\
&=\frac{\partial}{\partial x^\mu{}'}\Phi(x(x{}'))\\
&=\frac{\partial}{\partial x^\nu}\Phi(x)*\frac{\partial x^\nu}{\partial
x^\mu{}'}\\ 
&=(\Lambda^{-1})^\nu{}_\mu\frac{\partial}{\partial x^\nu}\Phi(x)\\
&=(\Lambda^{-1})_\mu{}^\nu\frac{\partial}{\partial x^\nu}\Phi(x)\\
\text{mit }Ü: \Lambda^T\Lambda&=\mathbbm{1}\\
\Rightarrow \Lambda^-1&=\Lambda^T
\end{align}
4-Divergenz $\partial_\mu V^\mu(x)$ ist eine skalare Größe
\begin{align}
\partial_\mu V^\mu(x)&=\frac{\partial}{\partial x_\mu} V^\mu(x)\\
&=(\frac{1}{c}\frac{\partial V^0(x)}{\partial t}+\vec{\nabla}*\vec{V})\\
&=\frac{1}{c}\frac{\partial V^0(x)}{\partial t}+\vec{\nabla}*\vec{V}
\end{align}
(3 Divergenz $\vec{\nabla}*\vec{V}=\text{div}\vec{V}$)\\
\subparagraph{d'Alembert Operator}
\begin{align}
\dalembert\Phi=\partial^\mu\partial_\mu\Phi=\frac{1}{c^2}\frac{\partial^2}{\partial
t^2} \Phi - (\vec{\nabla})^2\Phi
\end{align} 
Antisymmetrische Ableitung eines 4-Vekors
\begin{align}
F_{\mu\nu}&=\partial_\nu V_\mu-\partial_\mu V_\nu= \frac{\partial
V_\mu}{\partial x^\nu}-\frac{\partial V_\nu}{\partial x^\mu} \\
\Rightarrow F_{\mu\nu}&=-F_{\nu\mu}\\
z.B.: F_{0i}&=\partial_0 V^i-\partial_i V^0\\
&=\frac{1}{c}\frac{\partial V_i}{\partial t}-(\vec{\nabla}V_0)_i
\end{align}
und $F_{ij}$ ist dual $rot\vec{V}=\vec{\nabla} x \vec{V}$\\
\begin{align}
~hochF^k=\frac{1}{2}\epsilon^{klm}*F_{lm}=\epsilon^{klm}*\partial_l
V_m=(\vec{\nabla}x\vec{V})^k
\end{align}

\subsection{Integration in D=4}
in D=3 
\subsubsection{Volumenintegral} $\int_V \mathrm{d}V \phi(\vec{x})=\int \mathrm{d}x_1 \int \mathrm{d}x_2
  \int \mathrm{d}x_3 \phi(x_1,x_2,x_3)$ \\
  $\rightarrow in D4 $\\ $\int_G \mathrm{d}\Omega \phi(x)=\int \mathrm{d}x^0\int \mathrm{d}x^1 \int \mathrm{d}x^2
  \int \mathrm{d}x^3 \phi(x^0,x^1,x^2,x^3)$\\ mit dem Gebiet $G$ im 4-D Raum\\
  Tansformationsverhalten: \\ für $G=$ der gesamte Raum
  \\$\Omega'=\mathrm{d}x^0{}'\mathrm{d}x^1{}'\mathrm{d}x^2{}'\mathrm{d}x^3{}'=^?\mathrm{d}\Omega$\\$x^\mu{}'=\Lambda^\mu{}_\nu
  x^\nu$\\ 
  Substitution: Jacobi Determinante:\\ $J=\left|\frac{\partial
  x^\mu{}'}{\partial x^\nu}\right|=\left|\Lambda^\mu{}_\nu\right|=1$ \\ 
  $\Rightarrow \int \mathrm{d}\Omega'\ldots=\int J*\mathrm{d}\Omega\ldots=\int \mathrm{d}\Omega\ldots$
\subsubsection{Flächenintegral}
in D=3
\begin{align}
\int_F \mathrm{d}\vec{f}*\vec{V}(\vec{x})=\int \mathrm{d}\sigma \vec{n}*\vec{V(\vec{x})}
\end{align}
$\mathrm{d}\vec{f}=$inf.Flächenelement\\
$\mathrm{d}r=$Flächeninhalt\\
$\vec{n}=$Einheitsvektor$\perp$ auf $\vec{F}$\\
(Beispiel: Integral über Stromdichte $\rightarrow$ Fluß)\\
für Fläche: Parametrisierung $\vec{x}(s,t)$
%Grafik 2
Tangentenvektoren $\vec{e}_s=\frac{\partial\vec{x}}{\partial s}$ \\
$\vec{e}_t=\frac{\partial\vec{x}}{\partial t}$\\
Flächennormale in P: $\vec{n}=\vec{e}_s x \vec{e}_t$
Flächeninhalt $\propto \mathrm{d}s*\mathrm{d}t$\\
$\Rightarrow \mathrm{d}\vec{f}=\mathrm{d}s*\mathrm{d}t*\frac{\partial\vec{x}}{\partial
s}x\frac{\partial\vec{x}}{\partial t}$\\
$\int \mathrm{d}\vec{f}*\vec{V}=\int \mathrm{d}f_i*V_i=\int \mathrm{d}s \int \mathrm{d}t
(\frac{\partial\vec{x}}{\partial s}x\frac{\partial\vec{x}}{\partial t})_i *
V_i$\\
$\mathrm{d}f_i=\mathrm{d}s\mathrm{d}t\epsilon^{ijk}\frac{\partial x_j}{\partial
s}x\frac{\partial x_k}{\partial t}$\\
$\mathrm{d}f_{jk}=\mathrm{d}s \mathrm{d}t \left(\frac{\partial x_j}{\partial
s}x\frac{\partial x_k}{\partial t}-\frac{\partial x_k}{\partial
s}x\frac{\partial x_j}{\partial t}\right)$\\
$\mathrm{d}f^i=\frac{1}{2}*\epsilon^{ijk} \mathrm{d}f_{jk}$\\
Integration über eine zweidimensionale Fläche in D=4\\
Flächenelement: $\mathrm{d}f^{\mu\nu}=\mathrm{d}s \mathrm{d}t \left(\frac{\partial x^\mu}{\partial
s}\frac{\partial x^\nu}{\partial t}-\frac{\partial x^\nu}{\partial s}
\frac{\partial x^\mu}{\partial t} \right)$\\
%mit ~ drüber
dazu dual:
$\mathrm{d}\tilde{f}^{\mu\nu}=\frac{1}{2}\epsilon^{\mu\nu\rho\sigma}\mathrm{d}f_{\rho\sigma}$\\
Flächenintegral:$\int_F \mathrm{d}\tilde{f_{\mu\nu}} *I*\ldots$\\
Das Verhalten der durch das Integral definierten Größe ergibt sich nach:\\
$\int_F \mathrm{d}\tilde{f}_{\mu\nu} *I^{\mu\nu}=[Pseudo]Skalar$
Falls $I$ eine skalare Funktion ist bekomen wir einen Tensor 2.Stufe als
Ergebnis des Integrals.\\
\subsubsection{Integration über 3-Dimensionale Hyperflächen}
benötigt Parametrisierung in $(s,t,u)$ d.h. $x^\mu(s,t,u)$\\
Tangentenvektoren $e^\mu_s=\frac{\partial x^\mu}{\partial s}$,
$e^\mu_t=\frac{\partial x^\mu}{\partial t}$, $e^\mu_u=\frac{\partial
x^\mu}{\partial u}$\\
in D=3: $(\vec{e}_s x \vec{e}_t) * \vec{e}_u$\\
in D=4: $\mathrm{d}S^\mu\nu\rho=\begin{pmatrix} 
\mathrm{d}x^\mu_s & \mathrm{d}x^\nu_s & \mathrm{d}x^\rho_s \\
\mathrm{d}x^\mu_t & \mathrm{d}x^\nu_t & \mathrm{d}x^\rho_t \\ 
\mathrm{d}x^\mu_u & \mathrm{d}x^\nu_u & \mathrm{d}x^\rho_u \\\end{pmatrix}
$
z.B.: $\mathrm{d}x^\mu_s=\mathrm{d}s\frac{\partial x^\mu}{\partial s}$ usw.\\
dual zu $\mathrm{d}S^{\mu\nu\rho}: \mathrm{d}S^\mu~=\frac{1}{6}
\epsilon^{\mu\nu\rho\sigma}*\mathrm{d}S_{\nu\rho\sigma}$\\
$\int_H \mathrm{d}S^\mu~ I_\mu$\\
$\mathrm{d}S^\mu~$ ist 4-Vektor
$\perp$ auf allen Richtugen in H
z.B. Hyperfläche $\mathbb{R}^3$
Parametrisierung: $s=x^1,t=x^2,u=x^3$\\
$\mathrm{d}S^{\mu\nu\rho}\rightarrow \mathrm{d}x^1\mathrm{d}x^2\mathrm{d}x^3$
$\mathrm{d}S^{0}~=\mathrm{d}x^1\mathrm{d}x^2\mathrm{d}x^3$


\subsubsection{Wegintegrale}
\begin{itemize}
  \item in $D=3$
\end{itemize}
\begin{align}
\int_C \mathrm{d}\vec{x}*\vec{V}(\vec{x})
\end{align}
Parametrisierung von C:
\begin{align}
\vec{x}=\vec{x}(\sigma)\\
\frac{\partial\vec{x}}{\partial\sigma}\\
\Rightarrow\int_C \mathrm{d}\sigma\\
\frac{\partial\vec{x}}{\partial\sigma}*\vec{V}(\vec{x(\sigma)})
\end{align}

\begin{itemize}
  \item in $D=4$
\end{itemize}
\begin{align}
\int_C \mathrm{d}x^\mu*V_\mu(x)
\end{align}
Parametrisierung von C:
\begin{align}
x^\mu=x^\mu(\sigma)\\
\frac{\partial x^\mu}{\partial\sigma}\\
\Rightarrow\int_C \mathrm{d}\sigma \frac{\partial x^\mu}{\partial\sigma}*
V^\mu(x(\sigma))\\
\sigma\rightarrow s \rightarrow\tau
\end{align}

\subsubsection{Integralsätze}
\begin{itemize}
  \item Gaußscher Satz
\end{itemize}
in 3-D
\begin{align}
\int_V \mathrm{d}^3x \vec\nabla \vec{V} = \int_{\partial V} \mathrm{d}\vec{f} \vec{V}
\end{align}
mit gebenenem Volumen $V$ und dessen Rand $\partial V$

in 4-D
\begin{align}
\int_G \mathrm{d}\Omega \partial_\mu V^\mu = \int_{\partial G} \mathrm{d}\tilde{S}_\mu V^\mu
\end{align}

\begin{itemize}
  \item Satz von Stokes
\end{itemize}

in 3-D:
\begin{align}
\int_F \mathrm{d}\vec{f} (\vec{\nabla}x\vec{V}) = \int_{\partial F} \mathrm{d}\vec{x} \vec{V}(x)
\end{align}

in 4-D
\begin{align}
\int_F \mathrm{d}f^{\mu\nu} \partial_\mu V_\nu= \int_F \mathrm{d}f^{\mu\nu}
\frac{1}{2}(\partial_\mu V_\nu-\partial_\nu V_\mu) = \int_{\partial F}
\mathrm{d}x^\mu V_\mu
\end{align}

Verallgemeinerungen:\\
%Mit Pfeil statt Punkt kennzeichen XXXX
\begin{itemize}
  \item Greensche Integralsätze
  \item Beziehung für Integrationen für D=2 und D=3  
\end{itemize}

\subsection{Relativistische Kinematik}
Kräftefreie Bewegung eines (punktförmigen) Teilchens:
%Mit Pfeil statt Punkt kennzeichen XXXX
\begin{itemize}
  \item Impuls, Energiedefinition
\end{itemize}

\subsubsection{Lagrangeformalismus}
Prinziep der minimalen Wirkung:
\begin{enumerate}
  \item generalisierte Koordinten finden ($q_i(t)$) (Ziel ist es die
  Zeitabhängigkeit der Koordinaten zu bestimmen.)
  \item verallgmeinerte Geschwindigkeiten
  ($\dot{q}_i(t)=\frac{\partial}{\partial t}q_i(t)$)
  \item Lagrangefunktion L($q_i$,$\dot{q}_i$,$t$) aufstellen
  \item Wirkung aufstellen $S[q_i(t)]$ (Funktional)
  \begin{align}
  S[q_i(t)]=\int_{t_a}^{t_b} \mathrm{d}t L(q_i(t),\dot{q}_i(t),t)
  \end{align}
  zu gegebenen Anfangs- und Endzeitpunkten $t_b$,$t_a$ kann das Funktional für
  beliebige Wege, auch für die physikalisch nicht relisierten, berechnet werden.
  \item Prinziep der kleinsten Wirkung besagt, dass die tatsächlich pysikalisch 
  realisierte Bewegung erfolgt auf Bahnkurve, für die $S[q_i(t)]$ minimal ist
  $\Rightarrow$ Variationsrechnung 
  \begin{align}
  q_i(t)\rightarrow q_i(t)+\delta q_i(t)\\
  \dot{q}_i(t)\rightarrow \dot{q}_i(t)+\delta \dot{q}_i(t)\\
  \Rightarrow \underline{\delta S = 0}
  %Unterstreichen als Ergebnis
  \end{align} 
\end{enumerate}
$\Rightarrow$ Euler-Lagrange-Gleichungen

\subsubsection{relativistische Kinematik}
punktförmiges kräftefreies Teilchen\\
Bahnkurve: $x^\mu(\sigma)$ mit $\sigma \rightarrow s,\tau$, da die Zeit in den
4-Vektoren $x^\mu$ bereits enthalten ist und eine andere sinnvolle
Parametrisierung nötig ist.\\
Wirkung ???\\
$S[x^\mu(\sigma)]$ 
Diese muss die folgenden Bedingungen erfüllen:
\begin{enumerate}
  \item{Lorentzinvarianz} $S[x^\mu(\sigma)]$ ist Skalar im engeren Sinn
   bezüglich der Lorentzinvarianz, so dass es möglich wird die
   Bewegungsgleichungen vom Bezugssystem unabhängig her zu leiten.
  \item{Einfachheit}Außerdem muss $S[x^\mu(\sigma)]$ einfach
   sein, so dass es auch mögich bleibt die Bewegungsgleichungen mit den nötigen
   Anfangsbedingen auch eindeutig zu lösen. $\rightarrow$ DGl. 2. Ordnung
  \item{Physikalische Erfharung} Die Bewegungsgleichungen sollten mit der
  Physikalischen Erfahrung (d.h. mit den Messwerten der Experimente)
  überinstimmen
\end{enumerate}
$\Rightarrow$ Ansatz: $S=-\alpha\int_a^b \mathrm{d}S$ mit $a,b\hat{=}$Anfangs- und
Endpunkte\\
Einfachste Möglichkeit: 
\begin{align}
\mathrm{d}s&=\sqrt{\mathrm{d}x_\mu \mathrm{d}x^\mu}\\
\mathrm{d}s&=c\mathrm{d}t\sqrt{1-\beta^2}\\
S&=-\alpha c \int_a^b \mathrm{d}t \sqrt{1-\beta^2}
\end{align}
$\Rightarrow$ Lagrangefunktion: 
\begin{align} 
L&=-\alpha c \sqrt{1-\beta^2} \\
&\text{mit }\beta=\frac{v}{c}\\
L \overset{\substack{\text{klass.}\\
\text{Grenzfall}}}{\underset{{v \ll c}}{\longrightarrow}}& 
-\alpha c \left(1-\frac{1}{2}\beta^2+\ldots\right) \Rightarrow L=-\alpha c +
\frac{\alpha}{2 c} v^2 + \ldots
\end{align}
Wobei die Konstante $-\alpha c$ keine Relevanz für die Minimierung des Problems
aufweist.
\begin{align}
\Rightarrow \frac{\alpha}{2 c}&=\frac{m}{2} \ \ \Rightarrow \alpha=m c \\
\Rightarrow S&=-mc\int^b_a \mathrm{d}s \\
L&=-mc^2\sqrt{1-\beta^2}
\end{align}
\subparagraph{Impuls}
\begin{align}
p^i=\frac{\partial L}{\partial v_i}, v_i=\dot{q}_i
\end{align}
relativistisch folgt
\begin{align}
\frac{\partial L}{\partial v}&=-mc^2\frac{1}{2\sqrt{1-\beta^2}}
*(-2\frac{v}{c^2})=\frac{m\vec{v}}{\sqrt{1-\beta^2}}\\
\Rightarrow
\vec{p}&=\frac{m\vec{v}}{\sqrt{1-\beta^2}}\underset{v \ll c}{\longrightarrow}
m\vec{v} + O(v^3)
\end{align}
\subparagraph{Energie}
Herleitung aus der Aussage, dass die Energieerhaltung und auch die Energie als
Größe von der Translationsinvarianz der Energie bezüglich der Zeit resultiert.
\begin{align}
E&=m\vec{v}-L=\frac{\partial L}{\partial \dot{\vec{x}}}*\dot{\vec{x}}-L\\
&=\frac{mv^2}{\sqrt{1-\beta^2}}+mc^2\sqrt{1-\beta^2}=\frac{m}{\sqrt{1-\beta^2}}(v^2+c^2(1-\beta^2))\\
\Rightarrow E&=\frac{mc^2}{\sqrt{1-\beta^2}}
\end{align}
$\rightarrow$ Ruheenergie:
\begin{align}
E\underset{v=0}{\longrightarrow}E_0&=mc^2\\
E\underset{v \ll c}{\longrightarrow}E_0&=mc^2+\frac{1}{2}mv^2+\ldots
\end{align}
Womit sich der klassische Term $\frac{1}{2}mv^2$ für die kinetische Energie
ergibt.\\
$\rightarrow $ Hamiltonfunktion\\
$E$ als Funktion des Impulses ergibt für die Hamiltonfunktion
\begin{align}
H=c\sqrt{p^2+m^2c^2}\\
\underset{\substack{v\ll c\\p \ll mc}}{\longrightarrow}
mc^2+\frac{p^2}{2m}=mc^2+E_{kin.,klass.}
\end{align} 
Bemerkung:\\
Wenn das Teilchen masselos ist gilt die folgende Approximation $E=cp$, die aus
dem Grenzwert der Hamiltonfunktion folgt ($\rightarrow$ Photonen)
\subparagraph{4-Impuls}
\begin{align}
&p^\mu=mcu^\mu\\
&p^0=\frac{E}{c}\\
\Rightarrow &p^2=m^2c^2\\
\nonumber&\text{Energie-Impuls-Beziehung}\\
\nonumber&\text{Massenschalenbedingung}\\
&(p^0)^2-(\vec{p})^2=m^2c^2
\end{align}

\subparagraph{Drehimpuls}
\begin{align}
\text{nicht-relativistisch: } &\vec{M}=\vec{r}x\vec{p}\\
\text{relativistisch: } &M^{\mu\nu}=\frac{1}{2}(r^\mu p^\nu-r^\nu p^\mu)\\
\rightarrow &M^{ij}\rightarrow \tilde{M}^k \\
&\uparrow \text{ist dual zu } \vec{M}
\end{align}
allgemeiner:
\begin{align}
\text{Symmetrien} &\leftrightarrow \text{Erhaltungsgröße}\\
\text{räumliche Translationsinvarianz} &\leftrightarrow \text{Impuls}\ \vec{p}\\
\text{zeitliche Translationsinvarianz} &\leftrightarrow \text{Energie}\ cE\\
%XXXX Hinzufügen der Klammerung über Punkt 2 und 3 für rel. Impulserhaltung
\text{räumliche Drehungen} &\leftrightarrow \text{Drehimpuls}\ \vec{M}
\end{align}
räumliche Drehungen: für infinetisemale Drehwinkel gilt:
\begin{align}
x^\mu \rightarrow x^{'\mu}&=x^\mu+\delta\Omega^{\mu\nu}x_\nu \\
\text{invarint } x^2&=(x')^{2}\\
(x')^2&=(x)^2+2\delta\Omega^{\mu\nu}x_\mu x_\nu\\
\Rightarrow &\delta\Omega^{\mu\nu} \text{ ist antisymmetrisch}\\
&\text{d.h. } \delta\Omega^{\mu\nu}=-\delta\Omega^{\nu\mu}\\
\delta S&=-p_\mu \delta x^\mu\\
&=-p_\mu \delta\Omega^{\mu\nu} x_\nu\\
&=-\delta\Omega^{\mu\nu} \frac{1}{2}(p_\mu x_\nu-p_\nu x_\mu)\\
&=-\delta\Omega^{\mu\nu}M_{\mu\nu}
\end{align}
Interpretation: $M^{0i}$ (System von Teilchen und deren Schwerpunkt)

\subsection{Allgemeine Feldtheorie}
In der nicht relativistischen Physik sind Felder nichts anderes als Hilfsgrößen
die eine Kraftwirkung auf bestimmte Teilchen ausüben und zur Beschreibung dieser
Kräfte dienen.\\ 
In der relativistischen Physik führt jede Bewegung von Teilchen zu einer
änderung des von ihm verursachten Feldes (analog zur klassischen Physik).
Durch die endliche Signalgeschwindigkeit ist die zeitliche Änderung des Feldes
an einem anderen Ort jedoch nicht trivial. $\rightarrow$ Bewegungsgleichungen
für Felder (Feldgleichungen).\\
$\rightarrow$ Felder: eigenständige, dynamische Größen
\paragraph{Wirkung}
\begin{enumerate}
  \item Die Wirkung muss einen Term enthalten der sich auf die Bewegung der Teilchen
bezieht ($x^\mu,u^\mu,\ldots$)
  \item Term für Wechselwirkung von Teilchen mit Feldern
  \item Term für die "freie"\ Bewegung der Felder 
\end{enumerate}
% Hinzufügen von "Koordinaten" der Felder? mit Klammer über die letzten beiden
% Punkte XXXX
\underline{Postulat}: Es gibt ein 4-Vektorpotential $A^\mu(x)$\\
empirische Erfahrung:
\begin{itemize}
  \item Teilchen sind durch eine \underline{Ladung} gekennzeichnet mit der
    Elementarladung $e$ ($e>0$) \\
	Teilchen tragen (ganzzahlige) Vielfache $q_i$ von $e$ (Quarks sind ausgenommen,
	da sie sich nicht direkt beobachten lassen, bei ihrer Einbeziehung stellt sich
	die Frage warum sie gerade drittelzahlige Ladungen tragen)
  \item Zur Beschreibung der elektromagnetischen Wechselwirkung werden
  benötigt:\\
  \begin{align}
  	(\vec{E},\vec{B}\underset{\substack{\text{Maxwell-}\\
    \text{gleichungen}}}{\leftarrow}\phi \vec{A}) \Rightarrow A^\mu
  \end{align}
\end{itemize}
\subparagraph{4-Potential}
\begin{align}
A^\mu(x)&=A^\mu(t,\vec{x})\\
&=(\underbrace{\phi(t,\vec{x})}_{\text{Skalar}},\underbrace{\vec{A}(t,\vec{x}))}_{
\substack{\text{Vektor (bez.} \\\text{räumlicher}\\\text{Drehungen)}}}
\end{align}
\paragraph{Wirkung?\newline}
Zunächst für 1) und 2), Teilchen mit Wechselwirkung\\
"Koordinaten": $x^\mu$, $A^\mu$\\
Kriterien
\begin{enumerate}
  \item Lorentzinvariant
  \item einfach (aus $\delta\!S \Rightarrow$ einfachte DGL.)
  \item physikalische Realität
\end{enumerate}
\begin{align}
  \Rightarrow S_{WW}&=-\frac{e}{c}\int_a^b \mathrm{d}x_\mu A^\mu(x)\\
  e&=\text{Elementarladung}
\end{align}
(Für Teilchen der Ladung $+e$)\\
Die Wirkung für ein Teilchen der Ladung $e$ im elektromagnetischen Feld ist:
\begin{align}
S&=\int^b_a\left(-mc\mathrm{d}s-\frac{e}{c}A^\mu \mathrm{d}x_\mu\right)\\
\Rightarrow S&=\int^b_a\left(-mc\mathrm{d}s-e\phi \mathrm{d}t +\frac{e}{c}\vec{A}\mathrm{d}\vec{x}\right)\\
\Rightarrow S&=\int^{t_b}_{t_a}\left(-mc^2\sqrt{1-\beta^2}-e\phi
+\frac{e}{c}\vec{A}\vec{v}\right) \mathrm{d}t\\
\nonumber\Rightarrow \text{Lagrangefunktion:}&\\
L&=-mc^2\sqrt{1-\beta^2}-\underbrace{e\phi+\frac{e}{c}\vec{A}\vec{v}}_{\substack{\text{Term
für}\\\text{Wechselwirkung}}}\\
\rightarrow \text{kanonischer Impuls: } &\frac{\partial L}{\partial \vec{v}}\\
\Rightarrow \vec{P}=\frac{\partial L}{\partial
\vec{v}}&=\underbrace{\frac{m\vec{v}}{\sqrt{1-\beta^2}}}_{\text{kin. Impuls }
\vec{p}}+\frac{e}{c}\vec{A}
\end{align} 
Hamiltonfunktion:
\begin{align}
H&=\vec{v}\frac{\partial L}{\partial \vec{v}}-L\ |\ \text{in}\ \vec{P}\\
\Rightarrow
H&=\underbrace{\frac{mc^2}{\sqrt{1-\beta^2}}}_{\substack{\text{kinetische}\\
\text{Energie}}}+ \underbrace{e\phi}_{\substack{\text{potentielle}\\
\text{Energie}}}
\end{align}
Dies ermöglicht die folgenden Ersetzungsvorschriften:
\begin{align}
H &\rightarrow H-e\phi\\
\vec{P} &\rightarrow \vec{P}-\frac{e}{c}\vec{A}\\
\Rightarrow H&=\sqrt{m^2c^4+c^2\left(\vec{P}-\frac{e}{c}\vec{A}\right)^2}+e\phi
\end{align}
Bewegungsgleichung:
\begin{align}
\frac{\mathrm{d}}{\mathrm{d}t}\frac{\partial L}{\partial \vec{v}}-\frac{\partial L}{\partial
\vec{x}}&=0\\
\text{berechne }\ \ \ \ \  \frac{\partial
L}{\partial\vec{x}}&=-e\vec{\nabla}\phi+ \frac{e}{c}\vec{\nabla}(\vec{A}\vec{v})\\
&=-e\vec{\nabla}\phi+\frac{e}{c}\left(
(\vec{v}\vec{\nabla})\vec{A}+\vec{v}\times(\vec{\nabla}\times\vec{A}) \right)\\
\rightarrow \frac{\mathrm{d}}{\mathrm{d}t}(\vec{p}+\frac{e}{c}\vec{A})&=-e\vec{\nabla}\phi
\frac{e}{c}\left( (\vec{v}\vec{\nabla})\vec{A}+\vec{v}\times(\vec{\nabla}\times\vec{A})
\right)\\
\frac{\mathrm{d}}{\mathrm{d}t}\vec{A}&=\frac{\partial}{\partial t}\vec{A}+
\frac{\partial\vec{A}}{\partial x^i}\frac{\mathrm{d}x^i}{\mathrm{d}t}\\
&=\frac{\partial}{\partial t}\vec{A}+(\nabla*\nabla)\vec{A}\\
\Rightarrow
\frac{\mathrm{d}}{\mathrm{d}t}\vec{p}&=-e\vec{\nabla}\phi+\frac{e}{c}\vec{v}
\times(\vec{\nabla}\times\vec{A}) -\frac{e}{c}\frac{\partial}{\partial t}\vec{A}\\
\text{Def.: }\vec{E}&=-\nabla\phi-\frac{1}{c}\frac{\partial}{\partial
t}\vec{A}\\
\vec{B}=\nabla \times \vec{A}
\Rightarrow \frac{\mathrm{d}\vec{p}}{\mathrm{d}t}&=e\vec{E}+\frac{e}{c}\vec{v}\times\vec{B} 
\text{ Lorentz Kraft}
\end{align}

Einsortieren:
\begin{align}
\vec{E}&=-\vec{\nabla}\phi-\frac{1}{c}\frac{\partial}{\partial t}\vec{A}
\text{, Elektrisches Feld}\\
\vec{B}&=\vec{\nabla}\times\vec{A} \text{, magn. Feld (magnetische Induktion)}
\end{align}
(Magnetfeldsärke für $\vec{H}$ in Materie dielektrische Verschiebung $\vec{D}$
in Materie)\\
zunächst betrachten wir Teilchen im Vakuum:\\
$\vec{E}=\vec{D}$ und $\vec{B}=\vec{H}$ (Wahl des Gaußschen
\underline{Einheitensystems})
\subsubsection{Eichinvarianz}
$\vec{E}$,$\vec{B}$ sind messbar\\
$A^\mu=(\phi,\vec{A})$ ist nicht eindeutig\\
4-Potentiale $A^\mu_i$ die zu gleichen $\vec{E}=\vec{D}$ und $\vec{B}=\vec{H}$
führen sind physikalisch äquivalent.\\
\underline{Eichtransforation} $A_\mu\rightarrow A_\mu'=A_\mu-\partial_\mu X$ mit
einer fast beliebigen Funktion $X(x)$, d.h.:\\
\begin{align}
\phi &\rightarrow \phi'=\phi-\frac{1}{c}\frac{\partial}{\partial t} X\\
\vec{A} &\rightarrow \vec{A}'=\vec{A}-\vec{\nabla}X\\
\vec{E}&\rightarrow\vec{E}+\vec{\nabla}\frac{1}{c}\frac{\partial}{\partial t}
X-\frac{1}{c}\frac{\partial}{\partial t} \vec{\nabla}X=\vec{E}\\
\vec{B}&\rightarrow\vec{B}+\vec{\nabla}x(\vec{\nabla}X)=\vec{B}
\end{align}
\subsubsection{noch mal Bewegungsgleichungen}
aus $\delta S=0$ bez. Variation $x^\mu(s)\rightarrow x^\mu(s)+\delta x^\mu(s)$
\begin{equation}
\delta S=\delta \int^b_a(-mc\mathrm{d}s-\frac{e}{c}A^\mu \mathrm{d} x_mu)=0
\end{equation}
\begin{itemize}
  \item{1.Term} \begin{align}\mathrm{d}s=\frac{\mathrm{d} x^\mu}{\mathrm{d}
  s}\mathrm{d} \delta x^\mu
  \end{align}
  \item{2.Term} \begin{align}x^\mu&\rightarrow x^\mu+\delta x^\mu\\
  \delta(A_\mu \mathrm{d}\!x^\mu)&=A_\mu \mathrm{d}\delta\!x^\mu+\delta
  \underline{A_\mu} \mathrm{d}x^\mu\\&=a+b\\
  \delta A_\mu(x)&=A_\mu(x+\delta x)-A_\mu(x)=\partial_\nu A_\mu \delta x^\nu
  \end{align}
\end{itemize}
Variation $\delta \mathrm{d}x^\mu$ im Integral\\ 
a):
\begin{align}
&-\frac{e}{c}\int^b_a A_\mu(x(s))\frac{\mathrm{d}\delta x^\mu}{\mathrm{d}s}\mathrm{d}s\\
=&\frac{e}{c}\int^b_a \frac{\mathrm{d}}{\mathrm{d}s}A_\mu(x(s))\delta x^\mu(s) \mathrm{d}s +
\text{Randterme}\\
=&\frac{e}{c}\int^b_a \partial_\nu A_\mu \frac{\mathrm{d}x^\nu}{\mathrm{d}s}\delta x^\mu \mathrm{d}s\\
=&\frac{e}{c}\int^b_a \partial_\nu A_\mu u^\nu\delta x^\mu \mathrm{d}s
\end{align}
b):
\begin{align}
-\frac{e}{c}\int^b_a \partial_\nu A_\mu u^\mu\delta x^\nu \mathrm{d}s
\end{align}

Daraus folgt:
\begin{align}
\delta S &= \int_a^b \mathrm{d}s \delta x^\mu \left( mc\frac{\mathrm{d}u_\mu}{\mathrm{d}s} +
\frac{e}{c} \left[ \partial_\nu A_\mu u^\nu - \partial_\mu A_\nu u^\nu
\right]\right)\\
&=\int_a^b \mathrm{d}s \delta x^\mu \left( mc\frac{\mathrm{d}u_\mu}{\mathrm{d}s} -
\frac{e}{c} \left[ \partial_\mu A_\nu - \partial_\nu A_\mu \right] u^\nu
\right)
\end{align}
Kovariante Form der Bewegungsgleichung:
\begin{align}
mc\frac{\mathrm{d}u_\mu}{s}=\frac{e}{c}F_{\mu\nu}u^\nu
\end{align}
Mit Felstärketensor(antisym.)
\begin{align}
F_{\mu\nu}=\partial_\mu A_\nu - \partial_\nu A_\mu\\
F_{\mu\nu}=F_{\mu\nu}(x)
\end{align}
Komponentenweise: $F_{\mu\nu}=\partial_\mu A_\nu - \partial_\nu A_\mu$\\
\begin{align}
F_{00}=F_{11}=F_{22}=F_{33}=0\\
F_{0i}=\frac{\partial A_i}{\partial x^0}-\frac{\partial A_0}{\partial x^i}
\rightarrow -\frac{1}{c}\frac{\partial}{\partial
t}\vec{A}-\vec{\nabla}\phi=\vec{E}\\
F_{i0}=-F{0i}=F^{0i}
F_{ik}=\frac{\partial A_k}{\partial x^i}-\frac{\partial A_i}{\partial x^k}\\
\rightarrow \frac{1}{2}\epsilon^{ijk}F_{jk}=(\vec{\nabla}x\vec{A})^i=B^i\\
\rightarrow F_{jk}=-\epsilon{jki}B^i
\end{align}
Explizit heißt dies also:
\begin{align}
\begin{pmatrix}
0 & E_1 & E_2 & E_3\\
-E_1 &0&-B_3&B_2 \\
-E_2 &B_3&0&-B_1\\
-E_3 &-B_2&B_1&0
\end{pmatrix}
\end{align}
Räumliche Komponenten der Bewegungsgleichung
\begin{align}
p_\mu&=mcu_\mu , &\mathrm{d}s=c\mathrm{d}t\sqrt{1-\beta^2}\\
u^0&=\frac{1}{\sqrt{1-\beta^2}} , &u^k=\frac{v^k}{c\sqrt{1-\beta^2}}
\end{align}
\begin{align}
-\frac{1}{c\sqrt{1-\beta^2}}\frac{\mathrm{d}p^i}{\mathrm{d}t}&=\frac{e}{c}\left(-E^i
\frac{1}{\sqrt{1-\beta^2}}-\epsilon_{ijk}B^l\frac{v^k}{\sqrt{1-\beta^2}}\right)\\
\frac{\mathrm{d}\vec{p}}{\mathrm{d}t}&=e\left(-\vec{E}+\frac{1}{c}\vec{v}c\vec{B}\right)
\end{align}
Zeitliche Komponenten:
\begin{align}
mc\frac{\mathrm{d}u_0}{\mathrm{d}s}=\frac{e}{c}F_{0i}u^i\\
\frac{\mathrm{d}}{\mathrm{d}t}\left(\frac{mc^2}{\sqrt{1-\beta^2}}\right)=e\vec{E}\vec{v}\\
\frac{\mathrm{d}}{\mathrm{d}t}\left( \xi \right)=e\vec{E}\vec{v}
\end{align}
Mit der Gesatenergie $\xi$
\subparagraph{Lorentzkraft}
Lorentzkraft + Gleichung für $\frac{\mathrm{d}\xi}{\mathrm{d}t}$ sind nicht unabhängig und damit 
aus $u^2=1$ folgen
\begin{align}
\frac{\mathrm{d}}{\mathrm{d}s}u^2=0, & \frac{\mathrm{d}}{\mathrm{d}s}u_\mu u^\mu=2u_\mu\frac{\mathrm{d}u^\mu}{\mathrm{d}s}=0
\end{align}
\subsection{Lorentztransformationen der Felder}
\subsubsection{Vektorpotential}
\begin{align}
A^\mu(x)\rightarrow A^{'\mu}(x')=\Lambda^\mu{}_\nu
A^\nu(x^\alpha=(\Lambda^{-1})^\alpha{}_\beta x{'\beta})\\
F^{\mu\nu}\rightarrow
F^{'\mu\nu}=\Lambda^\mu{}_\alpha \Lambda^\nu{}_\beta F^{\alpha\beta}
\end{align}
ausgeschrieben ergibt dies z.B. für einen Lorentzboost entlang der
$x^1$-Richtung:
\begin{align}
E'_2=E_1 , & E^{'}_2=\frac{E_2-\beta B_3}{\sqrt{1-\beta^2}} , &
E^{'}_3=\frac{E_3+\beta B_2}{\sqrt{1-\beta^2}}\\
B'_2=B_1 , & B^{'}_2=\frac{B_2+\beta E_3}{\sqrt{1-\beta^2}} , &
B'_3=\frac{B_3-\beta E_2}{\sqrt{1-\beta^2}}
\end{align}
\begin{itemize}
  \item ein reines elektrisches Feld in einem Bezugssystem
  ($\vec{E}\neq0$,$\vec{B}=0$)\\ $\rightarrow
  \vec{B}'=\frac{1}{c}\vec{v}x\vec{E}'$ \\ $\vec{B}'\perp \vec{E}'$,
  $\vec{B}'\perp \vec{v}$, $\vec{E}'\perp \vec{v}$
  \item genauso für $\vec{E}\leftrightarrow\vec{B}$
  \item und umgekehrte Aussgage gilt: falls in einem Bezugssystem
  $\vec{E}'\perp\vec{B}'$, dann existiert ein Bezugssystem in dem $\vec{E}=0$
  oder $\vec{B}=0$\\ (z.B. für $v=c\frac{B'}{E'}<c$,das heißt falls
  $B'<E'\rightarrow\vec{B}=0$)
\end{itemize}

Bew.Gl
Ladung e (Kopplungskonstante)\\
elektromagnetsiches Feld $A^\mu$\\
kovariant: $mc\frac{\mathrm{d}u^\mu}{\mathrm{d}s}=\frac{e}{c} F^{\mu\nu} u^\nu$ (Lorentzkraft)\\
Feldstärketensor $F^{\mu\nu}=\partial^\mu A^\nu-\partial^\nu A^\mu \\
\rightarrow (\vec{E},\vec{B})$\\
Eichinvarianz: $A^\mu\rightarrow A^\mu-\partial^\mu\chi$\\
\subsubsection{Lorentz-Transformation}
Invarianten des elektromagnetischen Feldes\\
($x^2$,$A^\mu A_\mu durchstreichen$,$F^{\mu\nu}F_{\mu\nu}$,usw.)\\
\begin{itemize}
  \item $F^{\mu\nu}F_{\mu\nu}$ ist Lorentz-Skalar
  \item $F^{\mu\nu}\tilde{F}_{\mu\nu}$ ist Lorentz-Pseudoskalar
\end{itemize}
\begin{align}
F^{\mu\nu}F_{\mu\nu}=-2\vec{E}^2+2\vec{B}^2=-2(\vec{E^2}-\vec{B}^2)\\
F^{\mu\nu}\tilde{F}_{\mu\nu}=4\vec{E}\vec{B}\\
\text{Invariante:} \vec{E}^2-\vec{B}^2 \text{ und } \vec{E}\vec{B}
\end{align}
folgende Aussagen sind Lorentzinvariant:
\begin{itemize}
  \item Die Beträge sind gleich: $|\vec{E}|=|\vec{B}|$
  \item Orthogonalität: $\vec{E} \perp \vec{B}$
  \item $|\vec{E}|>|\vec{B}| \text{ bzw. } |\vec{E}|<|\vec{B}|$
\end{itemize}
\paragraph{Ergänzung}
Betrachte $\vec{F}=\vec{E}+i\vec{B}$\\
$\rightarrow$ Lorentztransformationen sind dann nur Drehungen dieses Komplexen
Vektors $\vec{F}$ mit komplexen Drehwinkeln\\
$\rightarrow$ Invarianten sind die Längen der Vektoren also
$\vec{F}^2=\vec{E}-\vec{B}+2i\vec{E}\vec{B}$ und es wird einsichtig, dass dies
die einzigen Invariante sind.
\subsection{Bewegungsgleichungen der Felder}
Ziel:Bestimmung der Bewgungsgleichungen der Felder\\
(Bewegungsgleichungen für Ladungen: $\vec{E}$,$\vec{B}$ sind 6 Komponenten,
$A^\mu$ sind 4 Komponenten $\Rightarrow$ Es müssen weitere Gleichungen für
$\vec{E}$,$\vec{B}$ gelten)
\paragraph{Erste Gruppe der Maxwellgleichungen}
\begin{align}
\vec{\nabla} \vec{E}&=
\vec{\nabla}x(-\vec{\nabla}\phi-\frac{1}{c}\frac{\partial}{\partial
t}\vec{A})=-\frac{1}{c}\frac{\partial}{\partial t}\vec{\nabla}\vec{A}\\
\vec{\nabla}\vec{B}&= \vec{\nabla}\vec{\nabla}x\vec{A}=0\\  
\end{align}
\begin{align}
\vec{\nabla} \vec{E}&=-\frac{1}{c}\frac{\partial}{\partial t}\vec{B} \\
\vec{\nabla}\vec{B}&= 0\\
\end{align}
In integraler Form:
\begin{itemize}
  \item aus $\vec{\nabla}x\vec{E}=-\frac{1}{c}\vec{\ddot{B}}$ ergibt unter
  Anwendung des Satzes von Stokes \begin{equation}
  \int_F d\sigma (\vec{\nabla}\times\vec{E})*\vec{n}=\oint_{\partial
  F}d\vec{x}\vec{E}=\frac{1}{c}\int_F d\sigma \vec{\ddot{B}}\vec{n}
  \end{equation}
  \item aus $\vec{\nabla}\vec{B}=0$ und dem Satz von Gauß folgt \begin{equation}
  \int_V \mathrm{d}^3x\vec{\nabla}\vec{B}=\oint_{\partial V}\mathrm{d}\vec{f}\vec{B}=0
  \end{equation}
  Was im endeffekt aussagt, dass es keine magnetischen Monopole gibt.
\end{itemize}

\paragraph{4-Schreibweise}
aus Definition von $F_{\mu\nu}=\partial_\mu A_\nu-\partial_\nu A_\mu$
\begin{align}
\rightarrow \partial_\delta F_{\mu\nu} +\partial_\mu F_{\nu\delta}
+\partial_\nu F_{\delta\mu} =0
\end{align} 
Ist ein total antisymmetrischer Tensor 3. Stufe\\
$\Rightarrow$ nicht trivial nur für $\nu\neq\mu$,$\mu\neq\delta$,$\nu\neq\delta$
$\rightarrow$ 4 nicht triviale Gleichungen\\
oder mit Hilfe von $\tilde{F}_{\mu\nu}$\\
\begin{align}
\rightarrow \partial_\mu\tilde{F}^{\mu\nu}=0
\end{align}

\paragraph{Zweite Gruppe der Maxwellgleichungen}
\begin{align}
\vec{\nabla}\vec{E}=4\pi\rho\\
\vec{\nabla}\times\vec{B}=\frac{1}{c}\vec{\ddot{E}}+\frac{4\pi}{c}\vec{j}
\end{align}
\begin{itemize}
  \item Zusätzliche Informationen: \\
   Ladungsdichte $\rho=\rho(\vec{x},t)$
   Stromdichte $\vec{j}=\vec{j}(\vec{x},t)$
  \item Quellend der Felder sind Ladungen und Ströme
  \item Ladungen und Ströme im Vakuum
  \item \"kontinuierliche\" Verteilungen aus dem Limes: Anzahl der gleadennen
  Teichenchen $\rightarrow \infty$
  \item Gaußsche Einheiten
  \item Integrale Form: \begin{equation}
  \int_V\vec{\nabla}\vec{E}\mathrm{d}^3x=\oint_{\partial V}\vec{E}\mathrm{d}\vec{f}=^!
  4\pi\int_V\mathrm{d}^3x\rho =4\pi Q
  \end{equation} 
  Gesamtladung in $V$: $Q=\int_V \mathrm{d}^3x \rho$
  \begin{equation}
  \int_F \mathrm{d}\vec{f}\vec{\nabla}\times\vec{B}=\oint_{\partial F}\mathrm{d}\vec{x}\vec{B}=^!
  \int_F\mathrm{d}\vec{f}(\frac{1}{c}\vec{\ddot{E}}+\frac{4\pi}{c}\vec{j})
  \end{equation}
  Was als Biot-Savait-Gesetz bekannt ist mit dem Verschiebungsstrom
  $\frac{1}{c}\vec{\ddot{E}}$ \\
  $\rightarrow \vec{B} \text{ aus } \vec{j}$
\end{itemize}
\paragraph{4-Schreibweise der 2.Gruppe der Maxwellschen Gleichungen}
Zunächst: 4-Stromdichte: $j^\mu=(c\rho,\vec{j})$\\
zur Begründung betrachte Vertelung vun Punktladungen $e_a$ für Teilchen 
$a=1,\ldots,N(\rightarrow \infty)$ am Ort $\vec{x}_a$
\begin{equation}
\rho(\vec{x},t)=\sum_{a=1}^N e_a\delta^3(\vec{x}-\vec{x_a}(t))
\end{equation}
Annahme: $\rho(\vec{x},t)$ gegeben, glatt und diffenrenzierbar\\
Volumenelement $\mathrm{d}^3x$ enthält dann die Ladung $\mathrm{d}e=\rho \mathrm{d}^3x$
für bewegte Ladung betrachte dann: Verschiebung um $\mathrm{d}x^mu$
\begin{align}
\rightarrow \mathrm{d}e \mathrm{d}x^\mu=\rho \mathrm{d}^3x \mathrm{d}x^\mu=\rho
\underbrace{\mathrm{d}^3x\mathrm{d}t}{\mathrm{d}\Omega}\frac{\mathrm{d}x^\mu}{\mathrm{d}t}
=\mathrm{d}\Omega \rho \frac{\mathrm{d} x^\mu}{\mathrm{d}t}
\end{align}
Transformationsverhalten
\begin{itemize}
  \item linke Seite:
	\begin{itemize}
  		\item $\mathrm{d}e$ ist Lorentzskalar
  		\item $\mathrm{d}x^\mu$ ist 4-Vektor 
	\end{itemize}
  \item rechte Seite: $\mathrm{d}\Omega$ ist Lorentzinvariante
\end{itemize}
\begin{equation}
\Rightarrow j^\mu=\rho\frac{\mathrm{d}x^\mu}{\mathrm{d}t} \text{ist 4-Vektor}
\end{equation}

Komponentenweise:
\begin{align}
j^0=\rho \frac{\mathrm{d}x^0}{\mathrm{d}t}=c\rho\\
\vec{j}=\rho\frac{\mathrm{d}\vec{x}}{\mathrm{d}t}=\rho\vec{v}=\vec{j}\\
\left( Q=\int_V \mathrm{d}^3x\rho=\frac{1}{c}\int_V \mathrm{d}^3xj^0(\vec{x},t) \right)
\end{align}

schreibe Maxwellgleichung mit Hilfe von $F^{\mu\nu}$ und $j^\mu$
\begin{align}
F^{0i}=-E^i=-F_{0i}\\
\epsilon^{ijk}F_{jk}=B^i
\end{align}
\begin{enumerate}
  \item \begin{align}
  \vec{\nabla}\vec{E}=\frac{\partial}{\partial x^i}E^i=-\partial_i F^{0i}\\
  \vec{\nabla}\vec{E}=4\pi\rho\\
  \Rightarrow \partial_iF^{i0}=\frac{4\pi}{c}j^0
  \Rightarrow \partial_\mu F^{\mu0}=\frac{4\pi}{c}j^0
  \end{align}
  \item \begin{align}
  (\vec{\nabla}\times\vec{B})^i=\epsilon^{ijk}\partial_jB_k\\
  &=\frac{1}{2}\epsilon^{ijk}\partial_j\epsilon_{klm}F^{lm}\\
  &=-\frac{1}{2}\left(\delta^i{}_l \delta^j{}_m -
  \delta^i{}_m \delta^j{}_l \right) \partial_jF^{lm}\\
  =-\frac{1}{2}\left(\partial_mF^{im}- \partial_lF^{li}\right)\\
  =  \partial_l F^{li}\\
  \frac{1}{c}\ddot{E}^i=-\partial_0F^{0i}\\
  \Rightarrow \partial_j F{ji}=-\partial_0F^{0i}+\frac{4\pi}{c}j^i\\
  \partial_\mu F^{\mu i}=\frac{4\pi}{c}j^i\\
  \Rightarrow \partial_\mu F^{\mu \nu}=\frac{4\pi}{c}j^\nu
  \end{align}
\end{enumerate}
\paragraph{kovariante Form der Maxwellgelichungen}
\begin{align}
\partial_\mu\tilde{F}^{\mu\nu}=0\\
\partial_\mu F^{\mu \nu}=\frac{4\pi}{c}j^\nu
\end{align}
Aus der 2. Gleichung folgt, dass die Quellen des elektromagnetischen Feldes
(bewegte) Ladungen sind.
Für ein System von Ladungen gilt
\begin{align}
\sum_{a=1}^N e_a \delta(x-x_a(t))\rightarrow_{N\rightarrow\infty}\rho(x)
\end{align} 
Wirkungsprinziep $\Rightarrow \partial_\mu F^{\mu \nu}=\frac{4\pi}{c}j^\nu$\\

Bisher gingen wir davon aus, dass $S_{WW}$ für ein Teilchen
\begin{align}
S_{WW}=-\frac{1}{c}\sum_a&e_a\int \mathrm{d}x_a^\mu A_\mu(x_a)\\
&\downarrow beliebige Ladungsverteilung\\
S_{WW}=-\frac{1}{c}\int&\mathrm{d}^3x\rho(x)\int \mathrm{d}t\frac{\mathrm{d}x^\mu}{\mathrm{d}t}A_\mu(x)\\
=-\frac{1}{c}\int&\mathrm{d}t\mathrm{d}^3x\underbrace{\rho(x)\frac{\mathrm{d}x^\mu}{\mathrm{d}t}}_{j^\mu(x)}A_\mu(x)\\
=-\frac{1}{c^2}\int&\mathrm{d}\Omega \underbrace{j^\mu A_\mu}_{\text{Lagrangedichte}}\\
\end{align}
\paragraph{Eichinvarianz}
Eichtransormation: $A_\mu\rightarrow A_\mu-\partial_\mu\chi$ mit Eichfunktion
$\chi$\\
Zusatzterm in $S_{WW}$\\
\begin{align}
\Delta S_{WW}&=-\frac{1}{c^2}\int \mathrm{d}\Omega j^\mu \partial_\mu\chi
\text{   partielle Integration}\\
&=\frac{1}{c^2}\underbrace{\int \mathrm{d}\Omega \partial_\mu(j^\mu\chi)}_{\text{nach
Gauß} \oint \mathrm{d}S_\mu(j^\mu\chi)}-\frac{1}{c^2}\int \mathrm{d}\Omega
\chi\underbrace{\partial_\mu j^\mu}_{=0 \text{ wegen der Forderung nach
Eichinvarianz}}\\
\oint \mathrm{d}S_\mu(j^\mu\chi)&=0 \text{ falls man die Oberfläche ins unendliche legt}
\end{align}
\paragraph{Stromerhaltung}
Forderung: $\partial_\mu j^\mu=0$ "Stromerhaltung"
\begin{align}
\frac{\partial}{\partial x^0}j^0+\frac{\partial}{\partial x^i}j^i=0 \\
\partial_\mu j^\mu=0\\
\Rightarrow \ddot{\rho}+\vec{\nabla}\vec{j}=0 \text{  Kontiniuitätsgleichung}
\end{align}
In integraler Form ergibt sich damit:
\begin{align}
\int_V \mathrm{d}^3x\ddot{\rho}(x)=-\int_V \mathrm{d}^3x \vec{\nabla}\vec{j} = \oint_{\partial
V}\mathrm{d}\vec{f}\vec{j}\\
\text{Gesamtladung im Volumen} \ddot{Q}_V=-\oint_{\partial V}\mathrm{d}\vec{f}\vec{j}
\text{Ladungserhaltung}
\end{align}

Wirkung der elektromagnetischen Felder:
\begin{equation}
S=S_M+\underbrace{S_{WW}+\overbrace{S_F}^{\text{für freie Felder}}}_{\text{für
wechselwirkende Felder}}
\end{equation}

Konstruktionsprinziep:
\begin{itemize}
  \item Lorentz-Invariante
  \item Superpositionsprinziep: lineare Differentialgleichungen $\Rightarrow S$
  quadratisch in Koordinaten/Feldern
  \item Eichinvarianz
\end{itemize}
Integrand enthält $A_\mu$,$F_{\mu\nu}$\\
Ansatz: $S_F=f\int \mathrm{d}\Omega F_{\mu\nu}F^{\mu\nu}$\\
\begin{itemize}
  \item Vorzeichen?
	\begin{align}
	F_{\mu\nu}F^{\mu\nu}=2(B^2-&E^2)\\
	&\uparrow \left(\frac{\partial \vec{A}}{\partial t}\right)^2 \text{ kann sehr
	groß werden}\\
	\Rightarrow &f<0 \text{   } (\Rightarrow S \text{ minimal})
	\end{align}
   \item Gaußsches Maßsystem $f=-\frac{1}{16\pi c}$
\end{itemize}
$\rightarrow$ Wirkung:
\begin{align}
S_F&=-\frac{1}{16\pi c}\int \mathrm{d}\Omega F_{\mu\nu}F^{\mu\nu}\\
&=\int \mathrm{d}t\underbrace{\int \mathrm{d}^3x \frac{E^2-B^2}{8\pi}}_{Lagrangedichte}
\end{align}
Weiter mit Ansatz: $S_F=-\frac{1}{16\pi c}\int \mathrm{d}\Omega F_{\mu\nu}F^{\mu\nu}$
\begin{itemize}
  \item Keine Ableitungen von $F$ in der Formel enthalten, da F bereits
  $\partial_mu A_nu$ enthält, worin auch Zeitableitungen enthaltne sind
  \\$\Rightarrow$ unabhängige Variablen in $S_F$:$A^\mu$\\
  Punktmechanik: $q_i$\\ Feldtheorie $A^\mu(\vec{x},t) \leftrightarrow $
  Koordinaten \\ $\partial_0 A^\mu(\vec{x},t) \leftrightarrow $
  Geschwindigkeit \\ kovariant: $\partial^\mu A^\nu$
  \item $F_{\mu\nu}\tilde{F}^{\mu\nu}$ 
  	\begin{enumerate}
	  \item vollständige 4-Divergenz \\ $\Rightarrow$ kein Beitrag zu den
	  Bewegungsgleichungen\\ (Annahme: $A^\mu \rightarrow 0$ für
	  $x\rightarrow\infty$)
	  \item Pseudoskalar\\ $\Rightarrow$ Paritätsverletzung
	\end{enumerate}
\end{itemize}
\paragraph{Herleitung der Bewegungsgleichungen des Feldes (Feldgleichungen)}
Variation $A_\mu \rightarrow A_\mu+\delta A_\mu$
\begin{align}
\rightarrow \delta S&=\delta S_{WW}+\delta S_F\\
&=-\frac{1}{c}\int \mathrm{d}\Omega \left[ \frac{1}{c} j^\mu \delta\!A_\mu +
\frac{1}{16\pi} \underbrace{\delta(F_{\mu\nu}F^{\mu\nu})}_{2(\delta F_{\mu\nu})
F^{\mu\nu}} \right]\\
&=-\frac{1}{c}\int \mathrm{d}\Omega \left[ \frac{1}{c} j^\mu \delta\!A_\mu +
\frac{1}{8\pi} F^{\mu\nu}\left( \partial_\mu\delta\!A_\nu-
\partial_\nu\delta\!A_\mu \right)\right]\\
F^{\mu\nu} \partial_\mu\delta\!A_\nu&=F^{\nu\mu} \partial_\nu\delta\!A_\mu
\text{ 1: Umbenenennen der Indizes}\\
F^{\nu\mu} \partial_\nu\delta\!A_\mu&=-F^{\mu\nu} \partial_\nu\delta\!A_\mu
\text{ 2: Vertauschen der Indizes von }F\\
\Rightarrow \delta S &= -\frac{1}{c}\int \mathrm{d}\Omega \left[ \frac{1}{c} j^\mu
\delta\!A_\mu - \frac{1}{4\pi} F^{\mu\nu} \partial_\nu\delta\!A_\mu \right]\\
&= -\frac{1}{c}\int \mathrm{d}\Omega \left[ \frac{1}{c} j^\mu
\delta\!A_\mu - \frac{1}{4\pi} \partial_\nu F^{\mu\nu} \delta\!A_\mu \right] +0
\text{ Die Randterme im Unendlichen sind wieder 0} \\
&= -\frac{1}{c}\int \mathrm{d}\Omega \underline{\left[ \frac{1}{c} j^\mu - \frac{1}{4\pi}
\partial_\nu F^{\mu\nu}\right]}_{!=0 \text{ aus } \delta\!S=0}
\underbrace{\underline{\delta\!A_\mu}}_{\text{beliebig}}\\
\Rightarrow \partial_\mu F^{\mu\nu}&=\frac{4\pi}{c}j^\nu \text{ Wieder die
Maxwellgleichungen}
\end{align}
Die Kontinuitätsgleichung ergibt sich aus $\partial_\nu\partial_\mu
F^{\mu\nu}=0 \rightarrow \partial_\mu j^\mu=0$

\subsubsection{Energie und Impuls des Feldes}
Feld trägt $E$,$\vec{p}$ ($\rightarrow p^\mu$)\\
Ausgangspunkt: Maxwellgleichungen
\begin{align}
\vec{\nabla}\times\vec{B}=\frac{1}{c}\ddot{\vec{E}}+\frac{4\pi}{c}\vec{j}
\text{    |}*\vec{E}\\
\vec{\nabla}\times\vec{E}=-\frac{1}{c}\ddot{\vec{B}} \text{    |}*\vec{B}
\end{align}
Differenz dieser beiden Gleichungen liefert:
\begin{align}
\vec{E}*\left(\vec{\nabla}\times\vec{B}\right)-\vec{B}*\left(\vec{\nabla}
\times\vec{E}\right)=\frac{1}{c}(\vec{E}\ddot{\vec{E}}+\vec{B}\ddot{\vec{B}})
+\frac{4\pi}{c}\vec{j}\vec{E}\\
\vec{\nabla}*\left( \vec{B}\times\vec{E}\right)=\frac{1}{2c}
\frac{\partial}{\partial t}(\vec{E}^2+\vec{B^2})+\frac{4\pi}{c}\vec{E\vec{j}}
\end{align} 
Poynting-Vektor $\vec{S}=\frac{c}{4\pi}\vec{E}\times\vec{B}$
\begin{align}
\frac{\partial}{\partial t}\left( \frac{\vec{E}^2+\vec{B}^2}{8\pi} \right)
=-\vec{\nabla}\vec{S}-\vec{E}\vec{j}
\end{align}
In integraler Form ergibt sich dann:
\begin{align}
\int_V \mathrm{d}^3x \vec{\nabla}\vec{S}&=\oint_{\partial V}\mathrm{d}\!\vec{f}\vec{S}\\
\underbrace{\int_V \mathrm{d}^3x \vec{E}\vec{j}}_{\text{Zeitableitung der Gesamtengerie
der Ladung}}&=\int_V
\underbrace{\mathrm{d}^3x\rho}_{e}\vec{v}\vec{E}=\frac{\mathrm{d}}{\mathrm{d}t}\epsilon\\
\underbrace{\frac{\partial}{\partial t}\left(\int_V \mathrm{d}^3x
\frac{\vec{E}^2+\vec{B}^2}{8\pi} +\epsilon
\right)}_{\frac{\partial}{\partial t}\left( \ldots \right)=0 \Rightarrow
\text{Erhaltungssatz}} =\underbrace{-\oint_{\partial
V}\mathrm{d}\vec{f}\vec{S}}_{\text{für } V\rightarrow\infty \Rightarrow =0}
\end{align}
$\Rightarrow$ Def.: Energiedichte des elektromagnetischen Feldes
\begin{equation}
W=\frac{1}{8\pi}(\vec{E}^2+\vec{B}^2)
\end{equation}
Für endliche Volumen gilt: Energiefluß: $\oint \mathrm{d}\vec{f}\vec{S}$
\begin{equation}
\vec{S}=\frac{c}{4\pi}\vec{E}\times\vec{B}
\end{equation}


\subsubsection{Allgemeine Feltheorie (Wdh.)}
Wirkung des elektromagnetischen Feldes 
\begin{align}
S=\frac{1}{c}\int \mathrm{d}^4x\left(\underbrace{-\frac{1}{16\pi}F_{\mu\nu}F^{\mu\nu}-
\frac{1}{c}j^\mu A_\mu}_{\text{Lagrangedichte}} \right) +\text{Materie}\\
\delta S=0 \text{ unter Variationen } A_\mu \rightarrow A_\mu+\delta A_\mu\\
\Rightarrow \partial_\mu F^{\mu\nu}=\frac{4\pi}{c}j^\nu
\end{align}
Energie-Bilanz:
\begin{align}
\frac{\partial}{\partial t}\left(\int_V \mathrm{d}^3x
\frac{E^2+B^2}{8\pi}+\epsilon_V\right)=-\oint_{\partial V} \mathrm{d}\vec{f}\vec{S}
\end{align}
Eneriedichte des Feldes: $W=\frac{E^2+B^2}{8\pi}$\\
Energiestromdichte (Poynting-Vektor):
$\vec{S}=\frac{c}{4\pi}\vec{E}\times\vec{B}$
\subsubsection{Vergleich von klassischer Punktmechanik und Feldtheorie}

\begin{tabular}{c|c}
klassische Punktmechanik & Feldtheorie\\  
\hline\hline
verallgemienerte Koordinaten $q_i(t)$ &
"Koordiaten":Felder $q(x^\mu)$ (z.B $A^\mu(x)$)
\\
($i\in \{1,\ldots,N\}$)&$x^\mu \in \mathbb{R}^4$, kontinuierlich\\
verallgemienerte Geschwindigkeiten $\dot{q}_i(t)$&
"Geschwindigkeiten" $\dot{q}(x^\mu)$ bzw. $\partial_\mu q$ (z.B.
$\partial^\mu A^\nu(x)$)\\
Lagrangefunktion $L(q_i,\dot{q}_i)$ & Lagrangedichte $\Lambda(q,\partial_\mu
q)$\\
Wirkung $S[q_i(t)]$ als Funktional & Wirkung $S=\int \mathrm{d}t \mathrm{d}^3x
\Lambda(q,\partial_\mu q)=S[q(x^\mu)]$\\
Bewegungsgleichung aus $\delta S=0$&Bewegungsgleichung aus $\delta S=0$\\
\end{tabular}

Bewegungsgleichungen aus dem Variationsprinziep
\begin{align}
\delta S=\frac{1}{c}\int \mathrm{d}\Omega \left( \frac{\partial
\Lambda}{\partial\!q}\delta\!q +\underbrace{\frac{\partial \Lambda}{\partial\!
(\partial_\mu\! q)} \partial_\mu\! \delta\!q}_{\text{Lösung: part. Int.}}
\right)
\end{align}
Forderung $\delta S=0$ für beliebige $\delta q$
\begin{align}
\Rightarrow  \frac{\partial
\Lambda}{\partial\!q} -\partial_\mu \frac{\partial \Lambda}{\partial\!
(\partial_\mu\! q)}=0
\end{align}
\subsubsection{Erhaltung von Größen}
\paragraph{Energieerhaltung}
\begin{enumerate}
  \item Translation der Zeit \\ Invarianz $\Rightarrow$ Erhaltungsgröße
  \item Abgeschlossenes System \\ $\rightarrow$ $L$,$\Lambda$ besitzen keine
  explizite Zeitabhängigkeit
\end{enumerate}
\paragraph{Impulserhaltung} 
Die Impulserhaltugn läst sich anlog herleiten:\\
$\Leftrightarrow$ Invarianz unter räumlihen Transformationen
\paragraph{allgemene Feldtheorie}
$\Lambda$ hängt nicht explizit von $x^\mu$ ab
\begin{align}
\partial_\mu \Lambda &= \frac{\partial \Lambda}{\partial q}
\partial_\mu\!q+\frac{\partial \Lambda}{\partial(\partial_\nu
q)}\partial_\mu(\partial_\nu q)\\
&=\partial_\nu \frac{\partial \Lambda}{\partial (\partial_\nu q)}
\partial_\mu\!q+\frac{\partial \Lambda}{\partial(\partial_\nu
q)}\partial_\nu(\partial_\mu q)\\
=\partial\left( \frac{\partial \Lambda}{\partial(\partial_\nu q)}\partial_\mu q
\right)\\
\Rightarrow \underbrace{\partial_\nu \left( \frac{\partial
\Lambda}{\partial(\partial_\nu q)}\partial_\mu q - g_\mu{}^\nu \Lambda
\right)}_{\partial_\nu T_\mu{}^\nu=0}=0\\
\partial_\mu j^\mu=0 \rightarrow \dot{Q}=0\\
0=\int_V \mathrm{d}\Omega \partial_\nu T_\mu{}^{\nu}=\oint_{\partial V} \mathrm{d}S_\nu
T_\mu{}^\nu
% XXXX Einfügen Grafik X1
\end{align}
bei fester Zeit $t_1$ und $t_2$ gilt 
\begin{align}
\mathrm{d}S^\nu \rightarrow \mathrm{d}S^0=\mathrm{d}x^1\mathrm{d}x^2\mathrm{d}x^3
\end{align}
$t_1$, $t_2$ beliebig
\begin{align}
0=\int \mathrm{d}^3x T_\mu{}^0|_{x^0=ct_2}-\int \mathrm{d}^3x T_\mu{}^0|_{x^0=ct_1}
\Rightarrow \text{Erhaltungsgröße } \int \mathrm{d}^3x T_\mu{}^0|_{x^0=ct}
\end{align}
Diese Erhaltungsgröße folgen aus der Translationsinvarianz
\begin{align}
\int \mathrm{d}^3x T_\mu{}^0=\text{const.}* P_\mu
\end{align}
zur Normierung
\begin{align}
T_{00}=\frac{\partial \Lambda}{\partial(\partial_0 q)}\partial_0 q
-\Lambda=\frac{\partial \Lambda}{\partial \dot{q}}\dot{q}-\Lambda\\
H=\underbrace{\frac{\partial L}{\partial v}}_{p} v-L\\
\text{Wirkung}=\left[ \text{Energie}*\text{Zeit}\right]\\
=\frac{1}{c}\int \mathrm{d}x^0\underbrace{\int \mathrm{d}^3x\Lambda}_{E}\\
\rightarrow P^\mu=\frac{1}{c}\int \mathrm{d}S_\nu T^{\mu\nu}
\end{align}
Beschreibt dann den 4-Impuls des Feldes
\paragraph{Bemerkungen}
\begin{itemize}
  \item $T^{\mu\nu}$ ist nicht eindeutig \\ ein beliebiger Tensor dritter Stufe
  $\Psi^{\mu\nu\rho}=-\Psi^{\mu\rho\nu}$\\ $\rightarrow
  \partial_\mu\partial_\nu\Psi^{\mu\nu\rho}=0$\\$\rightarrow
  \partial_\nu(T^{\mu\nu}+\partial_\rho \Psi^{\mu\nu\rho} )=0$
  \item $P^\mu=\frac{1}{c}\int \mathrm{d}S_\nu \left( T^{\mu\nu}+\partial_\rho
  \Psi^{\mu\nu\rho} \right)\rightarrow \oint \mathrm{d}f_{\nu\rho} \Psi^{\mu\nu\rho}=0 $
  \item Zusatzforderung: $T^{\mu\nu}$ soll symmetrisch sein aus: Drehimpuls und
  Impuls erfüllen die gewohnten Beziehung: $M^{\mu\nu}=\int\left( x^\mu
  \mathrm{d}P^\nu-x^\nu \mathrm{d}P^\mu \right)$\\$=\frac{1}{c}\int \mathrm{d}S_\rho\left( x^\mu
  T^{\nu\rho}-x^\nu T^{\mu\rho} \right)$\\ Die Drehimpulserhaltung folgt dann
  aus der Invarianz unter Drehungen\\ $\partial_\rho\left( x^\mu
  T^{\nu\rho}-x^\nu T^{\mu\rho} \right)=0$
\end{itemize}
\begin{align}
g_\rho{}^\mu T^{\nu\rho}+x^\mu \underbrace{\partial_\rho
T^{\nu\rho}}_{=0}-g_\rho{}^\nu T^{\mu\rho}+\underbrace{x^\nu \partial_\rho
T^{\mu\rho}}_{=0}=0\\
\Rightarrow T^{\nu\mu}-T^{\mu\nu}=0
\end{align}
d.h. man fordert dass $T^{\nu\mu}$ symmtetrisch ist. Explizit: durch
geeignete Wahl von $\psi^{\mu\nu\rho}$ \\
$T^{\mu\nu}$ heißt Energie-Impuls-Tensor

\paragraph{Feldtheorie Wdh.}
\begin{align}
q\rightarrow \text{ elektromagn } A_\mu(x)
\dot{q} \rightarrow \partial_\mu A_\nu(x)
\text{Wirkung} S=\frac{1}{c}\int \mathrm{d}^4x \Lambda(A_\mu(x),\partial_\mu
A_\nu(x))
\end{align}
\paragraph{Lagrangedichte Wdh.}
Konstruktion: Lorentz-Invariante
Eichinvarianz
$\rightarrow$ Bewgeungsgleichungen (Euler-Lagrange-Gleichungen)
\begin{align}
\partial_\mu \frac{\partial \Lambda}{\partial(\partial_\mu
A_\rho)}-\frac{\partial \Lambda}{\partial_\rho}=0 \text{    } (q\rightarrow
A_\rho)
\end{align}
Translationsinvarianz $\Rightarrow$ Erhaltungssatz
\begin{align}
\partial_\rho T_\mu{}^\nu=0
\end{align}
4-Impulserhaltung
\begin{align}
P^\mu=\frac{1}{c}\int_{\partial V} \mathrm{d}S_\nu T^{\mu\nu} 
\end{align}
Wobei $\partial V$ eine Raumartige Hyperfläche darstellt mit $\mathrm{d}S_\nu$ dem
Normalenvektor auf dieser Fläche, was dazu führt, das dieser Vektor nur eine
nicht verschwindende Komponenten in zeitlicher Richtung besitzt.
\begin{align}
\Rightarrow P^\mu=\frac{1}{c}\int \mathrm{d}^3x T^{\mu0}
\end{align}

Zunächst 
\begin{align}
T_\mu{}^\nu=\frac{\partial \Lambda}{\partial(\partial_\nu q)}\partial_\mu
q-g_\mu{}^\nu\Lambda\\
=\frac{\partial \Lambda}{\partial(\partial_\nu A_\rho)}\partial_\mu
A_\rho-g_\mu{}^\nu\Lambda
\end{align}
lässt sich symmetrisieren.
Interpretation der Komponenten von $T^{\mu\nu}$
\begin{enumerate}
  \item $T^{00} \rightarrow$ Energiedichte
  \item $T^{i0} \rightarrow$ Impulsdichte
  \item aus $\partial_\rho T^{\mu\rho} \rightarrow
  \frac{1}{c}\frac{\partial}{\partial t} T^{\mu 0}+\frac{\partial}{\partial
  x^i} T^{\mu i}=0$ \\ Integralform über den Satz von Gauß für $\mu=0$:\\
  $\frac{1}{c}\int_V \mathrm{d}^3x \frac{\partial}{\partial t} T^{0 0}=-\int_{\partial
  V} \mathrm{d}f_i T^{0i}$ \\ $\rightarrow$ zeitliche Änderung der Energie$=$Energie die
  durch die Oberfläche wegfließt\\ $cT^{0i}=S^i$Energiestromdichte mit dem
  (Poynting-Vektor $S^i$)
  \item für $\mu=i$ genauso \\ $\frac{1}{c}\frac{\partial}{\partial t}
  T^{j0}=-\frac{\partial}{\partial x^i} T^{ji}$\\ $\frac{\partial}{\partial
  t}\frac{1}{c}\int_V\mathrm{d}^3x T^{j0}=-\oint_{\partial V}\mathrm{d}f_i T^{ji}$\\$\rightarrow$
  $T^{ji}$ ist Impulsstromdichte\\zeitliche Änderung des Impulses$=$Impulsstrom
  durch Oberfläche \\ $T^{ji}$ heißt auch Maxwellscher Spannungstensor
  \begin{align}
  T^{\mu\nu}=\begin{pmatrix}W&&\frac{1}{c}\vec{S}&\\&&&\\
  \frac{1}{c}\vec{S}&&\sigma^{ij}&\\&&&\end{pmatrix}_{\text{sym.}}
  \end{align}\\ für das elektromagnetsiche Feld \begin{align}
  \Lambda =-\frac{1}{16\pi}F_{\alpha\beta}F^{\alpha\beta} \text{,   }
  F_{\alpha\beta}=\partial_\alpha A_\beta-\partial_\beta A_\alpha \\
  \Rightarrow T^{\mu\nu}=-\frac{1}{4\pi}\partial^\mu A_\rho
  F^{\nu\rho}+\frac{1}{16\pi} g^{\mu\nu} F_{\alpha\beta}F^{\alpha\beta}
  \end{align}
  Rechenregel \begin{align}
  \frac{\partial}{\partial(\partial_\nu A_\rho)}\partial_\alpha A_\beta
  =\delta^\nu_\alpha\delta^\mu_\beta\\
  \frac{\partial}{\partial(\partial_\nu A_\rho)}\partial^\alpha A^\beta
  =g^{\nu\alpha}g^{\mu\beta}\\
  \text{symmetrisiere} \Phi^{\mu\nu\rho}=A^\mu F^{\nu\rho} \text{
  (Vorfaktor?)}\\
  \rightarrow T^{\mu\nu}=\frac{1}{4\pi}\left(
  F^{\mu\rho}F_{\rho}{}^{\nu} \frac{1}{4} g^{\mu\nu}
  F_{\alpha\beta}F^{\alpha\beta}\right)
  \end{align}
  nachrechnen
  \begin{align}
  T^{00}=\frac{1}{8\pi}(E^2+B^2)\\
  cT^{0i}=S^i
  \end{align}
  maxwellscher Spannungstensor
  \begin{align}
  \sigma_{11}=\frac{1}{8\pi}\left( E_2^2+E_3^2-E_1^2+B^2_2+B^2_3+B^2_1 \right)\\
  \sigma_{12}=-\frac{1}{4\pi}\left( E_1+E_2+B_1B_2 \right)
  \end{align}
  \item zusätzlich: Beitrag der Teilchen: Index $a$\\
  \begin{itemize}
  \item Massendichte $\mu(\vec{x})=\sum_am_a\delta^{(3)}(\vec{x}-\vec{x}_a)$
  	(analog zur Lagrangedichte)\\
  	mit den Bahnkurven der Teilchen $\vec{x}_a(t)$\\
  \item Geschwindigkeitsstromdichte 
  	$u^\mu(x)=\sum_a u^\mu_a \delta^{(3)}(\vec{x}-\vec{x}_a)$
  \item Massenstromdichte: $\rightarrow \frac{1}{c}\mu\frac{\mathrm{d}x^\rho}{\mathrm{d}t}$(analog zur
  	Ladungsstromdichte)\\
  \item Impulsdichte $T^{\rho0}=c^2\mu u^\rho$\\ \begin{align}
  	\rightarrow T^{\rho\sigma}=\mu c u^\rho u^\sigma \frac{\mathrm{d}S}{\mathrm{d}t}
  	\end{align} 
  \end{itemize}
\end{enumerate}
für elektromagenteische Felder: $T_\rho{}^\rho=0$
für Telchen: $T_\rho{}^\rho=\mu c^2\sqrt{1-\beta^2}$
allgemein gilt $T_\rho{}^\rho\geq0$
Massenerhaltung: $\partial_\rho\left( \frac{1}{c}\mu \frac{\mathrm{d}x^\rho}{\mathrm{d}t}
\right)=0$

\section{Elektrodynamik}
\subsection{Maxwell-Gleichungen}
\begin{align}
\vec{\nabla}\vec{E}=4\pi\rho\\
\vec{\nabla}\times\vec{B}=\frac{1}{c}\dot{\vec{E}}+\frac{4\pi}{c}\vec{j}\\
\Rightarrow \partial_\mu F^{\mu\nu}=\frac{4\pi}{c}j^\nu\\
\vec{\nabla}\vec{B}=0\\
\vec{\nabla}\times\vec{E}=-\frac{1}{c}\dot{\vec{B}}\\
\Rightarrow \partial_\mu \tilde{F}^{\mu\nu}=0
\end{align}
$\Rightarrow$ Lorentzkraft
\begin{align}
\dot{\vec{p}}=e\vec{E}+\frac{e}{c}\vec{v}\times\vec{B}\\
\Rightarrow mc\frac{\mathrm{d}u^\mu}{\mathrm{d}S}=\frac{e}{c}F^{\mu\nu}u_\nu
\end{align}
\subsubsection{Klassifizierung von Problemstellungen}
\begin{itemize}
  \item Wie berechne ich das Feld einer gegebenen (möglicherweise bewegten)
  Ladungsverteilung
  \item Wie bewegen sich Ladungen in vorgegebenen Feldern
\end{itemize}
Betrachte die Spezialfälle
\begin{enumerate}
  \item Statische Felder $\dot{\vec{E}}=0$ und/oder $\dot{\vec{B}}=0$\\
  Elektrostatik \\ Magnetostatik
\begin{enumerate}
  \item homogene Felder $\checkmark$
  \item inhomogene Felder \\ wichtiger Spezialfall: Coulombproblem (rel.)
  \item Multipolfelder 
\end{enumerate}
  \item Elektromagnetische Wellen \\ $\dot{\vec{E}}\neq0$ und
  $\dot{\vec{B}}\neq0$
  \begin{enumerate}
  \item im Ladungsfreien Raum, d.h. $\rho=0$, $\vec{j}=0$
  \item zeitabhängige Felder für bewegte Ladungen \\ $\rightarrow$ Abstrahlung
  von Elektromagnetischen Wellen
\end{enumerate}
\end{enumerate}

\subsection{Statische Felder}
\subsubsection{Zum Coulombproblem}
suche statische Lösung zu unbewegten Ladungen
\begin{equation}
\rho\neq0,\dot{\rho}=0,\vec{j}=0
\end{equation}
Maxwellgleichungen reduzieren sich auf 
\begin{align}
\vec{\nabla}\vec{E}=4\pi\rho \text{ und }
\underbrace{\vec{\nabla}\times\vec{E}=0}_{\text{Ansatz:}
\vec{E}=-\vec{\nabla}\phi}\\
\laplace\phi=-4\pi\rho
\end{align}
für eine im Ursprung befindliche Punktladung\\
Invarianz unter Drehungen $\rightarrow$ Ansatz: $\vec{E}=\vec{x}g(|\vec{x}|)$
\begin{equation}
\int \mathrm{d}^3x \vec{\nabla}\vec{E}=\oint \mathrm{d}\vec{f} \vec{E} \overset{!}{=}  \int
\mathrm{d}^3x4\pi\rho=4\pi e
\end{equation} 
wähle $V=$Kugel 
\begin{equation}
\Rightarrow \int \mathrm{d}^3x
\vec{\nabla}\vec{E}=\overset{!}{=}\int_\text{Kugeloberfläche} \mathrm{d}\cos(\theta)\mathrm{d}\rho
\underbrace{|\vec{x}|^3 g(|\vec{x}|)}=4\pi|\vec{x}|^3 g(|\vec{x}|)\\
\Rightarrow \vec{E}=e\frac{\vec{x}}{|\vec{x}|^3}
\end{equation}
\paragraph{Wiederholung}
Elektrodynamik\\
Coulombproblem\\
Maxwell-Gleichungen:\\
$\oplus$ Annahme: $\dot{\rho}=0\rightarrow \dot{\vec{E}}=0$ ist möglich\\
$\oplus$ Annahme: $\dot{j}=0\rightarrow \vec{B}=0$ ist möglich\\
$\rightarrow \vec{\nabla}\times\vec{E}=0 \rightarrow$ Ansatz
$\vec{E}=-\vec{\nabla}\phi$\\
$\rightarrow$ Poissongleichung $\laplace\phi=e\delta(\vec{x})$\\
$\rightarrow \vec{E}=e\frac{\vec{x}}{|\vec{x}|^3}$, $\phi=\frac{e}{|\vec{x}|}$\\
Für ein System von Punktladungen gilt dann:
\begin{align}
\rho=\sum_a e_a \delta(\vec{x}-\vec{x}_a)\\
\rightarrow \phi=\sum_a e_a \frac{1}{|\vec{x}-\vec{x}_a|}
\end{align}
Für kontinuierliche Ladungsverteilungen gilt demnach:
\begin{align}
\phi(\vec{x})=\int \mathrm{d}^3x'\frac{\rho(\vec{x}')}{|\vec{x}-\vec{x}'|}\\
\end{align}
Dabei wird die folgende Relation verwendet:
\begin{align}
\underbrace{\laplace
\frac{1}{|\vec{x}|}}_{\text{Distribution}}=-4\pi\delta(\vec{x})
\end{align}
Diese besagt im Endeffekt, dass man $\laplace \frac{1}{|\vec{x}|}$ auch als
Distribution auffassen kann, die dann wie ein Vielfaches der Deltdistribution
wirkt.\\
Feldtensor:\\
Elektrostatische Energie für ein System von Ladungen\\
Energiedichte $T^{00}=W=\frac{1}{8\pi}\left(\vec{E}^2+\vec{B}^2\right)$\\
Energie 
\begin{align}
U&=\frac{1}{8\pi}\int \mathrm{d}^3x \vec{E}^2\\
&=-\frac{1}{8\pi}\int \mathrm{d}^3x \vec{E}\vec{\nabla}\phi\\
&=-\frac{1}{8\pi}\int^{+\infty}_{-\infty} \mathrm{d}^3x \left(
\underbrace{\vec{\nabla}\left(\vec{E}\phi\right)}_{=0}-\phi\vec{\nabla}\vec{E}\right)\\
&=\frac{1}{2}\int \mathrm{d}^3x\phi\rho \text{ für } \rho=e\delta(\vec{x}-\vec{x}_0)
\end{align}
Für eine Punktladung gilt demnach
\begin{align}
U=\frac{1}{2}e\phi(0) \text{ ist Divergent}
\end{align}
$\Rightarrow$ die Selbstenergie ist divergent.\\
$\rightarrow$ Elektrodynamik selbst zeigt, dass sie nur einen begrentzten
Anwendungsbereich besitzt.\\
$\rightarrow$ Quantenelektrodynamik\\
$\rightarrow$ Modifikation der Physik bei kleinen Abständen\\
für Elektronen\\
\begin{align}
\frac{e^2}{2r_0}&=mc^2 \\
\Rightarrow r_0&=\frac{e^2}{2mc^2}=3*10^{-15}m \text{
mit }r_0\text{ als klassischem Elektronenradius}
\end{align}
$\rightarrow$ für ein System von Ladungen: Wechselwirkungsenergie
\begin{align}
U=\frac{1}{3}\sum_a e_a \left(\sum_{b} \frac{e_b}{|\vec{x}_a-\vec{x}_b|} \right)
\end{align}
Selbstenergie streichen:
\begin{align}
U'=\frac{1}{3}\sum_a e_a \left(\sum_{b\neq a} \frac{e_b}{|\vec{x}_a-\vec{x}_b|}
\right)
\end{align}
\paragraph{Relativistishes Teilchen im Coulombpotential}
Coulombpotential $\phi=\frac{e'}{r}$ (Ladung $e'$ im Ursprung)\\
Testladung $e \ll e'$ die das oben angegebene Potnetial nicht stört.
Koordinaten der Testladung $\vec{x}=(r \sin(\theta), r \cos(\theta))$\\
Durch Invarianz des Potentials unter Drehungen gilt die Drehimpulserhaltung
und damit bewegt sich das Teilchen nur in einer Ebene.\\
(Lorentz-Kraft $\rightarrow$ Bew.-Gl.)\\
Impuls: 
\begin{align}
\vec{p}=\gamma m \dot{\vec{x}} \text{ , }
\gamma&=\frac{1}{\sqrt{1-\frac{\dot{\vec{x}}^2}{c^2}}}\\
&=\gamma m \left(\dot{r}\text{ }\sin(\theta)+r \dot{\theta} \text{
}\cos\theta,\dot{r} \text{
}c\phi=\frac{e}{(R-\vec{R}\vec{\beta})}|_{t'=t-\frac{R}{c}}os\theta+r\dot{\theta} \text{ }\sin\theta,0\right)\\
p^2&=\gamma^2m^2(\dot{r}^2+r^2 \theta^2)
\end{align}
Drehimpuls: $\vec{M}=\vec{x}\times\vec{p}$\\
$M=|\vec{M}|=\gamma mr^2\dot{\theta}$\\
Energie (Gesamtenergie)
\begin{align}
\epsilon=c\sqrt{\vec{p}^2+m^2c^2}+\frac{\alpha}{r}
\end{align}
2 Erhaltungsgrößen

\begin{align}
\epsilon=c\sqrt{\vec{p}^2+m^2c^2}+\frac{\alpha}{r}\\
M=\gamma mr2\dot{\theta} \text{ mit }p_r=\gamma m \dot{\vec{r}}\\
\gamma&=\frac{1}{\sqrt{1-\frac{\dot{\vec{x}}^2}{c^2}}}
\end{align}
Wobei $\dot{\vec{x}}^2$ die Größen $\dot{r}$ und $\dot{\theta}$ enthält.
\begin{align}
\epsilon=\epsilon(\dot{r},\dot{\theta})\\
M=M(\dot{r},\dot{\theta})\\
\Rightarrow \text{ berechne }\dot{r},\dot{\theta} \text{ als Funktionen von }
\epsilon,M\\
\dot{\theta}^2=\frac{M^2c^4}{r^2(\epsilon r -\alpha)^2}\\
\dot{r}^2=c^2\frac{(\epsilon r -\alpha)^2-c^2(M^2+m^2c^2r^2)}{(\epsilon r
-\alpha)^2}\\
\Rightarrow \frac{\mathrm{d}\theta}{\mathrm{d}r}=f(r,\epsilon,M)\\
\dot{\theta}=\frac{\mathrm{d}\theta}{\mathrm{d}t}\text{,  }\dot{r}=\frac{\mathrm{d}r}{\mathrm{d}t}
\end{align}
Bahnkurve $\theta=\theta(r)$ oder $r=r(\theta)$ oder
$f(r,\theta)=const$\\
\subparagraph{Fallunterscheidungen}
\begin{enumerate}
  \item $\alpha>0$, $\alpha<0$
  \item $|\alpha|=cM$, $|\alpha|<cM$, $|\alpha|>cM$
\end{enumerate}
Dies führt zu folgenden Folgerungen für den ersten Punkt:\\
$\alpha>0$ abstoßende Ladungen\\
$\rightarrow \epsilon>0 \forall r$\\
für $r\rightarrow 0: \epsilon\rightarrow\infty$\\
$\rightarrow$ "Hyperbel" (im Relativistischen nur nährungsweise für kleine
$\gamma$)
Bemerkungen:
keine Lösung ist periodisch (für $\alpha>0$ und $\alpha<0$):
\begin{align}
\text{z.B. } &|\alpha|<Mc\\
(c^2M^2-\alpha^2)\frac{1}{r}&=c\sqrt{(M\epsilon)^2-m^2c^2(M^2c^2-\alpha^2)}
\cos\left(\theta\sqrt{1-\frac{\alpha^2}{2M^2}}\right)-\epsilon\alpha
\end{align}
offene Rosetten

für $\alpha<0$, $|\alpha|>Mc$
für $\theta\rightarrow\infty$ gilt $r\rightarrow0$
Also stürzt das Teilchen ins Zentrum. Allerdings tut es das in endlicher Zeit.
\subsubsection{Felder von vorgegebenen Ladungsverteilungen}
Maxwell-Gleichungen für Statik:\\
$\rightarrow \rho$,$\vec{j}$ legen Divergen und Rotation mit der Annahme, dass
das Feld im unendlichen verschwindet, also für $\vec{x}\rightarrow\infty$ gilt
$\phi(\vec{x})\rightarrow0$
\subparagraph{Satz:}
Ein Vektorfeld ist eindeuig festgelegt, wenn in allen Raumpunkten die Quellen
$\vec{\nabla}\vec{V}$ und die Wirbel $\vec{\nabla}\times\vec{V}$ bekannt sind
und wenn es im unendlichen hinreichend schnell verschwindet.\\ 
Ergänzung: Satz von Helmholtz
Ein Vektorfeld $\vec{V}(\vec{x})$ das einschließlich seiner Ableitungen mit
hinreichender Ordnung gegen 0 geht, wenn x gegen unendlich geht lässt sich
dieses als 


%XXXX Einfügen der Seiten 3-5

\subsection{Multipolentwicklung}
\subsubsection{Dipolterm}
%XXXX Bild X2 einfügen

Für $R>>|\vec{x}_a|$ lässt sich das Potential für Punktladungen $e_a$ sinvoll
entwickeln:
\begin{align}
\phi&=\sum\frac{e_a}{|\vec{x}-\vec{x}_a|}
\text{Taylorentwicklung } \Rightarrow \phi=\frac{Q}{R}-
\vec{D}\vec{\nabla}\frac{1}{R}+\ldots\\
Q&=\text{Gesamtladung}=\sum e_a\\
D&=\text{Dipolmoment}=\sum e_a \vec{x}_a \rightarrow \int
\mathrm{d}^3x\rho(\vec{x})\vec{x} \text{  (Dipolpotential)  }
\end{align}
Sortiert nach positiven und negativen Ladungen
\begin{align}
\rightarrow \vec{D}&=\sum_{pos} e_a^+\vec{x}_a-\sum_{neg} e_a^-\vec{x}_a\\
&\text{Mit dem Ladungsschwerpunkt}\\
\vec{X}^\pm&=\frac{\sum e^\pm_a\vec{x}_a}{\sum e^\pm_a}\\
Q^\pm&=\sum e^\pm_a\\
\vec{D}&=Q^+\vec{X}^+-Q^-\vec{X}^-\\
&\text{z.B.: }Q=Q^+-Q^-=0 \leftarrow \text{ Wichtiger Spezialfall}\\
\vec{D}&=Q^+(\vec{X}^+-\vec{X}^-)
\end{align} 
für $Q=0$
\begin{align}
\phi(\vec{x})=-\vec{D}\vec{\nabla}\frac{1}{R}\\
=\vec{D}\frac{\vec{x}}{R^3}=\vec{D}\frac{\vec{e}_x}{R^2}\\
\rightarrow \phi \propto \frac{1}{R^2}
\end{align}
Für die Feldstärke gilt dann
\begin{align}
\vec{E}=-\vec{\nabla}\phi\\
=-\frac{\vec{D}}{R^3}+3\frac{\vec{D}\vec{x}}{R^5}\vec{x}\propto \frac{1}{R^3}
\end{align}
besitz eine axialsymmetrische Komponente gegeben durch $\frac{\vec{D}}{R^3}$
\subsubsection{Quadrupolmoment}
\begin{align}
\phi^{(2)}(\vec{x}))=\underbrace{\frac{1}{2}\sum_a e_a x_a^i
x_a^j}_{\substack{\text{Eigenschaften der}\\\text{Ladungsverteilung}}}
\underbrace{ \frac{\partial^2}{\partial x^i \partial x^j}
\frac{1}{R}}_{\substack{\text{Position des}\\ \text{Beobachters}}}\\
\frac{\partial^2}{\partial x^i \partial x^j} \frac{1}{R}
=\frac{\partial}{\partial x^j} \frac{x^i}{R^3}= \underbrace{
-\frac{\delta^{ij}}{R^3} + 3 \frac{x^ix^j}{R^5}}_{\text{Spur} = 0}\\
\laplace \frac{1}{R}=0
\end{align}
Der Anteil $\propto \delta^{ij}$ in $\frac{1}{2}\sum e_a x_a^i x_a^j$
verschwindet wegen der verschwindenden Spur.\\
Definition: Quadrupolmoment
\begin{equation}
Q^{ij}=\sum e_a \left( 3e x_a^i x_a^j-\delta^{ij} |\vec{x}_a|^2\right)
\end{equation}
mit 
\begin{align}
\sum Q^{ii}=\sum_{i,j} Q^{ij}\delta_{ij}=0
\end{align}
$Q^{ij}$ ist also Spufrei und symmetrisch
\begin{align}
\Rightarrow \phi^{(2)}=\frac{1}{6}Q^{ij} \frac{\partial^2}{\partial x^i
\partial x^j} \frac{1}{R} \propto \frac{1}{R^3}
\end{align}




\subsubsection{Multipolentwicklung allgemein}
\begin{align}
\phi=\phi^{(0)}+\phi^{(1)}+\phi^{(2)}+\phi^{(3)}+\phi^{(4)}+\ldots\\
\phi^{(n)}=\text{n-te Ableitung von }\frac{1}{R}\propto \frac{1}{R^{n+1}}
\end{align}
Charakteristische Eigneschaften der Ladungsverteilung\\
\begin{tabular}{c|c}
Multipole & $\phi^{(n)}$\\
\hline
$Q=$Gesamtladung$=$Monopol & $\sum \frac{e_a}{R}=\frac{Q}{R}$\\
$\vec{D}=$Dipolmoment &$\frac{D}{R^2}\cos(\theta)$\\
$Q_{ij}=$ Quadrupolmoment & $\frac{Q}{R^3}\frac{1}{4}\left(3
\cos^2(\theta)-1\right)$\\
\end{tabular}
n-te Ableitung:\\ 
(2n)-Polmoment $M{i_1\ldots i_n}$\\
symmetrische, spurfreie Tensoren n-ter Stufe mit $2n+1$  Komponenten\\\\
\begin{tabular}{c|c}
Ordnung & relevante Komponenten\\
\hline
Monopol & 1\\
Dipol & 1\\
Quadruopol & 2 (Hauptwerte)\\
$\vdots$&$\vdots$\\
2n-Pol & 2(n-1)
\end{tabular}\\

Multipolentwicklung entspricht der Entwicklung nach den Lösungen der
Laplace/Poisson-Gleichung nach Kugelflächenfunktionen
\paragraph{Einschub Quantenmechanik}
Schrödingergleichung
$\rightarrow \laplace$ für kin. Energie\\
für zentralsymmetrische Systeme $\rightarrow$ sphärische Koordinaten\\
Eignezustände des Drehimpulses
\paragraph{Laplace in sphärischen Koordinaten}
\begin{align}
\laplace\phi=\frac{1}{r^2}\frac{\partial}{\partial r}\left( r^2
\frac{\partial\phi}{\partial r} \right)+\frac{1}{r^2 \sin(\theta)}
\frac{\partial}{\partial \theta}\left( \sin(\theta) \frac{\partial\phi} 
{\partial \theta} \right)+\frac{1}{r^2 \sin^2(\theta)} \frac{\partial^2\phi} 
{\partial \Phi^2}
\end{align}
Lösung durch faktorisierung der Lösungsfunktion 
\begin{align}
\phi(r,\theta,\Phi)=R(r)P(\theta)f(\Phi)\\
\text{für R:  Lösungen}\propto c^{(1)}r^{l+1}+c^{(2)} r^{-l}\\
\text{für P:  }\frac{1}{\sin(\theta)}\frac{\mathrm{d}}{\mathrm{d}\theta}\left( \sin(\theta)
\frac{\mathrm{d}P(\theta)}{\mathrm{d}\theta}\right)+\left( l(l+1)-\frac{m^2}{\sin^2(\theta)} 
P(\theta) \right)=0\\
\text{für f:  }\frac{\mathrm{d}^2}{\mathrm{d}\Phi^2}f(\Phi)+m^2f(\Phi)=0\\
\Rightarrow \text{Lösung:} f=e^{im\Phi} \text{ , }m=0,\pm1,\pm2,\ldots\\
\text{für }P(\theta) P_l^m(\theta)=(-1)^m
(1-cos^2(\theta))^{\frac{m}{2}}\frac{\mathrm{d}^m}{\mathrm{d}cos(\theta)^m}P_l(cos(\theta))\\
l=0,1,2\ldots \text{ und } m=-l,-l+1,\ldots,l-1,l\\
\text{und }P_l(z)=\frac{1}{2^ll!}\frac{\mathrm{d}^l}{\mathrm{d}z^l}(z^2-1)^l\\
\text{darraus folgen Kugelflächenfunktionen}\\
Y_{lm}(\theta,\Phi)=\sqrt{\frac{2l+1}{4\pi}}\sqrt{\frac{(l-m)!}{(l+m)!}}
P^m_l(cos(\theta))e^{im\Phi}
\end{align}
Vollständigkeit
\begin{align}
\sum_{l=0}^\infty\sum_{m=-l}^l Y_{lm}(\theta,\Phi) Y_{lm}^*(\theta',\Phi')
=\delta(cos(\theta)-cos(\theta'))\delta(\Phi-\Phi')
\end{align}
Orthoonalität:
\begin{align}
\int_{-1}^{+1}\mathrm{d}cos\theta\int_0^{2\pi}\mathrm{d}\Phi Y_{lm}(\theta,\Phi)
Y_{l'm'}^*(\theta,\Phi)=\delta_{mm'}\delta_{ll'}
\end{align}
$\rightarrow$ beliebige Funktion auf der Einheitskugel ist Linearkombination
der $Y_{lm}$ (Basis)
\subsubsection{Multipolentwicklung (Wiederholung)}
\begin{align}
\phi(\vec{x})=\sum_a\frac{e_a}{|\vec{x}-\vec{x}_a|}
\rightarrow_{|\vec{x}|=R>>|\vec{x}_a|} \sum^\infty_{l=0}\frac{M_{2l}}{R^{l+1}}\\
M_{2l}\rightarrow2l-\text{Pole}\\
\laplace\phi(\vec{x})=-4\pi\delta(\vec{x})
\end{align}
Eigenfunktionen des Laplace-Operators\\
Entwicklung nach Kugelflächenfunktionen
\begin{align}
Y_{lm}(\theta,\Phi)\rightarrow P_l^m,P_l
\end{align}
Legendre Polynome\\
Vollständigkeit\\
Orthogonalität\\
\paragraph{Anwendung}
Kontinuierliche Ladungsverteilung $\rho(\vec{x})$\\
axialsymmetrisch\\
$\Phi$-Abhängigkeit nicht vorhanden\\
$\rightarrow$ nur $m=0$, nur $P_l$\\
explizit für Greensche Funktion
\begin{align}
\frac{1}{|\vec{x}-\vec{x}'|}&=\frac{1}{\sqrt{\vec{x}^2-2\vec{x}\vec{x}'+\vec{x}'^2}}\\
&=\begin{cases}\frac{1}{R\sqrt{1-2\frac{r'}{R}\cos\alpha+\frac{r'^2}{R^2}}}
\text{  } R>r' \text{ mit } t=\frac{r'}{R} \text { und }z=\cos\alpha\\
\frac{1}{r'\sqrt{1-2\frac{R}{r'}\cos\alpha+\frac{R^2}{r'^2}}} \text{  } r'>R
\text{ mit } t=\frac{R}{r'} \text { und }z=\cos\alpha\end{cases}\\
\underbrace{\frac{1}{\sqrt{1-2zt+t^2}}}_{\text{erzeugende
Funktion der Legende Polynome}}&=\sum^\infty_{l=0}t^l P_l(z)
\end{align}
Spezialfall: $z=1$:
\begin{align}
\frac{1}{\sqrt{(1-t)^2}}=\frac{1}{1-t}=\sum^\infty_{l=0}t^l
\underbrace{P_l(1)}_{=1}\\
\rightarrow
\phi(\vec{x})=-4\pi\int \mathrm{d}^3x'\frac{\rho(\vec{x}')}{|\vec{x}-\vec{x}'|}\\
=-4\pi\sum^\infty_{l=0}\frac{1}{R^{l+1}}\underbrace{\int \mathrm{d}r' r'^2\int \mathrm{d}z 2\pi
\rho(r',z)r'^l P_l(z)}_{M_{2l}}
\end{align}
\paragraph{Anwendung für allgemeine Ladungsverteilung} $\rho(\vec{x})$
Greensche Funktion: $\laplace
G(\vec{x},\vec{x}')=\delta^{(3)}(\vec{x}-\vec{x}')$\\
Polarkoordinaten: $\vec{x}\rightarrow R,\theta,\Phi$\\
$\vec{x}'\rightarrow R',\theta',\Phi'$ (fest)\\
Vollständigkeit der $Y_{l,m}$:
\begin{align}
\delta^{(3)}(\vec{x}-\vec{x}')=\frac{1}{R^2}
\delta(R-r')\delta(\Phi-\Phi')\delta(\theta-\theta')\\
=\frac{1}{R^2}\delta(R-r')\sum_{l,m}Y_{l,m}(\theta,\Phi)Y^*_{l,m}(\theta',\Phi')\\
\end{align}
Somit folgt als Ansatz für $G(R,\theta,\Phi,r',\theta',\Phi')$
\begin{align}
G(R,\theta,\Phi,r',\theta',\Phi')=
\sum_{l,m}A_{l,m}(R,r',\theta',\Phi')Y_{l,m}(\theta,\Phi)\\
\rightarrow
A_{l,m}(R,r',\theta',\Phi')\underset{\text{Ansatz}}{=}G_R(R,r')
Y_{l,m}(\theta',\Phi')\\
\Rightarrow \frac{1}{R^2}\frac{\partial}{\partial R}\left( R^2
\frac{\partial}{\partial R} \right)
G_R-\frac{l(l+1)}{R^2}G_R=-\frac{1}{R^2}\delta(R-r')
\end{align}
Ansatz: Potenzen $G_R(R,r')=a(r')R^b$\\
$\rightarrow$ Lösungen: $b=l$, oder $-(l+1)$\\
$\rightarrow$ für $R>r'$ und $G_R \overset{R\rightarrow\infty}{\rightarrow}0$:
$G_R=N_l\frac{r'^l}{R^{l+1}}$
Ergebnis für $G$:
\begin{align}
G(\vec{x},\vec{x}')=-\sum_{l,m}\frac{1}{2l+1}\frac{r'^l}{R^{l+1}}
Y_{l,m}(\theta,\Phi)Y^*_{l,m}(\theta',\Phi')
\end{align}
für das Potential gilt dann:
\begin{align}
\phi(\vec{x})&=-4\pi\int \mathrm{d}^3x' G(\vec{x},\vec{x}')\rho(\vec{x}')\\
&=\sum_{l,m}\frac{4\pi}{2l+1}\frac{1}{R^{l+1}}Y_{l,m}(\theta,\Phi)M_{l,m}\\
\text{mit} M_{lm}&=\int \mathrm{d}r' (r')^2 \int \mathrm{d}\cos\theta'\int \mathrm{d}\Phi' (r')^l
Y^*_{l,m}(\theta',\Phi') \rho(\vec{x}')\\=\text{sphärische Multipolmomente}
\end{align}
Somit können wir die Eigenschaften von $\rho$ faktorisieren.
\paragraph{Anwendung: Wechselwirkungsenergie}
für Systeme von Ladungen in vorgegebenem Potential $\phi(\vec{x})$\\
Variation von $\phi(\vec{x})$ klie über Längen des Systems\\
$\rightarrow$ Entwicklung nach Potenzen von $\vec{x}'$\\
\begin{align}
\phi(\vec{x}')=\phi(0)+\vec{x}'\vec{\nabla}\phi(0)+\frac{1}{2}x_i'x_j'
\frac{\partial^2\phi}{\partial x_j'\partial x_i'}(0)+\ldots
\end{align}
Wechselwirkungsengergie für ein System von Punktladungen
\begin{align}
U=\sum_a e_a\phi(\vec{x_a})\\
\text{Einsetzen:}
U=Q\phi(0)+\underbrace{\vec{D}}_{\text{Orientierung Winkel}\cos\alpha}
\overbrace{\vec{\nabla}\phi(0)}_{-\vec{E}}+ \frac{1}{6}Q_{ij}\frac{\partial^2\phi}{\partial x_j'\partial x_i'}(0)+\ldots\\
\frac{\partial U}{\partial\alpha}\rightarrow \text{ Drehmoment}\\
U=Q\phi-\vec{D}\vec{E}+\ldots
\end{align}
\subsection{Magnetostatik}
\begin{align}
\vec{E}\rightarrow\vec{B}\\
\rho\rightarrow\vec{j} \text{    Rotation}\\
\phi\rightarrow\vec{A}
\end{align}
Maxwell-Gleichungen:
\begin{align}
\dot{\rho}=0, & \dot{\vec{j}}=0, & \vec{j}\neq0\\
\rightarrow \text{ese gibt Lösungen: } \dot{\vec{E}}=0, &\dot{\vec{B}}=0,
&\vec{B}\neq0
\end{align}
phsikalisch statische Stromverteilung $=$ bewegte Ladungen\\
$\Rightarrow$ Zeitabhängigkeit $\Rightarrow \dot{\rho}\neq0$, 
$\dot{\vec{E}},\dot{\vec{B}}\neq0$\\
$\Rightarrow$ Magnetostatik ist Nährung\\
physikalische Nährung: zeitliche Mittelung!\\
Diese ist möglich falls:
\begin{align}
\overline{\dot{\vec{E}}}=\frac{1}{T}\int_0^T \mathrm{d}t
\frac{\mathrm{d}\vec{E}}{\mathrm{d}t}=\underbrace{\frac{1}{T}}_{\text{groß, }T\rightarrow\infty}
\left( \vec{E}(T)-\vec{E}(0) \right)\rightarrow0
\end{align}
Somit ist die Vorraussetzung $\vec{E}(t)$ beschränkt.\\
$\rightarrow$bewegte Ladungen, die sich in einem endlichen räumlichen Gebiet mit
endlichen Impulsen bewegen.

\subsection{Magnetostatik}
Ströme - bewegte Ladungen\\
Zeitmittelung
\begin{align}
\bar{\dot{\vec{E}}}=\frac{1}{T}\int_0^T \mathrm{d}t
\frac{\mathrm{d}\vec{E}}{\mathrm{d}t}=\frac{\vec{E}(0)-\vec{E}(0)}{T}\rightarrow 0 \text{ für }
T\rightarrow \infty \text{ und } \vec{E}(t) \text{ ist begentzt}
\end{align}
Stationäre Bewegung
\begin{align}
\rightarrow \bar{\dot{\rho}}=0\\
\text{Kontinuitätsgleichung} \vec{\nabla}\bar{{\vec{j}}}=0\\
0=\int \mathrm{d}^3x\left( \frac{\partial}{\partial x^i}\bar{j}^i \right)x^k=
-\int \mathrm{d}^3x \bar{j}^i \left( \frac{\partial}{\partial x^i}x^k \right)=-\int \mathrm{d}^3x
\bar{j}^k
\end{align}
Vorraussetzunge für Anwendbarkeit der Magnetostatik
\begin{itemize}
  \item Ladungen zu allen Zeiten in endlichem Raumgebiet
  \item endliche Impulse
  \item ausreichend lange Messzeit (schnelle Anpassung der Ladungen an geänderte
  äußere Bedingungen)
  \item schwache Magnetfelder
\end{itemize}

\paragraph{Grundlgeichungen der Magnetostatik}
\begin{align}
\vec{\nabla}\bar{\vec{B}}&=0 \rightarrow \text{Ansatz:
}\bar{\vec{B}}=\vec{\nabla}\times \bar{\vec{A}}\\
\vec{\nabla}\times\bar{\vec{B}}&=\frac{4\pi}{c}\vec{j}
\end{align}
Eichbedingung: Coulombeichung $\vec{\nabla}\vec{A}=0$
\begin{align}
\vec{\nabla}\times\vec{\nabla}\times\bar{\vec{A}}=\vec{\nabla}(
\vec{\nabla}\bar{\vec{A}})-\laplace \bar{\vec{A}}=-\laplace \bar{\vec{A}}\\
\rightarrow \laplace \bar{\vec{A}}=-\frac{4\pi}{c}\bar{\vec{j}} \text{ Siehe
Elektrostatik}
\end{align}
Als Lösung folgt also:
\begin{align}
\bar{\vec{A}}(\vec{x})=\frac{1}{c}\int \mathrm{d}^3x' \frac{\bar{\vec{j}}(\vec{x'})}
{|\vec{x}-\vec{x}'|}
\rightarrow \bar{\vec{B}}(\vec{x})=\frac{1}{c}\int \mathrm{d}^3x'
\frac{\bar{\vec{j}}(\vec{x'})\times\left( \vec{x}-\vec{x}' \right)}
{|\vec{x}-\vec{x}'|} \text{  für} \bar{\vec{A}},\bar{\vec{B}}\rightarrow 0\text{
im Unendlichen}
\end{align}
\subparagraph{Multipolentwicklung (Feld in große Abständen))}
\begin{align}
\frac{1}{|\vec{x}-\vec{x}'|}=\frac{1}{R}-\vec{x}'\vec{\nabla}\frac{1}{R}+\ldots\\
\frac{1}{|\vec{x}-\vec{x}'|}=\frac{1}{R}-\frac{\vec{x}'\vec{x}}{R^3}+\ldots\\
\text{in }\bar{\vec{A}} \text{eingesetzt}
\text{1. Term:} \int \mathrm{d}^3 x' \frac{\bar{\vec{j}}(\vec{x}')}{R}=0
\text{(d.h. keine Monopole)}\\
\text{2. Term:} \frac{1}{cR^3}\int \mathrm{d}^3 x'\bar{\vec{j}}(\vec{x}')(\vec{x}'\vec{x})\\
\end{align}
für Punktladungen gilt also
\begin{align}
\vec{j}(\vec{x}')&=\sum_a e_a \vec{v}_a \text{ , }
\vec{v}_a=\frac{\mathrm{d}\vec{x}_a}{\mathrm{d}t} %\\
\text{ auszurechnen}\\
\overline{\sum_a e_a \vec{v}_a(\vec{x}_a\vec{x})} &=\overline{\sum_a e_a
\frac{\mathrm{d}\vec{x}_a}{\mathrm{d}t}(\vec{x}_a\vec{x})}\\
 &=\underbrace{\overline{\frac{1}{2}\frac{\mathrm{d}}{\mathrm{d}t}\sum_a e_a
\vec{x}_a(\vec{x}_a\vec{x})}}_{\substack{=0 \text{ falls Ladungen in einem}\\ \text{ räumlich
begrentzten Bereich bleiben}}} +\frac{1}{2}\overline{\sum_a e_a
\frac{\mathrm{d}\vec{x}_a}{\mathrm{d}t}(\vec{x}_a\vec{x})}-\frac{1}{2}\overline{\sum_a e_a \vec{x}_a
(\vec{v}_a\vec{x})}\\
 &=\frac{1}{2}\overline{\sum_a e_a \left( \vec{v}_a
(\vec{x}_a\vec{x})-\vec{x}_a
(\vec{v}_a\vec{x}) \right)}=\frac{1}{2}\sum_a e_a \overline{\left(
\vec{x}_a\times\vec{v}_a \right)\times\vec{x}}
\end{align}
$\rightarrow$ Definition:
magnetsiches Moment: $\vec{m}=\frac{1}{2c}\sum_a e_a\overline{\left(
\vec{x}_a\times\vec{v}_a \right)}$\\
$\Rightarrow$ 2.Term in der Multipolentwicklung für $\bar{\vec{A}}$\\
\begin{align}
\vec{A}=\frac{1}{R^3}\vec{m}\times\vec{x}\\
\vec{B}=\vec{\nabla}\times\left( \frac{1}{R^3}\vec{m}\times\vec{x} \right)\\
=\frac{3\vec{n}(\vec{m}\vec{n})-\vec{m}}{R^3}\text{ ,
}\vec{n}=\frac{\vec{x}}{R}
\end{align}
Drehimpuls: $\vec{L}=\sum_a m_a \vec{x}_a\times\vec{v}_a$
falls Verhältnis $\frac{e_a}{m_a}$ für alle Teichen gleich ist gilt:
\begin{equation}
\vec{m}=\frac{e}{2cm}\vec{L}
\end{equation} 
magnetisches Dipolmoment $\leftarrow$ mechanischer Drehimpuls.\\
Lorentz-Kraft: $\bar{\vec{F}}=\sum
\frac{e}{c}\overline{\vec{v}\times\vec{B}}=\overline{\frac{\mathrm{d}}{\mathrm{d}t}\left(\ldots
\right)}=0$\\
Drehmoment: 
\begin{align}
\bar{\vec{K}}=\sum\frac{e}{c}\overline{\underbrace{\vec{x}\times(\vec{v}
\times\vec{B})}_{\vec{v}(\vec{x}\vec{B}-\vec{B}(\vec{v}\vec{x})}}\neq0\\
=\overline{\vec{v}(\vec{x}\vec{B}}-\xout{\frac{1}{2}\left(
\overline{\vec{B}\frac{\mathrm{d}\vec{x}^2}{\mathrm{d}t}} \right)}\\
%\bar{\vec{K}}=\sum_a \frac{e_a}{c} \overline{\vec{v_a}(\vec{x_a}\vec{B}}\\
%=\vec{m}\times\bar{\vec{B}}
\end{align}
\subparagraph{Larmor Präzession}
\begin{align}
\vec{m}\times\bar{\vec{B}}=\frac{2mc}{e}\dot{\vec{m}}\\
\dot{\vec{m}}=-\vec{\Omega}\times\vec{m}\\
\vec{\Omega}=\frac{e}{2mc}\vec{B}\\
\vec{K}=\frac{\mathrm{d}\vec{L}}{\mathrm{d}t}=\frac{\Delta\vec{L}}{\Delta t}\\
\Delta t \text{groß }>>\text{Zeitmittelung}
\end{align}
Vorsicht, alle Mittelungen hier sind nur über die Mikroskopischen Skalen gedacht
und kein Zeitmittel über den gesamten betrachteten Zeitraum.

\subparagraph{Magnetischer Dipol in äußerem Feld}
$\vec{B}$ sei zeitlich konstant\\
Herleitung über den Lagrangeformalismus. (in Magnetostatik, also $\phi=0$)\\
(allgemein: $ S=-\frac{1}{c^2}\int d^4x j^\mu A_\mu\rightarrow \frac{1}{c}\int
\mathrm{d}t \int \mathrm{d}^3x \vec{j}\vec{A}$)
\begin{align}
L&=\sum_a\frac{e_a}{c}\vec{v}_a\vec{A} \text{ für eine Punktladung}\\
&|\vec{A}=\frac{1}{2}\vec{B}\times\vec{x} \\&|\text{ für ein konstantes und
 Ortsunabhängiges }B\text{-Feld}
&=\sum_a\frac{e_a}{2c}\vec{v}_a(\vec{B}\times\vec{x})=\sum_a\frac{e_a}{2c}
\vec{B}(\vec{x}\times\vec{v}_a)=\vec{m}\vec{B}
\end{align}
In der Elektrostatik galt:
\begin{align}
L_E=\vec{D}\vec{E} \text{  Dipol-Energie}
\end{align}
$\rightarrow$Kraftwirkungen

%XXXX Ergänzungen der Magnetostatik Blatt Y1

\subsection{Elektrodynamik im Kontinua}
mikroskopisch: 
\begin{itemize}
  \item Punktförmige Ladungen im Vakuum
  \item teilweise gebunden, halten sich in  begrenzten Raumgebieten auf
  \item ortsfeste Ladungen deren Bewegung man vernachlässigt (Atomkerne)
  \item mehr oder weniger frei bewegliche Elektronen
  \item magnetische Momente und Spins treten auf
\end{itemize}
$\Rightarrow$ Quantenmechanik oder Quantenfeldtheorie müssen zur Beschreibung
verwendet werden.
\subparagraph{Ziel: makroskopische Eigenschaften}
räumliche Mittelung (groß/klein)
Wirft Fragen auf:
\begin{itemize}
  \item Wie groß darf man die Volumina dieser Mittelung wählen
  \item Ab wann beginnt die Nährung eines Kontinuums zu nicht
  vernachlässigbarenFehlern zu führen
\end{itemize}
Diese Mittelung wird hier nicht weiter notiert sondern für alle Größen implizit
angenommen.

\paragraph{Überblick}
(keine mikroskopischen Eigenschaften betrachtet, dies ist Thema der
Statistischen Physik)
\begin{itemize}
  \item Leiter und Nichtleiter (Dielektrika)
  \item frei bewegliche Ladungen - verschiebbare Ladungen
  \item Magnetismus
  	\begin{itemize}
  		\item Paramagnetismus
  		\item Diamagnetismus
  		\item Ferromagnetsimus 
	\end{itemize} 
\end{itemize}
\subsubsection{Elektrostatik von Leitern}
Frei bewegliche Ladungen\\
$\Rightarrow$ jedes elektrische Feld führt dazu, dass die Ladungen ihre
Positionen ändern, d.h. eine Bewegugn dieser Ladungsträger findet statt\\
$\Rightarrow$ ein elektrisches Feld erzeugt Ströme \\
Die Ladungsträger sind jedoch nicht völlig frei beweglich, durch Streuung an
anderen Elektronen und Atomrümpfen verliegen sie Energie und werden
Abgebremst.\\
$\Rightarrow$ Ströme dissioieren Energie\\
$\rightarrow$ keine \underline{stationären} Ströme\\
$\Rightarrow$ $\vec{E}$ im Inneren von Leitern verschwindet schnell\\
$\Rightarrow$ frei bewegliche Ladungen: an der Oberfläche\\
$\rightarrow$ Problemstellung für Elektrostatik
\begin{itemize}
  \item Feld im Außenraum
  \item Ladungsverteilung auf den Oberflächen der Leiter
\end{itemize}
Im Außenraum:
\begin{itemize}
  \item $\vec{\nabla}\vec{E}=0$
  \item $\vec{\nabla}\times\vec{E}=0$
\end{itemize} 
Ansatz: $\vec{E}=-\vec{\nabla}\phi$ mit $\laplace \phi=0$\\
und an der Oberfläche
%XXXX Zeichnung Y2
$E_z$ in der Nähe der Oberfläche: $\neq0$
für homogene Oberflächen (wir nehmen an, dass die betrachtete Auflösung der
Oberfläche so groß ist, dass Atome und Elektronenverteilung quasikontnuierlich
sind):\\
$\rightarrow$ $\frac{\partial E_z}{\partial x}$,$\frac{\partial E_z}{\partial
y}$ bleiben endlich.\\
Wegen $\vec{\nabla}\times\vec{E}=0 \Rightarrow$ auch $\frac{\partial
E_x}{\partial z}$ und $\frac{\partial E_y}{\partial z}$ sind endlich\\
$\Rightarrow$ $E_x$,$E_y$ sind stetig in $z$-Richtung\\
$\Rightarrow$ $\vec{E}_{\text{tangential}}=0$ auf der Oberfläche\\
$\Rightarrow$ wegen $\vec{E}=-\vec{\nabla}\phi$ sind somit alle
Leiteroberflächen in der Elektrostatik Äquipotentialflächen.\\
 \\
Maxwellgleichungen mit Randbedingungen
\begin{itemize}
  \item $\vec{E}_{\text{tangential}}=0$ auf Oberflächen
  \item $E_{\text{normal}}=4\pi\sigma$ (Herleitung unten)
\end{itemize}

Ladungsverteilung auf Oberflächen\\
Flächenladungsdichte $\sigma=\rho \mathrm{d}z$ ($\rho=\frac{Q}{\mathrm{d}V}=\frac{Q}{\mathrm{d}F
\mathrm{d}z}=\sigma \frac{1}{\mathrm{d}z}$)
%XXXX Einfügen der Grafik Y3
\begin{align}
\vec{\nabla}\vec{E}&=4\pi\rho\\
\underset{\text{Volumenintegral}}{\Rightarrow}\int \mathrm{d}^3x \vec{\nabla}\vec{E}&=
4\pi \int \mathrm{d}^3x \rho\\
\underset{\text{Satz von Gauß}}{\Rightarrow}\oint \mathrm{d}\vec{f}\vec{E}&=4\pi\oint
\mathrm{d}\vec{f}\vec{n}\sigma\\
\Rightarrow E_{\text{normal}}&=4\pi\sigma 
\end{align}

\subparagraph{Elektrostatik von Nichtleitern}
Keine frei beweglichen Ladungsträger\\
$\rightarrow$ $\vec{E}\neq0$ im Inneren möglich\\
Bezeichnungen:
\begin{itemize}
  \item mikroskopisch $\vec{e}$
  \item makroskoopisch $\vec{E}=\frac{1}{V}\int_V \mathrm{d}^3x \vec{e}(\vec{x})$
\end{itemize}

Maxwellgleichungen nach räumlicher Mittelung
\begin{align}
&\vec{\nabla}\vec{E}=4\pi\rho&\vec{\nabla}\times\vec{E}=0 
\end{align}
wichtiger Fall: keine zusätzlichen Ladungen auf dem Nichtleiter
\begin{align}
\rightarrow \int \mathrm{d}^3x \rho =0
\end{align}
für beliebig geformtes Volumen $\Rightarrow$
\begin{align}
\rho=-\vec{\nabla}\vec{P}
\end{align}
Betrachte Oberlfäche, die den Körper ganz einschließt
%XXXX Zeichnung Y4 neben Gleichung einfügen
 \begin{align}
 0=\int \mathrm{d}^3x\rho=-\int \mathrm{d}^3x \vec{\nabla}\vec{P}\underset{\text{Gauß}}{=}-\oint
 \mathrm{d}\vec{f}\\
 \Rightarrow
 \vec{P}=0 \text{ im Außenraum}
 \end{align}
Bemerkung: $\vec{P}$ ist damit nicht eindeutig bestimmt
\begin{align}
\vec{P}\rightarrow \vec{P}+\vec{\nabla}\times\vec{f} 
\end{align}
Führt für beliebige Vektorfunktion $f$ zu keiner Änderung der physikalischen
Aussage. \\
$\vec{P}$ wird dielektrische Polarisation genannt\\
$\vec{P}\neq0$ in polarisierbaren Medien\\
betrachte nun das Volumenelemnent an der Oberfläche
%XXXX Füge GRafik Y5 ein 
\begin{align}
\Rightarrow \underline{P_{\text{Normal}}=\sigma}
\end{align} 
$\vec{P}$ ist die Dichte des elektrischen Dipolmoments
\begin{align}
\int \mathrm{d}^3x \vec{x}\rho(\vec{x})=-\int \mathrm{d}^3x \vec{x}(\vec{\nabla}\vec{P})\\
\rightarrow \int \mathrm{d}^3\!x x_k\rho(\vec{x})&=-\int \mathrm{d}^3x
\left(\partial_i(x_kP_i)-P_i\underbrace{(\partial x_k)}_{\delta_{ik}}\right)\\
&=\underbrace{\int \mathrm{d}^3x\left(\partial_i(x_kP_i)\right)}_{\text{Satz von
Gauß+Randterme}=0} \int \mathrm{d}^3x P_k\\
=\int \mathrm{d}^3x P_k
\end{align}
zusammen ergibt sich
\begin{align}
&\vec{\nabla}\vec{E}=4\pi\rho &\rho=-\vec{\nabla}\vec{P}\\
\Rightarrow &\vec{\nabla}\vec{D}=0 & \vec{D}=\vec{E}+4\pi\vec{P}
\end{align}
bei zusätzlichen Ladungen $\vec{\nabla}\vec{D}=4\pi\rho_{\text{extern}}$\\
$\vec{D}$ heißt dielektrische Induktion oder (di)elektrische Verschiebung\\
Feld $\vec{E}$ verschiebt im mikroskopischen Ladungen und erzeugt damit eine
Dipoldichte, die von den Materialeigenschaften abhängt.\\
Zusätzliche Informationen: Zusammenhang zwischen $\vec{E}$ und $\vec{D}$\\
allgemein $\vec{D}=\vec{D}(\vec{E})$\\
Potenzreihenentwicklung (für schwache Felder):\\
\begin{equation}
D_i=D_{i,0}+\sum_{k=1}^3 \epsilon_{ik}E_k+\sum_{k=1}^3 \epsilon_{ikl}E_kE_l \ldots
\end{equation} 
\begin{itemize}
  \item $\vec{D}_0\neq0$ z.B. in Kristallen möglich (Permanenz)
  \item i.a. ist $\epsilon$ ein Tensor
  \item einfachste Situation: $\vec{D}=\epsilon\vec{E}$
\end{itemize}

allgeimein gilt $\epsilon>1$.

\subsection{Wiederholung}
\subsubsection{Magentostatik}
Zeitmittelung ("schnelle" Bewegungen)\\
$\bar{\dot{\vec{E}}}=0$\\
Multipolentwicklung, magnetisches Moment\\
\subsubsection{Elektrostatik für Leiter (Randwertprobleme)}
\subsubsection{Elektrostatik von Materie}
räumliche Mittelung\\
mikroskopisches elektrisches Feld $\vec{e}(\vec{x})$\\
makroskopisches elektrisches Feld $\vec{E}(\vec{x})$\\
\begin{align}
\vec{E}(\vec{x})=\frac{1}{V(\vec{x})}\int_{V(\vec{x})} \mathrm{d}^3x' \vec{e}(\vec{x}')
\end{align}

\subsection{Magnetostatik in Materie}
mikroskopisch $\vec{b}(\vec{x})$\\
\begin{align}
\vec{\nabla}\vec{b}(\vec{x})=0 & \vec{\nabla}\times\vec{b}(\vec{x})=\frac{1}{c}
\dot{\vec{e}}(\vec{x})+\frac{4\pi}{c}\rho\vec{v}
\end{align}
gemitteletes Feld\\
(Zeitmittelung: $\bar{\dot{\vec{e}}}=0$\\
raumliche Mittelung $\vec{b}\rightarrow\vec{B}$\\
\begin{align}
\vec{\nabla}\vec{b}=0 & \vec{\nabla}\times\vec{B}(\vec{x})=\frac{4\pi}{c}\overline{\rho\vec{v}}
\end{align}
Materie:
%Bild Y6, Y7 einfügen

\begin{align}
\int_{\partial V} \mathrm{d}\vec{f} \overline{\rho \vec{v}}=0
\end{align}
Ansatz
\begin{align}
\overline{\rho \vec{v}}=c\vec{\nabla}\times\vec{M}
\end{align}
im Allgemeinen gilt in Materie $\vec{M}\neq0$, denn
\begin{align}
c \int_{F} \mathrm{d}\vec{f} \vec{\nabla}\times\vec{M}=c \oint_{\partial F} \mathrm{d}\vec{x}
\vec{M}=0
\end{align}
(Analogie: Polarisation $\int_V \mathrm{d}^3x\rho=\int \mathrm{d}^3x\vec{\nabla}\times\vec{P}=0$\\
$\overline{\rho \vec{v}}$ einsetzen:
\begin{align}
\vec{\nabla}\times\vec{B}=4\pi \vec{\nabla}\times\vec{M}\\
\text{Def.:} \vec{B}-4\pi\vec{M}=\vec{H}\\
\Rightarrow \vec{\nabla}\times\vec{H}=0
\end{align}
\begin{itemize}
\item $\vec{M}$ ist nicht eindeutig:
\begin{align}
\vec{M}\rightarrow\vec{M}+\vec{\nabla}f\\
\vec{\nabla}\times(\vec{\nabla}f)=0
\end{align}
festlegen: mikroskopische Eigenschaften
\end{itemize}
$\vec{M}=$Dichte des magnetischen Moments
\begin{align}
\frac{1}{2c}\int \mathrm{d}^3x \vec{x}\times\overline{\rho\vec{v}}=\frac{1}{2}\int \mathrm{d}^3x
\vec{x}\times(\vec{\nabla}\times\vec{M})\\
=\frac{1}{2}\underbrace{\int_{\partial V}
\vec{x}\times(\mathrm{d}\vec{f}\times\vec{M}})_{=0}-\frac{1}{2}\int \mathrm{d}^3x
\underbrace{(\vec{M}\times\vec{\nabla})\times\vec{x}}_{=-2\vec{M}}\\
=\int \mathrm{d}^3x \vec{M}
\end{align}
\subparagraph{Bemerkungen zu $\vec{M}$}
\begin{itemize}
\item $\vec{M}=0$ in magnetischen Materialien
\item einfache Situation(häufig!): $\vec{B}=\mu\vec{H}$\\
$\mu=$ magnetische Permiabilität\\
$\vec{M}=\chi\vec{H}$\\
$\chi=$magnetische Suzeptibilität $\chi=\frac{\mu-1}{4\pi}$
\item Aus der Thermodynamik lässt sich folgern $\Rightarrow \mu>0$
\item $\mu \ll 1$, relativistischer Effekt $\propto \frac{v^2}{c^2}$
\item Dies lässt sich vor allem in Kristallstrukturen erreichen
\item Anisotrope Materialien $\rightarrow B_i=\mu_{ik}H^k$
\item im Allgemeinen (vor allem bei starken Feldern): $\vec{B}=\vec{B}(\vec{H})$
\end{itemize}

\subsection{Elektromagnetische Wellen}
Bisher: $\dot{\vec{E}}=\dot{\vec{B}}=0$
Jetzt: $\dot{\vec{E}}\neq0\text{ , } \dot{\vec{B}}\neq0$ im Vakuum: $\rho=0$, $\vec{j}=0$
\begin{align}
\vec{\nabla}\times\vec{E}=-\frac{1}{c}\dot{\vec{B}} & \vec{\nabla}\vec{E}=0\\
\vec{\nabla}\times\vec{B}=-\frac{1}{c}\dot{\vec{E}} & \vec{\nabla}\vec{B}=0
\end{align}
Für dieses Problem gibt es nicht triviale Lösungen!\\
für 4-Potentiale: Coulomb-Eichung
\begin{align}
A^0=\phi=0 & \vec{\nabla}\vec{A}=0\\
\rightarrow \vec{E}=-\frac{1}{c}\dot{\vec{A}} \text{,} & \vec{B}=\vec{\nabla}\times\vec{A}\\
\text{z.B.} \vec{\nabla}\times\vec{B}=\underbrace{\vec{\nabla}\times\vec{\nabla}\times\vec{A}}_{-\laplace \vec{A}+\vec{\nabla}(\vec{\nabla}\vec{A})}
\overset{=}{!}-\frac{1}{c^2}\frac{\partial^2}{\partial t^2}\vec{A}
\end{align}
$\rightarrow$ Wellengleichung
\begin{align}
\left(\laplace-\frac{1}{c^2}\frac{\partial^2}{\partial t^2}\right)\vec{A}=0
\end{align}
\subparagraph{in 4-Schreibweise}
Maxwellgleichungen: $\partial_\mu F^{\mu0}=0$ (für $j^\nu=0$)
Einsetzen:
\begin{align}
F^{\mu\nu}=\partial^\mu A^\nu -\partial^\nu A^\mu\\
\Rightarrow \partial_\mu\partial^\mu A^\nu -\partial^\nu (\partial_\mu A^\mu\\)=0
\end{align}
Mit der Lorentzeichung ($\partial_\mu A^\mu=0$) ergibt sich:
\begin{align}
\partial_\mu\partial^\mu A^\nu=0, & \text{d'Alembert-Operator}\\
\dalembert=\partial_\mu\partial^\mu=\partial^2=\frac{1}{c^2}\frac{\partial^2}{\partial t^2}-\laplace
\end{align}
Dies gilt auch für jede Komponente von $A^\mu$,$\vec{E}$,$\vec{B}$ (linare Operatoren) im folgenden $f=A^\mu$,$\vec{E}$,$\vec{B}$
\begin{align}
\dalembert f=0\\
\left(\frac{1}{c^2}\frac{\partial^2}{\partial t^2}-\laplace\right) f=0
\end{align}
\subparagraph{Lösung der Wellengleichung}
zunächst 1-dimensional $f(x,t)$
\begin{align}
\left(\frac{\partial^2}{\partial t^2}-c^2\frac{\partial^2}{\partial x^2}\right) f=0
\end{align}

neue Variable: $\xi=t-\frac{x}{c}$, $\eta=t+\frac{x}{c}$\\
invertiert: $t=\frac{\xi+\eta}{2} $, $x=c\frac{\eta-\xi}{2}$
\begin{align}
\frac{\partial}{\partial\xi}=\frac{1}{2}\left(\frac{\partial}{\partial t}-c \frac{\partial}{\partial x} \right)
\frac{\partial}{\partial\eta}=\frac{1}{2}\left(\frac{\partial}{\partial t}+c \frac{\partial}{\partial x} \right)\\
\Rightarrow \frac{\partial^2}{\partial \xi \partial\eta}f=0
\end{align}
allgemeinen Lösung: $f(\xi,\eta)=f_1(\xi)+f_2(\eta)$\\
beziehungsweise $f(x,t)=f_1(x-ct)+f_2(x+ct)$\\
($f_1$,$f_2$ sind unbestimmt)\\
(Randwerte!)\\
z.B.: $f_2=0$, also $\vec{E}=\vec{E}(x-ct)$\\
Feldkomponenten habne gleiche Werte für alle 
\begin{align}
x=ct+\text{Konstanten}
\end{align}
$\rightarrow$ Feldkonfiguration bewegti sich in x-Richtung mit Lichtgeschwindigkeit (analog  für $f_1=0$)

\subparagraph{Monochromatische Wellen}
Lösung mit zeitabhängigkeit $\propto \cos(\omega t+\alpha)$\\
Einsetzen in die Wellengleichung:
\begin{align}
\frac{\partial^2}{\partial t^2 }f=-\omega^2f\\
\laplace f+ \frac{\omega^2}{c^2}f=0\\
\Rightarrow f\propto \cos(\omega (t\pm\frac{x}{c})+\alpha_0)
\end{align}
Idee komplexe Funktionen
\begin{align}
e^{i\alpha}=\cos(\alpha)+i\sin(\alpha) , Re(e^{i\alpha})=\cos(\alpha)\\
\text{Regel: } e^{i\alpha}e^{i\beta}=e^{i(\alpha+\beta)}\\
\text{Für monochromatische Wellen: } \vec{A}=Re(\vec{A}_0 * e^{i
\omega(t-\frac{x}{c})})
\end{align}

\subsubsection{Wiederholung}
Wellengleichung im Vakuum 
\begin{align}
\dalembert A^\mu=0\\
\left( \laplace - \frac{1}{c^2} \frac{\partial^2}{\partial t^2} \right)\vec{A}=0
\end{align}
(innerhalb der Lorentz-, Coulomb-Eichung)\\
Allgemeine Lösung for $A^i$,$E^i$,$B^i\rightarrow f(x,t)$
\begin{equation}
f=f_1(x-ct)+f_2(x+ct)
\end{equation}
Basislösungen\\
Monochromatische Wellen $f_1\propto \cos(\omega \left( t-\frac{x}{c}
\right))$\\
oder $\vec{A}=Re(\underbrace{\vec{A_0}}_{constant} *
e^{i\omega(t-\frac{x}{c})}$\\ 
für beliebige Richtungen lässt sich dies verallgemeinern:\\
Einheitsvektor $\vec{n}$ 
\begin{align}
\vec{A}=Re(\underbrace{\vec{A_0}}_{constant} * e^{i(\vec{k}\vec{x}-\omega t)}\\
\text{mit }\vec{k}=\frac{\omega}{c}\vec{n}\\
\text{und }\vec{A}_0= \text{konstant, komplex}\\
\vec{k}\vec{x}-\omega t =\text{ Phase, Phasenwinkel, }\phi\\
\end{align}
Eine Phasendifferenz $2\pi \rightarrow \text{in} x$\\
Wellenlänge: $\frac{\omega}{c}\lambda=2\pi$\\
$\lambda=\frac{2\pi c}{\omega}$\\
$\Rightarrow$ für $\vec{E}$,$\vec{B}$
\begin{align}
\vec{E}&=-\frac{1}{c}\dot{\vec{A}}=i\frac{\omega}{c}\vec{A}\\
\vec{B}&=\vec{\nabla}\times\vec{A}=i\vec{k}\times\vec{A}=i\frac{\omega}{c}\vec{n}\times\vec{A}\\
&=i\vec{k}\times\frac{c}{i\omega}\vec{E}=i\frac{\omega}{c}\vec{n}\times\frac{c}{\omega}\vec{E}\\
\Rightarrow \vec{B}&=i\vec{n}\times\vec{A}\\
\text{Eichbedingung}: & \vec{\nabla}\vec{A}=0
\end{align}
%XXXX Einfügen Bild Y8
\begin{align}
\Rightarrow \vec{n}\vec{A}=0
\end{align}
$\Rightarrow$ elektromagnetische Wellen sind transversal.\\
$\Rightarrow \vec{E}\perp\vec{B}\perp\vec{n}\perp\vec{E}$\\
und $|\vec{E}|=|\vec{B}|$\\
\paragraph{in 4-Schreibweise}
\begin{align}
\vec{k},\omega \rightarrow k^\mu&=(\frac{\omega}{c},\vec{k})\\
\text{Phase: }k_\mu x^\mu&=kx=\omega t- \vec{k}\vec{x}\\
\vec{A}^\mu&=\vec{A}^\mu_0 e^{ikx}\\
\text{Wellengleichung: } \dalembert A^\mu&=0\\
(-ik)^2 A^\mu&=0 \\
\Rightarrow &k^2=0 \text{, d.h. } \frac{\omega^2}{c^2}-\vec{k}^2=0
\end{align}
%\paragraph{}
Welcher Energie und Impulstransport durch das Feld gegeben?\\
\begin{align}
\text{Poynting-Vektor: }\vec{S}&=\frac{4\pi}{c}\vec{E}\times\vec{B}\\
&=\frac{4\pi}{c}\vec{E}\times\left( \vec{n}\times\vec{E}\right)\\
&=\frac{4\pi}{c}\vec{n}\vec{E}^2=\frac{4\pi}{c}\vec{n}\vec{B}^2\\
\text{Energiedichte: }&\\
W&=\frac{1}{8\pi}\left( \vec{E}^2+\vec{B}^2 \right)\\
&=\frac{1}{4\pi}\vec{E}^2\\
\Rightarrow \vec{S}&=\vec{n}cW\\
\text{Impulsdichte: }&\\
\vec{P}&=\frac{1}{c^2}\vec{S}=\vec{n}\frac{W}{c}\\
\text{in der relativistschen Mechanik:}&\\
p&=\frac{1}{c}E \text{ wir für }m=0 \rightarrow \text{Photon}
\end{align}
\subsubsection{Polarisation}
Richtung der Felder: z.B. $\vec{E}$ ($\vec{B}=\vec{n}\times\vec{E}$)
\begin{align}
\vec{E}&=Re(\vec{E}_0 e^{i(\vec{k}\vec{x}-\omega t)}))\\
\vec{E}_0&=\text{konstant und komplex}
\end{align}
globale Phase: $\vec{E}_0^2=|\vec{E}_0^2|e^{-2i\alpha}$\\
Ansatz: $\vec{E}_0=\vec{b}e^{-i\alpha}$\\
$\vec{b}^2$ ist reell, aber Komponenten von $\vec{b}$ sind komplex.\\
Ansatz: $\vec{b}=\vec{b}_1+i\vec{b}_2$ ($\vec{b}_1$,$\vec{b}_2$ reell)\\
\begin{align}
\vec{b}^2&=\vec{b}_1^2+2i\vec{b}_1\vec{b}_2-\vec{b}_2^2\\
\Rightarrow &\vec{b}_1\vec{b}_2=0 \text{ , }\vec{b}_1\perp\vec{b}_2
\end{align}
Wähle Koordinatensystem mit Achesen parallel zu $\vec{n}$, $\vec{b}_1$,
$\vec{b}_2$
$\Rightarrow$ x-Achse $\parallel \vec{n}$\\
$\Rightarrow$ y-Achse $\parallel \vec{b}_1$\\
$\Rightarrow$ z-Achse $\parallel \vec{b}_2$\\
\begin{align}
\vec{E}=Re(\vec{E}_0 e^{i(\vec{k}\vec{x}-\omega t)})\\
E_x=0\\
E_y=b_1 \cos(\vec{k}\vec{x}-\omega t -\alpha)\\
E_z=\pm b_2 \sin(\vec{k}\vec{x}-\omega t -\alpha)\\
\end{align}
i.A. $b_1\neq b_2$
\begin{align}
\frac{E_y^2}{b_1^2}+\frac{E_z^2}{b_2^2}=1
\end{align}
Was eine Ellipsendarstellung ist.
%XXXX einfügen Grafik Y9
allgemein elliptische Polarisation
\begin{itemize}
  \item $|b_1|=|b_2|$\\ zirkulare Polarisation \\ 2 Umlaufrichtungen ($b_1=\pm
  b_2$) \\ $|\vec{E}|=$constant
  \item Sonderfall $b_1=0$ oder $b_2=0$ \\ lineare polarisation \\ (2
  Freiheitrsgrade)
\end{itemize}
Überlagerung zweier linear polarisierter Wellen $\rightarrow$ elliptische
Polarisation. \\ Basislösungen $\checkmark$\\
Fourier-Zerlegung ($\cos(\omega t)$, $e^{i\omega t}\rightarrow$vollständig)
\begin{itemize}
  \item für periodische Lösungen: $f(t+T)=f(t)$\\ 
  Fourierreihe: 
  \begin{itemize}
  \item Grundfrequenz $\omega_0=\frac{2\pi}{T}$, \begin{align}
  f(t)&=\sum_{n=-\infty}^\infty f_n e^{-i\omega_0nt}\\
  \text{mit } f_n&=\frac{1}{T}\int_{-\frac{T}{2}}^{+\frac{T}{2}}\mathrm{d}t f(t)
  e^{i\omega_0 t} \\
  \text{für} f^*(t)&=f(t) \Rightarrow f_n^*=f_{-n}
  \end{align}
\end{itemize}
\item allgemein: Fourier-Integrale \\(diskretes Spektrum von 'Grund'frequenzen
+ kontinuierliches Spektrum) \begin{align}
f(t)&=\frac{1}{2\pi}\int_{-\infty}^{+\infty} \mathrm{d}\omega f(\omega)e^{-i\omega t}
\text{mit} f(\omega)&=\int_{-\infty}^{+\infty} \mathrm{d}t f(t)e^{+i\omega t}\\
&\text{für reelle } f(t) \Rightarrow f^*(\omega)=f(-\omega)
\end{align}
\end{itemize}

\subparagraph{Wiederholung Ebene, monochromatische Welle}
\begin{itemize}
  \item Eben $\rightarrow \vec{k}$ ist Konstante
  \item monochromatisch $\rightarrow \omega$ ist Konstante
\end{itemize}
\begin{align}
\vec{A}&=Re(\vec{A}_0 e^{i\left(\vec{k}\vec{x}-\omega t\right)})\\
\dalembert \vec{A}&=\left( \laplace -\frac{1}{c^2}\frac{\partial^2}{\partial
t^2} \right)\vec{A}=0\\
\vec{k}&=\frac{\omega}{c}\vec{n} &\left( k_\mu k^\mu=0 \right)\\
\vec{n}\vec{E}&=0, &\vec{B}=\vec{n}\times\vec{E} \text(  (transversal))\\
\vec{S}&=cW\vec{n}
\end{align}
\begin{align}
\left( \text{QM} \vec{p}=\hbar\vec{k}\text{, }E=\hbar\omega \text{, wie Teichen
mit }m=0 \right)
\end{align}
Basislösungen\\
Fourierentwicklung\\
Ebene Wellen$=$Basislösungen\\
Orthogonalitätsrelation
\begin{equation}
\int_{-\infty}^{+\infty} \mathrm{d}t e^{i(\omega-\omega')t}=2\pi\delta(\omega-\omega')
\end{equation}
speziell: für periodische Lösungen gilt:\\
$\rightarrow$ disktretes Spektrum
\begin{equation}
f(t)=\sum_{-\infty}^{+\infty} f_n  e^{-i\omega_0 n t}
\end{equation}
Periode $T$: $\omega_0 T=2\pi$\\
für kontinuierliches Spektrum
\begin{equation}
f(t)=\frac{1}{2\pi}\int_{-\infty}^{+\infty} f(\omega)  e^{-i\omega_0 n t}
\end{equation} 
\underline{Intensität} ?\\
Energieinhalt $\propto E^2$, $B^2$, d.h. $f^2(t)$\\
Zeitmittelung\\
für diskretes Spektrum, $\omega_0 T=2\pi$
\begin{align}
f(t)&=\sum_{n=-\infty}^{+\infty} f_n  e^{-i\omega_0 n t} \text{  (}f \text{ ist
reel:}f_n^*=f_{-n}\\
f^2(t)&=\sum_{n,m=-\infty}^{+\infty} f_n f_m  e^{-i\omega_0 (n+m) t}\\
\overline{f^2(t)}&=\frac{1}{T} \int_{-\frac{T}{2}}^{\frac{T}{2}} f^2(t) \mathrm{d}t\\
&=\sum_{n,m=-\infty}^{+\infty} f_n f_m \underbrace{\frac{1}{T}
\int_{-\frac{T}{2}}^{\frac{T}{2}} e^{-i\omega_0 (n+m) t}\mathrm{d}t}_{=0 \text{ außer
für n+m=0}}\\
&=\sum_{n,m=-\infty}^{+\infty} f_n f_m \delta_{n,-m}\\
&=\sum_{n=-\infty}^{+\infty} f_n f_{-n}\\
&=2\sum_{n=1}^{+\infty} |f_n|^2 \text{ (} f_0=f_0^*=0\text{, da kein
Wellenterm)}
\end{align}
analog für kontinuierliche Spektren
\begin{align}
\overline{f^2(t)}&=\frac{2}{2\pi}\int_0^\infty \mathrm{d}\omega |f(\omega)|^2\\
\text{Bedingungen:}&\\
&f(\omega)\underset{\omega\rightarrow\infty}{\longrightarrow}0\\
&f(\omega)\underset{\omega\rightarrow0}{\longrightarrow} \text{ endlich}
\end{align}
Auch statische Felder lassen sich nach fourierkomponenten zerlegen:
Basistransformation $e^{i\vec{k}\vec{x}}$
\begin{align}
\phi(\vec{x})=\int_{-\infty}^{+\infty} \frac{\mathrm{d}^3k}{(2\pi)^3}
e^{i\vec{k}\vec{x}}\tilde{\phi}(\vec{k})\\
\phi(\vec{k})=\int_{-\infty}^{+\infty} \mathrm{d}^3x
e^{i\vec{k}\vec{x}}\phi(\vec{x})
\end{align}
\subsubsection{Feld einer Punktladung}
Poissongleichung: $\laplace\phi=-4\pi e \delta^{(3)}(\vec{x})$
\begin{align}
\laplace\phi=\frac{1}{(2\pi)^3}\int_{-\infty}^{+\infty} \mathrm{d}^3k e^{i\vec{k}\vec{x}}
\tilde{\phi}(\vec{k})(i\vec{k})^2\\
\Rightarrow \tilde{\laplace \phi}(\vec{k})=-\vec{k}^2 \tilde{\phi}(\vec{k})\\
\tilde{\laplace \phi}(\vec{k})=\int_{-\infty}^{+\infty}\mathrm{d}^3x e^{-i\vec{k}\vec{x}}
\underbrace{(\laplace\phi(\vec{k})}_{-4\pi e \delta^{(3)}(\vec{x})}\\
=-4\pi e
\end{align}
Poissongleichung für Fouriertransformation von $\phi$: $\tilde{\phi}$
\begin{align}
\vec{k}^2\tilde{\phi}(\vec{k})&=4\pi e\\
\Rightarrow \text{Lösung:} \tilde{\phi}&=\frac{4\pi e}{\vec{k}^2}\\
\phi(\vec{x})&\text{ inverse Fouriertransformation}\\
\phi(\vec{x})&=\frac{4\pi e}{(2\pi)^3}\int \mathrm{d}^3k
\frac{e^{i\vec{k}\vec{x}}}{\vec{k}^2}
\end{align}
\subparagraph{Fouriertransformierte des Elektrischen Feldes}
\begin{align}
\tilde{\vec{E}}(\vec{k})=-i\frac{4\pi e}{\vec{k}^2}\vec{k}
\end{align}
\subsection{Einführung in die Quantenfeldtheorie}
\subsubsection{Eigenschwingungen des Feldes}
Wenn man die Eigenschwingungen des harmonischen Oszillators aus der
Quantenmechanik als Grundschwingungen des Feldes postuliert gelangt man zur
Quantenfeldtherie. \\
Motivation: Quantenmechanik, d.h. Quantenfeldtheorie:\\
Hamiltondichte $\rightarrow$ Verallgemeinerter Hamiltonoperator\\
Coulombeichung $\phi=0$, $\vec{\nabla}\vec{A}=0$ (Vakuum)\\
$\vec{A}$ in endlichen Volumina $\rightarrow$ Fourierreihen\\
\begin{align}
V&=L_xL_yL_z \text{(Quader)}\\
\rightarrow \text{Ansatz: }
\vec{A}&=\sum_{\vec{k}}\vec{A}_{\vec{k}} e^{i \vec{k} \vec{x}}\\
&\text{mit } k_x=\frac{2\pi}{L_x}n_x \text{, } k_y=\frac{2\pi}{L_y}n_y \text{,
 }
k_z=\frac{2\pi}{L_z}n_z \text{, } n_{x,y,z}\in\mathbb{N}
\end{align}
$\vec{A}$ ist reel: $\vec{A}_{\vec{k}}=\vec{A}^*_{-\vec{k}}$\\
Coulombeichung: $\vec{k}\vec{A}_{\vec{k}}=0$\\
Wellengleichung: $\ddot{\vec{A}}_{\vec{k}}+c^2\vec{k}^2\vec{A}_{\vec{k}}=0$\\
Feldkonfiguration: $\vec{A}(\vec{x},t)\longrightarrow\vec{A}_{\vec{k}}(t)$\\
Berechne Energie der Felder
\begin{align}
\epsilon&=\frac{1}{8\pi}\int \mathrm{d}^3x\left(\vec{E}^2+\vec{B}^2\right)\\
\vec{E}&=-\frac{1}{c}\dot{\vec{A}}=-\frac{1}{c}
\sum_{\vec{k}}\dot{\vec{A}}_{\vec{k}}e^{i\vec{k}\vec{x}}\\
\vec{B}&=\vec{\nabla}\times\vec{A}=i
\sum_{\vec{k}}\vec{k}\times\vec{A}_{\vec{k}}e^{i\vec{k}\vec{x}}\\
\epsilon&=\frac{1}{8\pi c^2}\left( \int \mathrm{d}^3x \left(
\sum_{\vec{k}}\dot{\vec{A}}_{\vec{k}}e^{i\vec{k}\vec{x}} \right) \left(
\sum_{\vec{k}}\dot{\vec{A}}_{\vec{k}}e^{i\vec{k}\vec{x}} \right)+\vec{B}^2
\right)\\
&=\frac{1}{8\pi c^2}\sum_{\vec{k}\vec{k}'}\dot{\vec{A}}_{\vec{k}}
\dot{\vec{A}}_{\vec{k}'} \int_V e^{i(\vec{k}+\vec{k}')\vec{x}}+\ldots\\
\text{enthält } \int_0^{L_x}\mathrm{d}x e^{i\frac{2\pi}{L_x}n_xx} &=\left\lbrace %XXXX Fächer 
\begin{array}{l l }
& L_x \text{ für }n_x=0 \\ 
& 0 \text{ sonst}
\end{array}\right.
%{L_x \text{ für }n_x=0}{0 \text{ sonst}}
\\
\Rightarrow V\delta_{n_x+n_x',0}\delta_{n_y+n_y',0}\delta_{n_z+n_z',0}
\Rightarrow \epsilon &=\frac{V}{8\pi c^2}\sum_{\vec{k}}\left(\dot{\vec{A}}_{\vec{k}}
\dot{\vec{A}}_{-\vec{k}} + \left( \vec{k}\times{\vec{A}}_{\vec{k}}\right)
\left( \vec{k}\times{\vec{A}}_{-\vec{k}}\right) \right)\\
&=\frac{V}{8\pi c^2}\sum_{\vec{k}}|\dot{\vec{A}}_{\vec{k}}|^2+c^2
\vec{k}^2|{\vec{A}}_{\vec{k}}|^2\\
\end{align}

\subsection{Ausstrahlung Elektromagnetischer Wellen}
(Mannteufel Dozierender)\\
Feldgleichungen in Gegenwart von Strömen ($\vec{j}\neq0$) und Ladungen
($\rho\neq0$)\\
\begin{align}
\partial_\mu F^{\mu\nu}=\frac{4\pi}{c}j^\nu\\
F^{\mu\nu}=\partial^\mu A^\nu-\partial^\nu A^\mu\\
\partial_\mu A^\mu=0 \text{ (Lorentzeichung)}\\
\underbrace{\partial_\mu \partial^\mu}_{\dalembert} A^\nu=\frac{4\pi}{c}j^\nu
\text{ (Wellengleichung)}\\
\end{align}
Der Einfachheit halber konzentrieren wir uns hier auf das skalare Potential.
\begin{align}
\laplace\phi-\frac{1}{c^2}\ddot{\phi}=-4\pi\rho\\
\rho=\underline{\mathrm{d}e}(t')\delta(\vec{x}') \text{ Ladung in einem infinietisemalen
Volumenelement}\\
\frac{1}{R^2}\frac{\partial}{\partial R} \left( R^2 \frac{\partial\phi}{\partial
R}\right)-\frac{1}{c}\frac{\partial^2}{\partial t^2}\phi=0 \text{ für }R\neq0\\
R=\vec{x}\\
\chi=R\phi \text{  }\Rightarrow \phi=\frac{\chi}{R}\\
\frac{\partial^2\chi}{\partial R^2}-\frac{1}{c^2}\frac{\partial^2\chi}{\partial
t^2}=0\\
\underbrace{\chi=\chi_1(t-\frac{R}{c})}_{\text{retardiert}} &
\underbrace{\chi=\chi_2(t+\frac{R}{c})}_{\text{avanciert}}
\end{align}
Physikalisch relevant sind nur die retardierten Lösungen
\begin{align}
\phi=\frac{\chi_1(t-\frac{R}{c})}{R}
\end{align}
Verlange für $R\rightarrow0$ in das Coulombpotential übergehen muss\\
\begin{align}
\Rightarrow \phi=\frac{\mathrm{d}e\left(t-\frac{R}{c}\right)}{R}
\end{align}
dies gilt für eine infinetisemale Ladung. Für beliebige Ladugnsverteilungen gilt
die Superposition der $\mathrm{d}e$-Lösungen.
\begin{align}
\phi(\vec{x},t)=\int \mathrm{d}^3x'
\frac{1}{R}\rho\left(\vec{x}',t-\frac{R}{c}\right)+\phi_0\\
\vec{A}(\vec{x},t)=\frac{1}{c}\int
\mathrm{d}^3x'\frac{1}{R}\vec{j}\left(\vec{x}',t-\frac{R}{c}\right)+\vec{A}_0\\
\vec{R}=\vec{x}-\vec{x}'
\end{align}
die sind die Retardierten Potentiale. Hier sind $\phi_0$ und $\vec{A}_0$ die
Lösungen der homogenen Differentialgleichung. Auch zeigt sich die endliche
Ausbreitungsgeschwindigkeit von Signalen als zentraler Effekt der speziellen
Relativitätstheorie in den Argumenten $t-\frac{R}{c}$ als eine
Signalverzögerung.
\subsubsection{Spezialfall: Lienard-Wiechert-Potentiale}
Quelle:\underline{Punktladung} an den Koordinaten $\vec{x}'=\vec{x}_0(t')$ \\
Aufpunkt: $\vec{x}$, Zeit $t$\\
Abstand zur Quelle $\vec{R}(t')=\vec{x}-\vec{x}_0(t')$\\
Signalaufzeit: $t-t'=\frac{R(t')}{c}$ mit $R(t')=|\vec{R}(t')|$
%XXXX Einfügen von Grafik Z1
Wähle Koordinatensystem in dem die Ladung zur Zeit $t'$ ruht
\begin{align}
\phi&=\frac{e}{R(t')}, & \vec{A}=0 \text{ Coulombpotential}\\
&=\frac{e}{c(t-t')}
\end{align}
Versuche nun diese Gleichung in eine Form zu bringen, bei der klar erkennbar
ist, das sie Lorentzinvariant ist.\\
In einem beliebigem Koordinatensystem: suche $A^\mu$ so, dass für $v=0$ gilt
$A^0=\phi_{\text{Coulomb}}$\\
\begin{align}
R^\mu=(c(t-t'),\vec{x}-\vec{x}') \text{ wofür gelten muss }R^\mu R_\mu=0\\
u^\mu \text{ Vierergeschwindigkeit}\\
u^\mu=\frac{1}{\sqrt{1-\beta^2}}(c,\vec{v}), \text{ , }
\beta=\frac{\vec{v}}{c}\\
\Rightarrow A^\mu=e\frac{u^\mu}{R_\nu u^\nu} \text{ Lösung im allgemeinen
Koordiantensystem}\\
R_\nu u^\mu=\frac{1}{\sqrt{1-\beta^2}}(R-\vec{R}\vec{\beta})
\end{align}
Komponentenweise
\begin{align}
\phi=\frac{e}{(R-\vec{R}\vec{\beta})}|_{t'=t-\frac{R}{c}}\\
\vec{A}=\frac{e\vec{\beta}}{(R-\vec{R}\vec{\beta})}|_{t'=t-\frac{R}{c}}
\end{align}
\subparagraph{Alternative Herleitung}
\begin{align}
\phi(\vec{x},t)=\int \mathrm{d}^3x' \frac{1}{R}
\underbrace{\rho(\vec{x}',t-\frac{R}{c})}_{=e
\delta(\vec{x}'-\vec{x}_0(t')\text{ , }t'=t-\frac{R}{c}} \text{ ,
}R=|\vec{x}-\vec{x}'|
\end{align}
Problem: Der $x'$-Integrand ist nicht direkt WS-Abhängig via $t'$
Trick: 
\begin{align}
\phi=\int \mathrm{d}^3x' \mathrm{d}t' \delta\left(t-t'-\frac{|\vec{x}-\vec{x}'|}{c}\right)
\frac{\rho(\vec{x}',t')}{|\vec{x}-\vec{x}'|}\\
\end{align}
Nun lassen sich $\vec{x}'$ und $t'$ unabhängigvoeinander Integrieren
\begin{align}
\phi=\int \mathrm{d}t' \delta\left(t-t'-\frac{|\vec{x}-\vec{x}_0(t')|}{c}\right)
\frac{e}{|\vec{x}-\vec{x}_0(t')|}\\
\end{align}
Nun muss $\delta$ umgeschrieben werden
\begin{align}
\delta(g(x))=\sum_{x_0=\text{Nulstellen von
}g(x)}\frac{\delta(x-x_0)}{|g'(x_0)|}\\
\hat{t}=t'+\frac{|\vec{x}-\vec{x}_0(t')|}{c}\\
\Rightarrow \delta(t-\hat{t})\\
\frac{\mathrm{d}\hat{t}}{\mathrm{d}t'}=\ldots=-\frac{\vec{\beta}\vec{R}}{R}\\
\Rightarrow \phi=\frac{e}{R-\vec{R}\vec{\beta}}
\end{align}
Für die resultierenden Felder ergibt sich nun:
\begin{align}
\vec{E}=-\frac{1}{c}\dot{\vec{A}}-\vec{\nabla}\phi\\
\vec{B}=\vec{\nabla}\times\vec{A}
\end{align}
Problem: Man benötigt Ableitungen nach $\vec{x}$ und $t$, aber $\phi$ und
$\vec{A}$ sind über $t'$ definiert.\\
Abhängigkeiten:
\begin{align}
\phi=\phi(\vec{R},\vec{\beta})\\
\vec{A}=\vec{A}(\vec{R},\vec{\beta})\\
\vec{R}=\vec{R}(\vec{x},t')\\
\vec{\beta}=\vec{\beta}(t')\\
t'=t'(\vec{x},t)
\end{align}
Damit folgt:
\begin{align}
\frac{\partial \phi(\vec{x},t)}{\partial x_i}=\frac{\partial \phi}{\partial
R_j}\left( \frac{\partial R_j}{\partial x_i} +\frac{\partial R}{\partial
t'} \frac{\partial t'}{\partial x_i}\right)+\frac{\partial \phi}{\partial
\beta_j}\frac{\partial \beta_j}{\partial t'}\frac{\partial t'}{\partial x_i}\\
\frac{\partial \phi}{\partial
R_j}=-\frac{e}{(R-\vec{R}\vec{\beta})^2}(\frac{R_j}{R}-\beta_j)\\
\frac{\partial \phi}{\partial
R_j}=-\frac{e}{(R-\vec{R}\vec{\beta})^2}(-R_j)\\
\frac{\partial R_j}{\partial x_i}=\delta_{ij}\\
\frac{\partial R_j}{\partial t'}=-c\beta_j\\
\end{align}
Hilfsgrößen
\begin{align}
\frac{\partial t'}{\partial t} \text{ , } \frac{\partial t'}{\partial
x_i}\text{: via }R^2\\
\vec{R}=\vec{x}-\vec{x}_0(t'(x,t))\\
|\vec{R}|=c*\left(t-t'(x,t)\right)\\
\end{align}
Für $\frac{\partial t'}{\partial t}$
\begin{align}
\frac{\partial R^2}{\partial t}=2\vec{R}\frac{\partial \vec{R}}{\partial
t}=2\vec{R}\left(-c\vec{\beta}\frac{\partial t'}{\partial t}\right)\\
\frac{\partial R^2}{\partial t}=2R\frac{\partial R}{\partial
t}=2\vec{R}c\left(1-\frac{\partial t'}{\partial t}\right)\\
\Rightarrow \frac{\partial t'}{\partial t}\left( R-\vec{\beta}\vec{R}
\right)=R\\
\Rightarrow \frac{\partial t'}{\partial t}=\frac{R}{R-\vec{\beta}\vec{R}}\\
\end{align}

Für $\frac{\partial t'}{\partial x_i}$
\begin{align}
\frac{\partial R^2}{\partial
x_i}=2\left(R_i-\vec{R}c\vec{\beta}\frac{\partial t'}{\partial x_i}\right)\\
\frac{\partial R^2}{\partial x_i}=2R\left(-c\frac{\partial t'}{\partial
x_i}\right)\\
\Rightarrow \frac{\partial t'}{\partial x_j}\left(-\vec{\beta}\vec{R}c
+c\right)=-R_i\\
\Rightarrow \frac{\partial t'}{\partial
x_i}=-\frac{R_i}{c(R-\vec{\beta}\vec{R})}\\
\frac{\partial \beta_j}{\partial t'}=\dot{\beta}_j\\
\end{align}
Als Endergebnis erhält man nun:
\begin{align}
\frac{\partial\phi}{\partial x_i}=-\frac{e}{(R-\vec{\beta}\vec{R})^2}\left[
\left( \frac{R_j}{R}-\beta_j \right)\left( \delta_{ij}-c\beta_j
\frac{(-R_i)}{(R-\vec{\beta}\vec{R})c} \right)+(R_j)\dot{\beta}_j\left(
\frac{-R_i}{(R-\vec{\beta}\vec{R})c} \right) \right]\\
=-\frac{e}{(R-\vec{\beta}\vec{R})^3}\left(
R_i(1-\beta^2+\vec{R}\dot{\vec{\beta}/c})-\beta_i(R-\vec{\beta}\vec{R})
\right)\\
\ldots\\
\vec{E}=e(1-\beta^2)\underbrace{\frac{\vec{R}-\vec{\beta}R}{(R-\vec{\beta}
\vec{R})^3}}_{\text{Geschwindigkeitsabhängig}}+
e\underbrace{\frac{\vec{R}[(R-\vec{\beta}\vec{R})\dot{\vec{\beta}}]}{c
(R-\vec{\beta}\vec{R})^3}}_{\text{Beschleunigungsabhängig}}\\
\Rightarrow \vec{B}=\frac{1}{R}\vec{R}\times\vec{E}, \vec{B}\perp\vec{E}
\end{align}
Anmerkung:\\
erster Term (gleichförmig bewegte Punktladung) erhält man auch über
Coulombfeld im Ruhesystem und Lorentztransformationen.
\subsubsection{Wiederholung Ausstrahlung von elektromagnetischen Wellen}
\begin{align}
\dalembert A^\mu=\frac{4\pi}{c}j^\mu
\end{align}
Zeit und Raumkoordinaten $t\pm\frac{R}{c}$\\
für Coulombfeld $\phi\propto\frac{\chi(t-\frac{R}{c})}{R}$\\
allgmeine Lösung: retariderte Potentiale:\\
\begin{align}
\phi(\vec{x},t)=\int \mathrm{d}^3x' \frac{1}{R}\phi(\vec{x}',t-\frac{R}{c})+\phi_0\\
\vec{A}(\vec{x},t)=\frac{1}{c}\int \mathrm{d}^3x'
\frac{1}{R}\vec{j}(\vec{x}',t-\frac{R}{c})+\vec{A}_0\\
\text{mit Signalgeschwindigeit }t=\frac{|\vec{x}-\vec{x}'|}{c} \text{ und }
R=|\vec{x}-\vec{x}'|
\end{align}
Spezialfall der bewegten Punktladungen:\\
Lienard-Wiechert:
\begin{align}
\vec{R}=\vec{x}-\vec{x}' \text{ , } R=|\vec{R}|\\
\phi(\vec{x},t)=\frac{e}{R-\frac{\vec{v}\vec{R}}{c}}|_{t'=t-\frac{R}{c}}\\
\vec{A}(\vec{x},t)=\frac{e\vec{v}}{c(R-\frac{\vec{v}\vec{R}}{c})}|_{t'=t-\frac{R}{c}}
\end{align}
%XXXX Einfügen von Grafik Z2
\subsubsection{Strahlungsfeld in großem Abstand}
Ladungen undStröme in begrentztem Raumgebiet\\
%XXXX Einfügen von Grafik Z3
Ladungselement $\mathrm{d}e=\rho \mathrm{d}^3x'$\\ am Ort $\vec{x}'$\\
Abstand $R=|\vec{x}-\vec{x}'|$ sei groß ($R>>|\vec{x}'|$,
$|\vec{x}'|\underset{<}{\approx}a$)
Entwicklung nach Potenzen von $\frac{|\vec{x}'|}{R}$
\begin{align}
R=|\vec{R}|=|\vec{x}-\vec{x}'|\approx
|\vec{x}|-\frac{\vec{x}\vec{x}'}{|\vec{x}|}+O({x_i'}^2)\\
R=r-\vec{v}\vec{x}' \text{ , }\vec{n}=\frac{\vec{x}}{|\vec{x}|}
\end{align}
Ersetze $R\rightarrow r$ im Nenner\\
$n$ in Zeitargument von $\rho$,$\vec{j}$ ist nicht "`kontrolliert"'
\begin{align}
\Rightarrow \phi(\vec{x},t)=\frac{1}{r}\int \mathrm{d}^3x'
\rho(\vec{x}',t-\frac{r-\vec{n}\vec{x}'}{c})\\
\vec{A}(\vec{x},t)=\frac{1}{rc}\int \mathrm{d}^3x'
\vec{j}(\vec{x}',t-\frac{r}{c}+\frac{\vec{n}\vec{x}'}{c})
\end{align}
Frage: Zeitabhängigkeit von $\rho$\\
\begin{align}
\rho(\vec{x}',t-\frac{r}{c}+\frac{\vec{n}\vec{x}'}{c})=
\rho(\vec{x}',t-\frac{r}{c})+\underbrace{\frac{\partial \rho}{\partial
t'}|_{t'=t-\frac{r}{c}}}_{\approx\frac{\Delta
\rho}{\Delta t}\propto \frac{\rho}{T}}\frac{\vec{n}\vec{x}'}{c}+\ldots
\end{align}
Definition $T$: Zeit in der sich die Ladungsverteilung merklich ändert
(charakteristische Zeitskala)
\begin{align}
\rho(\vec{x}',t-\frac{r}{c})\approx \rho(\vec{x}',t-\frac{r}{c})\left(1+
\underbrace{\frac{\vec{n}\vec{x}'}{c T}}_{\underset{<}{\approx}\frac{a}{cT}}+\ldots
\right)
\end{align}
Zusätzliche Forderung: 
\begin{align}
a \ll cT \\
\frac{a}{T} \ll c
\end{align}
Fourierzerlegung $\rightarrow$ spektrale Vertielung des abgestrahlten Feldes\\
Fouriertransformation von $\vec{j}$: $\rightarrow
\vec{j}(\vec{x},\omega)=\int \mathrm{d}t e^{i\omega t} \vec{j}(\vec{x},t)$
\begin{align}
\vec{j}(\vec{x},t-\frac{r}{c}+\frac{\vec{n}\vec{x}'}{c})=\int
\frac{\mathrm{d}\omega}{2\pi} e^{-i\omega (t-\frac{r}{c}+\frac{\vec{n}\vec{x}'}{c})}
\vec{j}(\vec{x}',\omega)\\
\text{daraus} \vec{A}(\vec{x},\omega)=\int \mathrm{d}t e^{i\omega t} \vec{A}(\vec{x},t)\\
=\int \mathrm{d}t e^{i\omega t} \frac{1}{cr} \int \mathrm{d}^3x'\int \frac{\mathrm{d}\omega'}{2\pi} 
e^{-i\omega (t-\frac{r}{c}+\frac{\vec{n}\vec{x}'}{c})}
\vec{j}(\vec{x}',\omega')\\
\text{ mit } \int \mathrm{d}t e^{it(\omega-\omega')}=2\pi \delta(\omega-\omega')\\
\vec{A}(\vec{x},\omega)=\frac{1}{cr} \int \mathrm{d}^3x' e^{i\omega
(\frac{r}{c}-\frac{\vec{n}\vec{x}'}{c})} \vec{j}(\vec{x}',\omega')\\
=\frac{e^{ikr}}{cr} \underbrace{\int \mathrm{d}^3x'
e^{-i\vec{k}\vec{x}'}}_{\text{Foriertransformation bezüglich }\vec{x}'}\\
\end{align}
Diese Gleichung stellt eine Kugelwelle dar.\\
\begin{align}
\frac{\partial}{\partial r}\left( \frac{e^{ikr}}{r}
\right)&=ik\frac{e^{ikr}}{r}-\underbrace{\frac{e^{ikr}}{r^2}}_{\text{vernachlässigbar}}\\
&=ik\frac{e^{ikr}}{r}\left(1+\underbrace{i\frac{1}{kr}}_{ \ll 1}\right)
\end{align}
Der Term mit $\frac{1}{kr}$ wird vernachlässigt weil wir annehmen, dass $kr>>1$
beziehungsweise $\frac{\omega}{c}r>>1$, was wir als zusätzliche Nährung
annehmen.\\
Wellengleichung: $e^{i\left(kx-\omega t\right)}$
Zu der Zeitskala $T$ gehört zwangsläufig eine Längenskala ($cT\approx\lambda$), die
eine typische Wellenlänge festlegt, die zu dem von uns betrachtetem System
gehört. Mit $\lambda=\frac{2\pi c}{\omega}$ ergibt sich die Annahme
$\frac{r}{\lambda}>>1$ was aussagt, dass die typische Wellenlänge klein ist
bezüglich des Abstandes des Beobachters von dem Ladungssystem ($\lambda \ll r$),
diesen Bereich nennt man Wellenzone.\\
In großem Abstand und für $\lambda \ll r$ (d.h. in der Wellenzone) sind $\vec{E}$
und $\vec{B}$ näherungsweise ebene Wellen.
\begin{itemize}
  \item characteristische Eigenschaft: \\ \begin{equation}
  \vec{E},\vec{B}\propto \frac{1}{r}
  \end{equation}
\end{itemize}
\subparagraph{Beispiel}
Strahlungsfeld eine bewegten Punktladung
\begin{align}
\vec{A}=\frac{e}{cr}*\frac{\vec{v}(t')}{1-\frac{\vec{n}\vec{v}(t')}{c}}\\
\text{mit }R=r-\vec{n}\vec{x}' &
t'=\frac{\vec{x}_0(t')\vec{n}}{c}=t-\frac{r}{c}\\
\text{Punktladung: } \vec{j}=e\vec{v}(t')\delta(\vec{x}'-\vec{x}_0(t'))\\
\vec{j}(\omega)=\int \mathrm{d}t' \vec{j} e^{i\omega t'}\\
\Rightarrow \vec{A}(\omega)=e*\frac{e^{ikr}}{cr}\int \mathrm{d}t' \vec{v}(t')e^{i\left(
\omega t'-\vec{k}\vec{x}_0(t') \right)}\text{ ,
}\vec{v}=\frac{\mathrm{d}\vec{x}_0}{\mathrm{d}t'}\\
=e*\frac{e^{ikr}}{cr}\int_{\text{Bahnkurve}} \mathrm{d}\vec{x}_0 e^{i\left(
\omega t'(\vec{x}_0)-\vec{k}\vec{x}_0 \right)}
\end{align}
\subsubsection{Intensität des Strahlungsfeldes}
Die Intensität ist die Energie pro Zeit pro Fläche, $r^2\mathrm{d}\Omega$\\
Pointing-Vektor 
\begin{align}
\vec{S}=\frac{c}{4\pi}B^2\vec{n}\\
\mathrm{d}I=\frac{c}{4\pi}\overline{B^2}r^2\mathrm{d}\Omega\\
\Rightarrow \frac{\mathrm{d}I}{\mathrm{d}\Omega}= \text{unabhängig von }r\\
\Rightarrow \text{spektrale Intensitätsvertielung}\\
\mathrm{d}I(\vec{n},\omega)=\frac{c}{2\pi}|\vec{B}(\omega)|^2r^2\mathrm{d}\Omega
\frac{\mathrm{d}\omega}{2\pi}
\end{align} 

\subsubsection{Wiederholung Strahlung}
Wellengleichung\\
Retardierung
\begin{align}
\phi(\vec{x},t)\longleftarrow&\rho(\vec{x}',t-\frac{R}{c})
&\text{mit }R=|\vec{x}-\vec{x}'| 
\end{align}
Bewegte Punktladung$\rightarrow$Lienard-Wiechert für Strahlungsfeld in großem
Abstand.\\
Charakterisierung der Quelle\\
\begin{itemize}
  \item typische Zeitskala $T$ und typische Wellenlänge $\lambda=cT$
  \item typische Längenskala $a$ ($\vec{x}'\vec{n}\underset{\sim}{<}a$)
\end{itemize}
retardierte Potnetiale
\begin{align}
\vec{A}(\vec{x},t)=\underset{\text{großer Abstand } r\gg a}{
\frac{1}{cr}\int\mathrm{d}^3x'
\vec{j}(\vec{x}',t-\frac{r}{c}+\frac{\vec{n}\vec{x}'}{c})} 
\end{align}
Fourierzerlegung
$\vec{A}(\vec{x}',\omega)=\frac{e^{ikr}}{cr}\int\mathrm{d}^3x'e^{i\vec{k}\vec{x}}\vec{j}(\vec{x}',\omega)$\\
Systematische Entwicklung von $e^{i\vec{k}\vec{x}}$\\
Forderung: \begin{align}
\vec{k}\vec{x}'\ll1\\
\frac{\omega}{c}\vec{n}\vec{x}'\underset{\sim}{<}\frac{\omega}{c}a=\frac{a}{\lambda}\ll1\\
\end{align}
Dies bedeutet, dass die Wellenlänge gegen $a$ groß ist.\\
typische Zeit:
\begin{align}
T\simeq \frac{a}{V}\text{ , }\ \ \ &\lambda\simeq c\frac{a}{V}\\
&\frac{a}{\lambda}\simeq\frac{v}{c}\ll1
\end{align}
Darraus folgern wir, dass für die Geschwindigleit gelten muss $v\ll c$, das
System darf also nicht relativistisch sein.\\
Im flgenden nehmen wir an $r\gg\lambda\gg a$, was als Wellenzone oder Fernzone
bezeichnet wird. Außerdem gibt es noch $\lambda\gg r\gg a$ was zur Statik führt,
da auch die Retardierung vernachlässigbar wird. Als letztes gibt es noch
$\lambda \gg a \simeq r$ was die komplexer zu beschreibende Nahzone
charakterisisert.\\
\paragraph{erster Term der Nährung}
\begin{align}
\vec{A}(\vec{x},t)=&\frac{1}{cr}\int\mathrm{d}^3x' \vec{j}(\vec{x}',t-\frac{r}{c})\\
&r=|\vec{x}| \text{ unabhängig von }\vec{x}'
\end{align}
z.B. bewegte Punktladungen $\vec{j}=\sum e\vec{v}
\delta(\vec{x'}-\vec{x}_0(t'))$
\begin{align}
\vec{A}(\vec{x},t)&=\frac{1}{cr}\sum e\vec{v}(t')\\
&=\frac{1}{cr}\frac{\partial}{\partial t'}\underbrace{\sum
e\vec{x}(t')}_{\vec{D} \text{ Dipolmoment}}\\
\Rightarrow \vec{A}(\vec{x},t)&= \frac{1}{rc}\dot{\vec{D}}\\
\rightarrow \vec{B}&=\frac{1}{c^2r}\ddot{\vec{D}}\times\vec{n}\\
\vec{B}&=\frac{1}{c^2r}\left(\ddot{\vec{D}}\times\vec{n}\right)\times\vec{n}\\
\end{align}
Diese Gleichungen beschreiben die sogenannte Dipolstrahlung.
\subparagraph{Bemerkungen}
\begin{itemize}
  \item $\vec{D}\propto\vec{x}' \Rightarrow
  \vec{E}\text{,}\vec{B}\propto\ddot{\vec{x}}',$\\ nur beschleunigte Ladungen
  strahlen. Gleichförmig bewegte Ladungen strahlen nicht. Dies lässt sich
  ebenfalls einfach aus dem Relativitätsprinzip herleiten.
  \item für ein System von Ladungen mit $\frac{e}{m}$ konstant gilt:\\
  $\vec{D}=\sum e_i\vec{x}_i'=\frac{e_j}{m_j}\sum
  m_i\vec{x}_i'=\frac{e}{m}\vec{X}M$\\ falls ein solches System nach
  außen kräftefrei ist kann es ebenfalls nicht strahlen, auch wenn die Teilchen
  sich gegenseitig beschleunigen.
\end{itemize}

\subparagraph{Intensität}
\begin{align}
\mathrm{d}I&=\frac{1}{4\pi c^3}\big( \underbrace{\ddot{\vec{D}}\times\vec{n}
}_{\angle(\ddot{\vec{D}},\vec{n})=\theta}\big)^2\mathrm{d}\Omega\\
&=\frac{1}{4\pi c^3}|\ddot{\vec{D}}|^2
\underbrace{\sin^2(\theta)}_{=0 \text{ f\"ur }\theta=0}\mathrm{d}\Omega\\
\end{align}
Also gibt es keine Abstrahlung in Richtung $\ddot{\vec{D}}$ und die
Abstrahlungsleistung ist maximal in Richtung $\perp \ddot{\vec{D}}$\\
Für die Gesamtintensität ergibt sich:
\begin{align}
I&=\intd I=\frac{1}{4\pi
c^3}\underbrace{|\ddot{\vec{D}}|^2}_{\substack{\text{falls } \ddot{\vec{D}}\\
\text{ unabhängig von }\Theta}} \underbrace{\int\mathrm{d} \Omega
\sin(\theta)}_{=8\pi/3}\\
&=\frac{2}{3c^3}|\ddot{\vec{D}}|^2
\end{align}
\subparagraph{Spektrum}
\begin{align}
\mathrm{d}I_\omega=\frac{4}{3c^3}|\ddot{\vec{D}}_\omega|^2\frac{\mathrm{d}\omega}{2\pi}
\end{align}
Hertzscher Dipol
\begin{align}
\vec{D}(t)=\vec{D}_0 e^{i\omega t}\\
\Rightarrow \ddot{\vec{D}}(t)=-\omega^2\vec{D}_0 e^{i\omega t}\\
\mathrm{d}I_\omega=\frac{4\omega^4}{3c^3}|\ddot{\vec{D}}_0|^2\frac{\mathrm{d}\omega}{2\pi}
\end{align}
Eigenschaften des Hertzschen Dipols:
\begin{itemize}
  \item $\mathrm{d}I\propto \omega^4$
  \item $\mathrm{d}I\propto \sin^2\theta$
\end{itemize}
\paragraph{Höhere Terme der Entwicklung}
\begin{align}
\vec{A}(\vec{x},t)&=\frac{1}{cr}\int\mathrm{d}^3x'
\vec{j}(\vec{x}',t-\frac{r}{c}+\frac{\vec{n}\vec{x}'}{c})\\
&= \frac{1}{cr}\int\mathrm{d}^3x'
\vec{j}(\vec{x}',t-\frac{r}{c})+\frac{1}{rc^2}\frac{\partial}{\partial
t'}\int\mathrm{d}^3x' \vec{n}\vec{x}'\vec{j}(\vec{x}',\underbrace{t-\frac{r}{c}}_{=t'})
\end{align}
für bewegte Punktladungen $\vec{j}=\sum e\vec{v} \delta(\vec{x'}-\vec{x}_0(t'))$
ergibt sich eine weiterer Beitrag:
\begin{align}
&\frac{1}{c^2r}\frac{\partial}{\partial t'}\sum
e\vec{v}(\vec{n}\underbrace{\vec{x}'}_{\text{Bahnkurven }\vec{x}_0(t')})\\
=&\frac{1}{c^2r}\sum e \left[\frac{1}{2}\vec{v}(\vec{n}\vec{x}')+
\frac{1}{2}\frac{\partial}{\partial t'} \left( \vec{x}' (\vec{n}\vec{x}') -
\frac{1}{2}\vec{x}'(\vec{n}\vec{v})\right)\right]\\
=&\frac{1}{c^2r}\sum e \Big[\frac{1}{2} \frac{\partial}{\partial t'} \left(
\vec{x}' (\vec{n}\vec{x}') \right)
+\underbrace{\frac{1}{2}(\vec{x}'\times\vec{v})}_{\vec{m}=\frac{1}{2c}\sum
e\vec{x}'\times\vec{v}}\times\vec{n}\Big]
\end{align}
Damit folgt:
\begin{align}
\vec{A}&=\frac{1}{cr}\dot{\vec{D}}+\frac{1}{cr}\dot{\vec{m}}\times\vec{n}+\frac{1}{2c^2r}\frac{\partial^2}{\partial^2}
\sum e\vec{x}'(\vec{n}\vec{x}')+\ldots\\
\Rightarrow
||\vec{A}&=\underset{\text{el. Dipol}}{\frac{1}{cr}\dot{\vec{D}}}+
\underset{\text{magn. Dipol}}{\frac{1}{cr}\dot{\vec{m}}\times\vec{n}}+
\underset{\text{Quadrupol}}{\frac{1}{6c^2r}\ddot{\vec{Q}}}+\ldots
\end{align}
Auffallend ist, dass alle Terme nur eine explizite Abhängigkeit von
$\frac{1}{r}$ enthalten. Es stellt sich also die Frage ob diese Entwicklung
tatsächlich immer kleiner werdende Korrekturen anbringt. 
\begin{align}
\dot{\vec{D}}&\propto \sum e\vec{x}'\propto a\\
\dot{\vec{m}}&\propto \sum e\vec{x}'\times\vec{v}\frac{1}{c} \propto
a\frac{v}{c} \ll a\\
\ddot{\vec{Q}}&\propto \frac{\partial}{\partial t}\sum
e\vec{x}'(\vec{n}\vec{x}')\propto a\frac{v}{x} \ll a
\end{align}
Somit ist die Entwicklung gerechtfertigt, da eine implizite Unterdrückung in den
Größen $\dot{\vec{D}},\dot{\vec{m}},\ddot{\vec{Q}}$ enthalten ist.

\paragraph{Beispiel}~
\\
Streuung an einer freien Ladung (z.b. Elektron, Ladung e, Masse m) einlaufende
Ebene monochromatische Welle.\\
%XXXX Einfügen Grafik z4
Frage effektiver Streuquerschnitt:
\begin{align}
\mathrm{d}\sigma=\frac{\overline{\mathrm{d}I}}{\overline{S}} \text{  zeitlich
gemittelt}
\end{align}
mit $\mathrm{d}I=$abgestrahlte Energie pro Raumwinkel in gegebener Richtung und
$\mathrm{d}\sigma=$abgestrahlte Energie pro Raumwinkel in gegebener Richtung pro
einfallende Energiefluß.\\
einlaufende Strahlung
\begin{align}
\vec{E}=\vec{E}_0 \cos(\vec{r}\vec{k}-\omega t -\alpha)\ \Rightarrow\ 
\vec{B}\ldots
\end{align}
Damit folgt für die Lorentzkraft:
\begin{align}
\left| \frac{e}{c}\vec{v}\times\vec{B}\right|\ll e\vec{E}
\end{align}
Das bedeutet für schwache Felder gilt:\\
$\Rightarrow$Bewegungsgleichung
\begin{align}
m\ddot{\vec{x}}=e\vec{E}=e\vec{E}_0 \cos(\omega t -\alpha)\\
\ddot{\vec{D}}=\frac{e^2}{m}\vec{E}\\
\Rightarrow \mathrm{d}I=\frac{1}{4\pi c^3}\left|
\ddot{\vec{D}}\times\vec{n}' \right|^2 \mathrm{\Omega}
\end{align}
einlaufende Welle normiert auf $\vec{S}=\frac{c}{4\pi}|\vec{E}|^2\vec{n}$\\
$\Rightarrow$Streuquereschnitt\\
\begin{align}
\mathrm{d}\sigma = \left(\frac{e^2}{mc^2} \right)^2 \sin^2(\theta)
\mathrm{d}\Omega
\end{align}
mit dem Streuwinkel $\theta=\angle(\vec{E},\vec{n}')$ folgt der\\
totaler Streuquerschnitt
\begin{align}
\sigma=\int\mathrm{d}\sigma=\left(\frac{e^2}{mc^2} \right)^2\int_{-1}^{+1}
\mathrm{d}\cos\theta \int_0^{2\pi}\mathrm{d}\phi \sin^2(\theta)
\end{align}

Thomson-Streuquerschnitt
\begin{align}
\sigma_{\text{Th}}=\frac{8\pi}{3}\left( \frac{e^2}{mc^2} \right)\\
\end{align}
Das interessante an dieser Beobachtung ist, dass dieses Ergebnis sich für die
entsprechenden Annahmen auch aus der Qantenmechanik und der Quantenfeldtheorie
herleiten lässt.\\
klassischer Elektronenradius:\\
\begin{align}
r_e=\frac{e^2}{mc^2}=3*10^{-15}\mathrm{m}
\end{align}
Vorraussetzung: lineare Polarisation
%XXXX einfügen Grafik Z5
$\vec{E}$ fest, $\vec{E}=\vec{e}E$ mit Einheitsvektor $\vec{e}$\\
Praktische Bedeutung: unpolarisiert\\
$\longrightarrow$ $\vec{e}$ beliebig/zufällig orientiert mit Nebenbedingung
$\vec{e}\perp\vec{n}$\\
$\longrightarrow$ Mittelung über $\vec{e}$ von $\left( |\vec{e}\times\vec{n}'|^2
\right)$
\begin{align}
\sin^2(\theta)&=1-\cos^2(\theta)\\
&=1-(\vec{e}\vec{n}')\\
&=1-e_i {n_i}'e_j {n_j}'\\
&=1-e_ie_j {n_i}'{n_j}'
\end{align}
mit der Mittelung über die Polarisation ergibt sich
\begin{align}
\overline{\sin^2(\theta)}=1-{n_i}'{n_j}' \overline{e_ie_j}
\end{align}
\begin{enumerate}
  \item $e_ie_j$ ist ein symmetrischer Tensor \\ $\Rightarrow\overline{e_ie_j}$
  \item $\overline{e_ie_j}$ hängt nur von $n_i$ ab
 	\begin{align}
		\overline{e_ie_j}=a\delta_{ij}+bn_in_j
	\end{align}
 \item $\mathrm{Sp}e_ie_j=\vec{e}^2=1
 \Rightarrow\overline{\mathrm{Sp}e_ie_j}=3a+b=1$
 \item $\vec{e}\perp\vec{n}$\\ $\Rightarrow \overline{n_ie_ie_j}=0
 =n_i(a\delta_{ij}+bn_in_j)=an_j+bn_j$ \\ $\Rightarrow a+b=0 \Rightarrow
 a=-b=\frac{1}{2}$
\end{enumerate}
\begin{align}
\overline{e_ie_j}&=\frac{1}{2}\left(\delta_{ij}-n_in_j \right)\\
\overline{\sin^2\theta}&=1-{n_i}'{n_j}'\frac{1}{2}\left(\delta_{ij}-n_in_j
\right)\\
&=\frac{1}{2}+\frac{1}{2}\underbrace{\left( \vec{n}\vec{n}'
\right)^2}_{\cos\theta}\\
\Rightarrow \mathrm{d}\sigma&\sim1+\cos^2\theta
\end{align}
Was eine der Kugelflächenfunktionen ist.
\paragraph{Beipiel einer Ladung auf einer Kreisbahn}~\\
Teilchenbeschleuniger\\
Synchrotronstrahlung\\
Ladung $e$, Masse $m$, Magnetfeld $B$\\
Damit lässt sich der Bahnradius errechnen: $r=\frac{mcv}{eB}\gamma$ mit
$\gamma=\frac{1}{\sqrt{1-\beta^2}}$\\
Kreisfrequenz
$\omega_B=\frac{v}{r}=\frac{eB}{mc}\sqrt{1-\beta^2}=\frac{E}{mc^2}$\\
Nach langer und gewissenhafter Rechnung lässt sich erhalten:\\
Gesamtintensität: 
$I=\frac{2}{3}\frac{e^4B^2}{(mc^2)^2}\frac{1}{c}\left(v\gamma\right)^2$\\
Winkelverteilung: $\frac{\mathrm{d}I/\mathrm{d}\Omega\
(\parallel)}{\mathrm{d}I/\mathrm{d}\Omega\
(\perp)}=\frac{4+3\beta^2}{8(1-\beta^2)^{5/2}}$\\
Damit gilt für $\beta\rightarrow0$ gilt $\frac{\mathrm{d}I\
(\parallel)}{\mathrm{d}I \ (\perp)}=\frac{1}{2}$\\
und für $\beta\rightarrow1$ gilt $\frac{\mathrm{d}I\
(\parallel)}{\mathrm{d}I \ (\perp)}\infty$ und damit ist die Gesamtintensität in
der Beschleunigerebene
\begin{align}
\vec{A}(\omega)&=\frac{e^{ikr}}{cr}\int_{\text{Bahnkurve}}\mathrm{d}\vec{x}_0
e^{i(\omega t-\vec{k}\vec{x}_0)}\\
\rightarrow \text{Integrale vom Typ} &\int_0^{2\pi}\mathrm{d}\phi*\sin(\phi)
e^{ik(\phi-\beta \cos(\theta)\sin(\theta))}\\&\text{  (Besselfunktionen)}
\end{align}

\section{Feldtheorie der Gravitation - Allgemeine Relativitätsthorie}
Lehrbücher
\begin{itemize}
  \item Theoretische Physik 5 Scheck (3. Ausgabe)
  \item Landau Lipschitz T.II 
\end{itemize}

\subsection{Phänomenologie Newtonsche Gravitation}
\begin{align}
F_G&=-G\frac{m_1*m_2}{r^2}\\
F_C&=\pm\kappa\frac{e_1*e_2}{r^2} \ \ \ \left( \kappa=\frac{1}{4\pi\epsilon_0}
\right)\\
G&=\text{Graavitationskonstante}
&=6,67*10^{-11}\mathrm{m}^3 \mathrm{kg}^{-1} \mathrm{s}^{-2}
\end{align}
vergleiche beide Kräfte für ein Proton: $\frac{F_G}{F_C}\approx10^{-36}$\\
Potential $\phi_G=\frac{GM}{r}$\\
für \underline{Quelle} der \underline{Masse} $M$\\
Die Bewegungsgleichungen würde man dann aus den entsprechenden Lagrangefunktnen
gewinnen.
\begin{align}
L&=T-V\\
&=\frac{1}{2}m_Tv^2-m_S\phi_G
\end{align}
Im Vergleich zur elektrostatischen Wechselwirkung wirkt dies naiv. Aus der
Elektrostatik ergibt sich.
\begin{align}
L_{EM}&=\frac{1}{2}m_Tv^2-e\phi_{EM}
\end{align}
Wobei $m_S$ und $e$ hier als Kopplungskonstanten fungieren.
Man muss beachten, dass auch der Geschwindigkeitsterm ein $m_T$ enthält, deren
Gleichheit mit dem $m_S$ der Kopplung mit dem Feld nicht trivial ist. Das $m_T$
des Geschwindigkeitsterms besagt wie stark eine wirkende Kraft in der Lage ist ein
Teilchen zu beschleunigen, es trifft eine Aussage über die Trägheit der Masse.
Das $m_S$ der Kopplung wiederum besagt wie stark das Teilchen in der Lage ist
das Feld zu sehen. Damit nennen wir $m_T$ die träge Masse und $m_S$ die schwere
Masse.\\
Aus experimentellen Beobachtungen wurde gefolgert $m_T=m_S$. Eine wichtige
Folgerung daraus ergibt sich, wenn man die aus der Lagrangegleichung folgende
Bewegungsgleichung aufschreibt.
\begin{align}
m_T \dot{\vec{v}}=-m_S \vec{\nabla}\phi_G
\Rightarrow \dot{\vec{v}}=- \vec{\nabla}\phi_G
\end{align}
Dies ist eine nicht triviale Folgerung, da in dieser Bewegungsgleichung die Masse
des bewegten Teilchens keine Rolle spielt. Dies wird als Äquivalenzprinzip der
Allgemeinen Relativitätstheorie bezeichnet. Diese erste Formulierung stellt das
schwache Äquivalenzprinzip der Allgemeinen Relativitätstheorie dar:\\
%\center{Die Träge und die Schwere Masse ist gleich.}
%XXXX change that that it works
\\
\paragraph{nun ein Beispiel:}~\\
Betrachte ein System von Massen in einem homogenenem Gravitatinsfeld.
Wenn man massen in ein homogenes Gravitationsfeld setzt so werden diese durch
das anliegende Gravitationsfeld und Wechselwirkungskräfte aufeinander
$\vec{F}_{ij}$
\begin{align}
m_i \dot{\vec{v}}_i&=-m_i\vec{\nabla}\phi+\sum_{j=1,i\neq j}^N\vec{F}_{ij}\\
m_i (\dot{\vec{v}}_i+\vec{\nabla}\phi)&=\sum_{j=1,i\neq j}^N\vec{F}_{ij}\\
\end{align}
Führt man nun eine Koordinatentransformation durch, so dass
\begin{align}
\dot{\vec{v}}{}'&=\dot{\vec{v}}_i+\vec{\nabla}\phi\\
\vec{x}_i{}'&=\vec{x}_i+\frac{1}{2}\vec{\nabla}\phi t^2\\
\Rightarrow m_i\vec{v}_i{}'&=\sum_{j=1,i\neq j}^N\vec{F}_{ij}\\
\end{align}
d.h. Bewegungsgleichungen enthalten das Gravitationsfeld \underline{nicht}. Das
neue Koordinatensystem ist jedoch kein Inertialsystem, das durch eine
Lorentz- oder Galileo-Transformation erreichbar wäre, da dieses
Koordinatensystem beschleunigt ist.\\
Daraus folgt das sogenannte starke Äquivalenzprinzip der allgemeinen
Relativitätstheorie.\\
%XXXX center it
Ein Gravitationsfeld ist äquivalent zu einer Transformation in ein
Nicht- Inertialsystem.\\
"lokal": in einem räumlich eingeschränkten Gebiet, das klein genug sein sollte,
so dass das Gravitationsfeld homogen erscheint (Frage der Genauigkeit).
Solche Gravitationsfelder, die durch eine Transformation \underline{überall}
beseitigt werden können bezeichnet man als scheinbare Gravitationsfelder.
Während sogenannte wahre Gravitationsfelder sich nur lokal durch eine
Transformation beseitigen lassen.\\
Die Allgemeine Relativitätstheorie ist damit eine Theorie der allgemeinen
Koordinatentransformationen und damit eine Ausformulierung der Geometrie von
Raum und Zeit.\\ 
Ein Beispiel dafür wäre ein rotierendes Koordinatensystem:
\begin{align}
x&=x'\cos(\omega t)-y'\sin(\omega t)\\
y&=x'\sin(\omega t)+y'\cos(\omega t)\\
z&=z'
\end{align}
Innerhalb der Speziellen Relativitätstheorie müsste man den 4-Abstand
aufschreiben. 
\begin{align}
\mathrm{d}s^2&=c^2\mathrm{d}t^2-\mathrm{d}x^2-\mathrm{d}y^2-\mathrm{d}z^2\\
\Rightarrow
\mathrm{d}s^2&=c^2\mathrm{d}t^2-\mathrm{d}x'{}^2-\mathrm{d}y'{}^2-
\mathrm{d}z'{}^2+2\omega y'\mathrm{d}x'\mathrm{d}t -2\omega x'
\mathrm{d}y'\mathrm{d}t
\end{align}
Im Allgemeinen gilt damit:
\begin{align}
\mathrm{d}s^2=g_{\mu\nu}\mathrm{d}x^\mu\mathrm{d}x^\nu
\end{align}
Was wir zu beginn ja bereits gesamt haben, allerdings kann bei allgemeinen
Transformationen der Metrische Tensor $g_{\mu\nu}$ eine vollbesetzte Matrix sein
deren Komponenten Orts oder Zeitabhängig sein können. Es gilt also:
\begin{itemize}
  \item $g_{\mu\nu}=g_{\mu\nu}(x)$ also orts und zeitabhängig
  \item $g_{\mu\nu}$ ist nicht diagonal
  \item $g_{\mu\nu}=g_{\nu\mu}$
\end{itemize}
$\rightarrow$ 10 unabhängige Funktionen, die zusammen $g_{\mu\nu}$ definieren.\\
$\rightarrow$ Gravitationsfeld wird also durch den Metrischen Tensor
$g_{\mu\nu}$ beschrieben.\\
$\rightarrow$ Gravitation ist Geometrie der Raum-Zeit
für Inertialsystem\\
Normalkoordinaten $\xi^\mu$ mit $\mathrm{d}s^2=\eta_{\mu\nu}\mathrm{d}
\xi^\mu\mathrm{d}\xi^\nu$ mit der Minkowski-Metrik
$\eta_{\mu\nu}=\text{diag}(1,-1,-1,-1)$\\
Äquivalenzprinzip (präzise Formulierung)\\
In jedem Punkt der Raum-Zeit $x_0$ kann man ein Bezugssystem konstruieren, so
dass
\begin{align}
g_{\mu\nu}(x_0)=\eta_{\mu\nu}, \ \ \ \frac{\partial g_{\mu\nu}(x_0)}{\partial
x^\alpha}\Big|_{x_0}=0
\end{align}
mit $\eta_{\mu\nu}=\text{diag}(1,-1,-1,-1)$\\
Das mathematische Konstrukt das eine Herleitung dieser Aussage ermöglicht ist
die Differentialgeometrie.\\
Eine Metrik (ene Raum-Zeit) heißt flach, wenn man eine Koordinatentransforation
finden kann, so dass $g_{\mu\nu}=\eta_{\mu\nu}$ überall gilt.\\
Eine Metrik (ene Raum-Zeit) heißt gekrümmt, falls keine
Koordinatentransforation existiert, so dass
$g_{\mu\nu}=\eta_{\mu\nu}$ überall gilt.\\
Betrachte nun ein rotierendes Bezugssystem innerhalb der SRT.\\
Lorentz-Kontraktion.
%XXXX Füge Grafik Z6 ein
Umfang U\\
Radius R\\
\begin{align}
\frac{U}{R}\neq2\pi
\end{align}
$\rightarrow$ nicht-euklidische Geometrie\\
Damit ist das Ziel der allgemeinen Relativitätstheorie allgemeine
Koordinatentransformationen zu finden, die in der Lage sind einen allgemeinen
Satz von Gleichungen zu liefern, die in jedem der Koordinatensysteme Gültigkeit
besitzen, also formgleich in allen physikalische relevanten Koordinatensystemen
sind.\\
Dies bezeichnete eine Kovariante Formulierung der Physikalischen Gesetze.
\subsection{Allgemeine Koordinatentransformationen}
\begin{itemize}
  \item Normalkoordinaten $\xi^\mu$ mit $\mathrm{d}s^2=\eta_{\mu\nu}\mathrm{d}
\xi^\mu\mathrm{d}\xi^\nu$ 
\item allgemeine Koordinaten $\xi^\mu\rightarrow x^\mu=x^\mu(\eta)$ die
beliebige in allen Komponenten differenzierbare und umkehrbare Funktionen
darstellen sollen.   
\end{itemize}
Differenzierbar: $\mathrm{d}x^\mu=\frac{\partial x^\mu}{\partial
\xi^\nu}\mathrm{d}\xi^\nu$\\
Analog: SRT Lorentz-Transformation linear
\begin{align}
\mathrm{d}x^\mu=\Lambda^\mu{}_\nu \mathrm{d}\xi^\nu
\end{align}
\subsubsection{Wiederholung}
lokale Normalkoordinaten: $\xi^\mu$ und
$\eta_{\mu\nu}=\text{diag}(1,-1,-1,-1)$\\
Transformation zu allgemeinen Koordinaten:
\begin{align}
x^\mu(\xi) \ \ (\text{bzw.} \eta^\mu(x))\\
\mathrm{d}x^\mu=\frac{\partial x^\mu}{\partial \xi^\nu} \mathrm{d}\xi^\nu 
\end{align}
Wobei $\frac{\partial x^\mu}{\partial \xi^\nu}$ Funktionen von $\xi$ sind und
damit die Transformationsmatrix $\Lambda^\mu{}_\nu$ definieren.\\
Transformation zischen allgemeinen Koordinaten
\begin{align}
x^\mu \rightarrow x^\mu{}'(x) \ \ (\text{umkehrbar})\\
\mathrm{d}x^\mu=\frac{\partial x^\mu}{\partial x^\nu{}'} \mathrm{d}x^\nu{}' 
\end{align}
Definition kontravarianter Vektor $A^\mu$:
\begin{align}
A\rightarrow A',\ \ \ A^\mu=\frac{\partial x^\mu}{\partial x^\nu{}'}A^\nu{}'
\end{align}
über den Gradienten:
Definition kovarianter Vektor $A_\mu$:
\begin{align}
\frac{\partial \phi}{\partial x^\mu}=\frac{\partial \phi}{\partial
x^\nu{}'}\frac{\partial x^\nu{}'}{\partial x^\mu{}}
\end{align}
\begin{align}
A\rightarrow A',\ \ \ A_\mu{}'=\frac{\partial x^\nu{}'}{\partial x^\mu}A_\nu
\end{align}
$\Rightarrow$ Tensoren und Skalarprodukte\\
Metrik
\begin{align}
\mathrm{d}s^2&=\eta_{\mu\nu}\mathrm{d}\xi^\mu\mathrm{d}\xi^\nu\\
&=\left( \eta_{\mu\nu} \frac{\partial \xi^\mu}{\partial x^\rho}\frac{\partial
\xi^\nu}{\partial x^\sigma} \right)\mathrm{d}x^\rho\mathrm{d}x^\sigma\\
&=g_{\mu\nu}\mathrm{d}x^\mu\mathrm{d}x^\nu\\
g_{\rho\sigma}&=\eta_{\mu\nu} \frac{\partial \xi^\mu}{\partial
x^\rho}\frac{\partial \xi^\nu}{\partial x^\sigma}
\end{align}
Bewegungsgleichungen für Kräftefreie Teilchen im Inertialsystem: 
\begin{align}
\frac{\mathrm{d}^2\xi^\mu}{\mathrm{d}s^2}=0
\end{align}
Zu allgmeinen Koordinaten Transformiert ergibt sich:
\begin{align}
\frac{\mathrm{d}}{\mathrm{d}s}\frac{\mathrm{d}\xi^\mu}{\mathrm{d}s}&=
\frac{\mathrm{d}}{\mathrm{d}s}\left( \frac{\mathrm{d}\xi^\mu}{\mathrm{d}x^\nu}
\frac{\mathrm{d}x^\nu}{\mathrm{d}s} \right)\\
&=\frac{\partial \xi^\mu}{\partial x^\nu}
\frac{\mathrm{d}^2x^\nu}{\mathrm{d}s^2}+\frac{\partial^2 \xi^\mu}{\partial-
x^\rho \partial x^\nu}
\frac{\mathrm{d}x^\nu}{\mathrm{d}s} \frac{\mathrm{d}x^\rho}{\mathrm{d}s}\\
0&=\frac{\mathrm{d}^2x^\lambda}{\mathrm{d}s^2}+\underbrace{\frac{\partial
x^\lambda}{\partial \xi^\mu}\frac{\partial^2 \xi^\mu}{\partial x^\mu
\partial x^\rho}}_{\Gamma^\lambda{}_{\nu\rho}}
\frac{\mathrm{d}x^\nu}{\mathrm{d}s} \frac{\mathrm{d}x^\rho}{\mathrm{d}s}\\
\Gamma^\lambda_{\nu\rho}&=\frac{\partial
x^\lambda}{\partial \xi^\mu}\frac{\partial^2 \xi^\mu}{\partial x^\mu
\partial x^\rho} \ \ \ {}^{\text{Christoffel-Symbole}\leftarrow
\text{ aus } g_{\mu\nu},\partial g_{\mu\nu}}_{\text{affine connection}}
\end{align}
$\rightarrow$ Bewegungsgleichung in allgemeinen Koordinaten
\begin{align}
\frac{\mathrm{d} u^\lambda}{\mathrm{d}s}=-\Gamma^\lambda_{\nu\rho}u^\nu u^\rho\\
\end{align}
Damit lässt sich $\Gamma$ physikalisch als Gravitationsfeld interpretieren oder
als Ausformung der Gravitationskraft benennen. Damit würde $g_{\mu\nu}$ das
Gravitationspotential beschreiben.\\
$\Gamma=0$ in jedem Inertialsystem\\
$\Gamma^\lambda{}_{\nu\rho}$ ist kein Tensor\\
Bewegungsgleichung $\partial u^\mu=0 \longrightarrow$ kovariante Ableitung:
$\mathrm{D}_\nu u_\rho$\\
Vektor $U^\mu$ in Normalkoordinaten
$\overset{\text{Ableitung}}{\longrightarrow}
Q^\mu{}_\nu=\frac{\partial}{\partial \xi^\mu} U^\mu$\\
$\downarrow \text{Transformation zu allgemeinen
Koordinaten}$$\downarrow\text{zu allgemeinen Koordinaten}$\\
Vektor $V^\mu=\frac{\partial x^\mu}{\partial \xi^\sigma}
U^\sigma\overset{?}{\underset{\text{Ableitung}}{\longrightarrow}}T^\mu{}_\lambda=
\frac{\partial x^\mu}{\partial \xi^\sigma}\frac{\partial x^\tau}{\partial
\xi^\lambda} Q^\sigma{}_\tau$\\
nachrechnen:
\begin{align}
\frac{\partial V^\mu}{\partial x^\lambda}&=\frac{\partial x^\mu}{\partial
\xi^\sigma}\frac{\partial U^\sigma}{\partial x^\lambda} +
\frac{\partial}{\partial x^\lambda} \left( \frac{\partial x^\mu}{\partial
\xi^\sigma} \right) U^\sigma\\
&=\frac{\partial x^\mu}{\partial \xi^\sigma}\frac{\partial
\xi^\tau}{\partial x^\lambda}\frac{\partial U^\sigma}{\partial
\xi^\tau}+\ldots\\
&=T^\mu{}_\lambda-\Gamma^\mu_{\lambda\nu}V^\nu
\end{align}
gewöhnliche Ableitung $\rightarrow$ kovariante Ableitung\\
\begin{align}
\frac{\mathrm{D}V^\mu}{\mathrm{D}x^\lambda}=\frac{\partial V^\mu}{\partial
x^\lambda}+ \Gamma^\mu_{\lambda\nu}V^\nu
\end{align}
Diese ist wieder ein Tensor (Prinziep der Kovarianz)\\
Kräftefreies Teilchen:\\
Ableitung entlang einer Kurve $x^\mu(\tau)$\\
Vektor $A^\mu(\tau)$\\
Transormationsverhalten $A^\mu{}'(\tau)=\frac{\partial x^\mu{}'}{\partial
x^\nu}(\tau) A^\nu(\tau)$ (für die Transformation $x\rightarrow x'$)\\
Ableitung 
\begin{align}
\frac{\mathrm{d} A^\mu{}'}{\mathrm{d}\tau}&=\frac{\partial
x^\mu{}'}{\partial x^\nu} \frac{\mathrm{d} A^\mu}{\mathrm{d}\tau}+
\frac{\partial^2 x^\mu{}'}{\partial x^\lambda \partial x^\nu} \frac{\partial
x^\lambda}{\partial \tau}A^\nu\\
\rightarrow \frac{\mathrm{D} A^\mu}{\mathrm{d}\tau}&=\frac{\mathrm{d}
A^\mu}{\mathrm{d}\tau}+\Gamma^\mu_{\nu\lambda}\frac{\mathrm{d}
x^\mu}{\mathrm{d}\tau}A^\lambda
\end{align}
Damit ergibt sich als Form der Bewegungsgleichung:
\begin{align}
\frac{\mathrm{D}A^\mu}{\mathrm{D}\tau}=0
\end{align}
Die in jedem Koordinatensystem gültig ist.
\paragraph{Beispiel für einen zweidimensionalen gekrümmten Raum}
Wir nehmen als Beispiel eine Kugeloberfläche:
%Füge Bild z7 ein
Wenn man annimmt, dass der Vektor bei einer kräftefreien Bewegung seine Richtung
nicht ändert. Dann kann man abhängig vom Weg mit verschiedenen Richtungen der
Vektoren am Nordpol enden. Die ist Paralleltransport entlang von Kurven auf der
Kugeloberfläche.
\paragraph{Paralleltransport}
Transport von Vektoren entlang eines geschlossenen Weges führt zur Änderung des
Vektors bei Bewegung durch einen gekrümmten Raum
\begin{align}
\mathrm{d}A^\mu &= - \Gamma^\mu_{\nu\lambda} A^\nu \mathrm{d}x^\lambda\\
\Delta A^\mu &= \oint \mathrm{d} A^\mu\underset{Krümmung}{\neq0}
\end{align}
$\rightarrow$ $\Delta A^\mu$ ist proportional zur Fläche $\sim
\mathrm{d}f^{\rho\sigma}$. 
\begin{align}
\mathrm{d}A^\mu=-\frac{1}{2}R^\mu{}_{\rho\nu\sigma}A^\nu f^{\rho\sigma} 
\end{align}
mit dem Krümmungstensor $R^\mu{}_{\rho\nu\sigma}$ (der auch die
Tensoreigenschaften besitzt). Man kann dies aus den Größen
$(\Gamma)^2$ und $\left( \frac{\partial \Gamma}{\partial x} \right)$ oder aus
$\frac{\partial^2 g}{\partial x\partial x}$ und $g\Gamma\Gamma$ herleiten, was
allerdings hässlich wird.\\
Ricci-Tensor $R_{\mu\nu}=g^{\rho\sigma}R_{\mu\rho\nu\sigma}$, der den Vorteil
hat symmetrisch zu sein ($R_{\mu\nu}=R_{\nu\mu}$)\\
Und das Krümmungsskalar $R=g^{\mu\nu}R_{\mu\nu}$\\
Suche kovariante Differentialgleichungen für $g_{\mu\nu}$
oder davon abgeleitete Größen, die ausreichend einfach sind um lösbar zu sein.\\
Als Quelle des Feldes fungiert hier die Masse, die jedoch alleine nicht
kovariant ist. Diese muss erweitert werden wobei man zum Energie-Impuls-Tensor
$T^{\mu\nu}$ gelangt. \\
(zusätzlich müssen für kleine Geschwindigkeiten und schwache Felder das
Newtonsche Gravitationsgesetz als Grenzfall heraus kommen.)\\
Es ergeben sich die Einsteingleichungen:
\begin{align}
R_{\mu\nu}-\frac{1}{2}g_{\mu\nu}R+\lambda g_{\mu\nu}=\frac{8\pi
G}{c^4}T_{\mu\nu}
\end{align}


\end{document}